
\Figure{!htb}{1}{Filtres.png}{Filtres \cite{merdrignac_systeme_2015}}
\Figure{!htb}{1}{classification.png}{classification \cite{merdrignac_systeme_2015}}
\Figure{!htb}{1}{LPWAN.png}{LPWAN connectivity}
\Figure{!htb}{1}{sdn-wise.png}{LPWAN connectivity}
\Figure{!htb}{1}{interoperability-stack.jpg}{Interoperability}
\Figure{!htb}{1}{stat_interoperability.png}{Key barriers in adopting the Industrial Internet \footfullcite{industrialinternetofthings_executive_}}
\Figure{!htb}{1}{wsn-IoT.png}{wsn-IoT}

\begin{itemize}
	\item Network selection
	\begin{itemize}
		\item MADM
		\begin{itemize}
			\item Ranking methods
			\item Ranking \& weighted methods
		\end{itemize}
		
		\item Game theory
		\begin{itemize}
			\item Users vs users
			\item Users vs networks
			\item Networks vs network
		\end{itemize}
	
		\item Fuzzy logic
		\begin{itemize}
			\item as a score method
			\item another theory
		\end{itemize}
			
		\item Utility function
		\begin{itemize}
			\item 1
			\item 2
		\end{itemize}
	\end{itemize}
\end{itemize}



\begin{table}[h!]
\scriptsize
	\begin{tabulary}{\textwidth}{L|L}
	\multirow{3}{*}{Naïve modes}           & Instantaneous  \\
	\                                      & Hist. average  \\
	\                                      & Clustering     \\\hline
	\multirow{9}{*}{Parametric models}     & Rarely used    \\
	\                                      & Traffic Models \\
	\                                      & Time Series    \\
	\                                      & Linear regression \\
	\                                      & ARIMA \\
	\                                      & Kalman filtering \\
	\                                      & ATHENA \\
	\                                      & SETAR \\
	\                                      & Gaussian Maximum Likelihood \\\hline

	\multirow{6}{*}{Non-Parametric models} & k-Nearest Neighbor           \\
	\                                      & Locally Weighted Regression \\
	\                                      & Fuzzy Logic    \\
	\                                      & Bayes Network  \\
	\                                      & Neural Network \\
	\                                      & Include temporal/spatial patterns \\
	\end{tabulary}
	\caption{\label{tab:models} Taxonomy of prediction models \cite{_short_2007}}
\end{table}

\begin{table}[h!]
\scriptsize
	\begin{tabular}{l|l|l|l}
	\textbf{Parameters}                                & \textbf{Parameters}     & \textbf{Type} & \textbf{Expected as}  \\\hline
	\multirow{4}{*}{\textbf{Network conditions}}       & network load            & Dynamic       & Minimized  \\
	\                                                  & network coverage        & Static        & Fixed      \\
	\                                                  & network connection time & Dynamic       & Minimized  \\
	\                                                  & available bandwidth     & Dynamic       & Minimized  \\\hline
	\multirow{5}{*}{\textbf{Application requirements}} & throughput              & Dynamic       & Minimized  \\
	\                                                  & delay                   & Dynamic       & Minimized  \\
	\                                                  & jitter                  & Dynamic       & Minimized  \\
	\                                                  & PLR                     & Dynamic       & Minimized  \\
	\                                                  & energy consumption      & Dynamic       & Minimized  \\\hline
	\multirow{2}{*}{\textbf{User preferences}}         & budget                  & Static        & Fixed      \\
	\                                                  & cost                    & Static        & Fixed      \\
	\                                                  & design                  &               &            \\\hline
	\multirow{2}{*}{\textbf{Mobile equipment}}         & energy                  & Dynamic       & Fixed      \\
	\                                                  & mobility                & Dynamic       & Fixed      \\\hline
	\end{tabular}
	\caption{\label{tab:network_selection} Network selection inputs and classification of parameters \cite{bendaoud_network_2019}}
\end{table}


QoS parameters \cite{meshinchi_qosaware_2018} \cite{chowdhury_survey_2018}
	\begin{itemize}
		\item Application layer
			\begin{itemize}
				\item Service time
				\item Service availability
				\item Service cost
				\item Service reliability
			\end{itemize}
			
		\item Network layers
				\begin{itemize}
					\item Bandwidth
					\item Packet loss
					\item Jitter
					\item Delay
				\end{itemize}
				
		\item Sensing layer
			\begin{itemize}
				\item Data accuracy
				\item Data collection delay
				\item Sampling rate
				\item WSN lifetime
				\item WSN coverage
			\end{itemize}

		\item Sensing layer
			\begin{itemize}
				\item Information accuracy
					\begin{itemize}
						\item Data accuracy
						\item Sensing time accuracy
						\item Spatial accuracy
						\item Reduce data redundancy
						\item Data packaging
					\end{itemize}
					
				\item Energy compsumption
					\begin{itemize}
						\item Sleep management
						\item Life time management
					\end{itemize}
					
				\item Coverage
					\begin{itemize}
						\item Sensing area
					\end{itemize}
			\end{itemize}
	\end{itemize}

\begin{table}[h!]
\scriptsize
	\begin{tabulary}{\textwidth}{L|L|L|L|L|L|L}
	\bf{Phy protocol}     & \bf{IEEE 802.15.4} & \bf{BLE}      & \bf{EPCglobal} & \bf{Z-Wave}              & \bf{LTE-M}           & \bf{ZigBee} \\\hline
	\bf{Standard Body}    &                    & IEEE 802.15.1 &                &                          &                      & IEEE 802.15.4, ZigBee Alliance \\\hline
	\bf{Radio band (MHz)} & 868/915/2400       & 2400          & 860-960        & 868/908/2400             & 700-900              & \\\hline
	\bf{MAC address}      & TDMA, CSMA/CA      & TDMA          & ALOHA          & CSMA/CA                  & OFDMA                & \\\hline
	\bf{Data rate (bps)}  & 20/40/250 K        & 1024K         & varies 5-640K  & 40K                      & 1G (up), 500M (down) & \\\hline
	\bf{Throughput}       &                    &               &                & 9.6, 40, 200kbps         &                      & \\\hline
	\bf{Scalability ???}  & 65K nodes          & 5917 slaves   & -              & 232 nodes                & -                    & \\\hline
	\bf{Range}            & 10-20m             & 10-100m       &                &                          &                      & \\\hline
	\bf{Addressing}       & 8|16bit            & 16bit         &                &                          &                      & \\\hline
	\end{tabulary}
	\caption{\label{tab:IoT_cloud} IoT cloud platforms and their characteristics \cite{al-fuqaha_internet_24}}
\end{table}

%\changefontsizes{6pt}
\begin{table}[h!]
\scriptsize
	\begin{tabulary}{\textwidth}{L|L|L|L|L}
	\bf{Characteristics}                           & \bf{6LoWPAN}              & \bf{LoRaWAN}                    & \bf{SigFox}                           & \bf{Narrowband}\\\hline
	\bf{Standar body}                              &                           & LoRa Alliance                   &                                       & 3GPP          \\\hline
	\bf{TX Active Power @ 3V}                      &                           &                                 &                                       &               \\\hline
	\bf{Frequency band}                            & 902-929                   & 902-928                         & 902                                   &               \\
	\bf{\footnotesize{(MHz)}}                      & 868-868.6                 & 863-870 and 434                 & 868                                   &               \\\hline
	\bf{Number of channels}                        & 0016 for 2400             & 80             for 915          & 25                                    &               \\
	\bf{\footnotesize{(channels for MHz)}}         & 0010 for 915              & 10             for 868 and 780  &                                       &               \\
	\                                              & 0001 for 868.3            &                                 &                                       &               \\\hline
	\bf{Channel bandwidth}                         & 0005 for 2400             & 0.125 and 0.50 for 915          & 0.0001-0.0012                         &               \\
	\bf{\footnotesize{(MHz)}}                      & 0002 for 915              & 0.125 and 0.25 for 868 and 780  &                                       &               \\
	\                                              & 0600 for 868.3            &                                 &                                       &               \\\hline
	\bf{Maximum data rate}                         & 0250 for 2400             & 0.00098-0.0219 for 915          & 0.1-0.6                               &               \\
	\bf{\footnotesize{(kbps for MHz)}}             & 0040 for 915              & 0.250-0.05     for 868 and 780  &                                       &               \\
	\                                              & 0020 for 868.3            &                                 &                                       &               \\\hline
	\bf{Channel modulation}                        & QPSK for 2400             & LoRa           for 915          & BPSK and GFSK                         &               \\
	\                                              & BPSK for 915              & LoRa and GFSK  for 868  and 780 &                                       &               \\
	\                                              & BPSK for 868.3            &                                 &                                       &               \\\hline
	\bf{Channel coding}                            & -085 for 2400             & -137                            & -137                                  &               \\
	\bf{\footnotesize{(dBm for MHz)}}              & -092 for 915              &                                 &                                       &               \\
	\                                              & -092 for 868.3            &                                 &                                       &               \\\hline
	\bf{Protocol data unit \footnotesize{(bytes)}} & 6+127                     & x + (19 to 250)                 & 12+ (0 to 12)                         &               \\\hline
	\bf{Channel coding}                            & Direct                    & CSS                             & Ultra                                 &               \\\hline
	\bf{Transmission range}                        & 10-100 m                  & 5-15 km                         & 10-50 km                              &               \\\hline
	\bf{Battery lifetime}                          & 1-2 years                 & <10 years                       & <10 years                             &               \\\hline
	\bf{Standard Body}                             & IETF                      &                                 &                                       &               \\\hline
	\bf{Security}                                  & Access Control List (ACL) &                                 &                                       &               \\\hline
	\bf{Uplink}                                    &                           &                                 & 100bps, 12 bytes/msg, max 140 msg/day &               \\\hline
	\bf{Downlink}                                  &                           &                                 & 8 bytes/msg, max 4 msg/day            &               \\\hline
	\bf{Scalability}                               &                           &                                 &                                       &               \\\hline
	\bf{Proprietary}                               &                           &                                 & \ok                                   &               \\\hline
	\bf{Cost}                                      &                           & High                            &                                       &               \\\hline
	
	\end{tabulary}
	\caption{\label{tab:LPWan_characteristics} LPWan Characteristics \cite{al-kashoash_comparison_2016}}
\end{table}
%\changefontsizes{7pt}
\begin{table}[h!]
\scriptsize
	\begin{tabulary}{\textwidth}{L|L|L|L|L|L}
	\bf{Feature}                     & \bf{Wi-Fi}                                            & \bf{802.11p}                                          & \bf{UMTS}                        & \textbf{LTE}                      & \textbf{LTE-A}                     \\\hline
	\bf{Channel width MHz}           & 20                                                    & 10                                                    & 5                                & 1.4, 3, 5, 10, 15, 20    & <100                 \\\hline
	\bf{Frequency band(s) GHz}       & 2.4 , 5.2                                             & 5.86-5.92                                             & 0.7-2.6                          & 0.7-2.69                 & 0.45-4.99                 \\\hline
	\bf{Bit rate Mb/s}               & 6-54                                                  & 3–27                                                  & 2                                & <300                & <1000                \\\hline
	\bf{Range km}                    & <0.1                                             & <1                                               & <10                         & <30                 & <30                  \\\hline
	\bf{Capacity}                    & Medium                                                & Medium                                                & \ko                              & \ok                      & \ok                       \\\hline
	\bf{Coverage}                    & Intermittent                                          & Intermittent                                          & Ubiquitous                       & Ubiquitous               & Ubiquitous                \\\hline
	\bf{Mobility support km/h}       & \ko                                                   & Medium                                                & \ok                              & <350                & <350                 \\\hline
	\bf{QoS support}                 & EDCA \scriptsize{Enhanced Distributed Channel Access} & EDCA \scriptsize{Enhanced Distributed Channel Access} & QoS classes and bearer selection & QCI and bearer selection & QCI and bearer selection  \\\hline
	\bf{Broadcast/multicast support} & Native broadcast                                      & Native broadcast                                      & Through MBMS                     & Through eMBMS            & Through eMBMS             \\\hline
	\bf{V2I support}                 & \ok                                                   & \ok                                                   & \ok                              & \ok                      & \ok                       \\\hline
	\bf{V2V support}                 & Native (ad hoc)                                       & Native (ad hoc)                                       & \ko                              & \ko                      & Through D2D               \\\hline
	\bf{Market penetration}          & \ok                                                   & \ko                                                   & \ok                              & \ok                      & \ok                       \\\hline
	\bf{Data rate}                   & <640 kbps                                        & 250 kbps                                              & 106–424 kbps                              & \ok                      & \ok                       \\\hline
	\end{tabulary}
	\caption{\label{tab:Tableppp} An example table.}
\end{table}


\begin{itemize}
	\item[\cite{qin_software_2014}] Many studies have identified \green{SDN} as a potential solution to the WSN challenges,
	as well as a model for \red{heterogeneous} integration.
	\item[\cite{qin_software_2014}] This \red{shortfall} can be resolved by using the \green{SDN approach.}
	\item[\cite{kobo_survey_2017}] \green{SDN} also enhances better control of \red{heterogeneous} network infrastructures.
	\item[\cite{kobo_survey_2017}] Anadiotis et al. define a \green{SDN operating system for IoT} that integrates SDN based WSN \textbf{(SDN-WISE)}.
		This experiment shows how \red{heterogeneity} between different kinds of SDN networks can be achieved.
	\item[\cite{kobo_survey_2017}] In cellular networks,
		OpenRoads presents an approach of introducing \green{SDN} based \red{heterogeneity} in wireless networks for operators.
	\item[\cite{ndiaye_software_2017}] There has been a plethora of (industrial) studies \green{synergising SDN in IoT}.
			The major characteristics of IoT are low latency,wireless access, mobility and \red{heterogeneity}.
	\item[\cite{ndiaye_software_2017}] Thus a bottom-up approach application of \green{SDN} to the realisation of \red{heterogeneous IoT} is suggested.
	\item[\cite{ndiaye_software_2017}] Perhaps a more complete IoT architecture is proposed,
			where the authors apply \green{SDN} principles in IoT \red{heterogeneous} networks.
	\item[\cite{bera_softwaredefined_2017}] it provides the \green{SDWSN} with a proper model of network management,
			especially considering the potential of \red{heterogeneity} in SDWSN.
	\item[\cite{bera_softwaredefined_2017}] We conjecture that the \green{SDN paradigm} is a good candidate to solve the \red{heterogeneity} in IoT.

\end{itemize}


\begin{table}[h!]
\scriptsize
	\begin{tabulary}{\columnwidth}{L|L|C|C|C|C|C}
	\textbf{Management architecture}                 & \textbf{Management feature}            & \textbf{Controller configuration} & \textbf{Traffic Control} & \textbf{Configuration and monitoring} & \textbf{Scapability and localization} & \textbf{Communication management}\\\hline
	\textbf{\cite{luo_sensor_2012} Sensor Open Flow} & SDN support protocol                   & Distributed                       & in/out-band              & \ok                                   & \ok                                   & \ok                              \\\hline
	\textbf{\cite{costanzo_software_2012} SDWN}          & Duty sycling, aggregation, routing     & Centralized                       & in-band                  & \ok                                   &                                       & \\\hline
	\textbf{\cite{galluccio_sdnwise_2015} SDN-WISE}  & Programming simplicity and aggregation & Distributed                       & in-band                  &                                       & \ok                                   & \\\hline
	\textbf{\cite{degante_smart_2014} Smart}        & Efficiency in resource allocation      & Distributed                       & in-band                  &                                       & \ok                                   & \\\hline
	\textbf{SDCSN}                                   & Network reliability and QoS            & Distributed                       & in-band                  &                                       & \ok                                   & \\\hline
	\textbf{TinySDN}                                 & In-band-traffic control                & Distributed                       & in-band                  &                                       & \ok                                   & \\\hline
	\textbf{Virtual Overlay}                         & Network flexibility                    & Distributed                       & in-band                  &                                       & \ok                                   & \\\hline
	\textbf{Context based}                           & Network scalability and performance    & Distributed                       & in-band                  &                                       & \ok                                   & \\\hline
	\textbf{CRLB}                                    & Node localization                      & Centralized                       & in-band                  &                                       &                                       & \\\hline
	\textbf{Multi-hope}                              & Traffic and energy control             & Centralized                       & in-band                  &                                       &                                       & \ok                              \\\hline
	\textbf{Tiny-SDN}                                & Network task measurement               & -                                 & in-band                  &                                       &                                       & \\
	\end{tabulary}
	\caption{\label{tab:Table} SDN-based network and topology management architectures. \cite{ndiaye_software_2017}}
\end{table}

\begin{table}
\scriptsize
	\begin{tabulary}{\textwidth}{L|C|C|C|C|C|C|C}
		\bf{Application protocol}                                                 & DDS                                     & CoAP                              & AMQP                              & MQTT                                & MQTT-SN & XMPP & HTTP\\\hline
		\bf{Service discovery}     & \multicolumn{3}{c}{mDNS}                & \multicolumn{4}{c}{DNS-SD}                                                                                                         \\
		\bf{Network layer}         & \multicolumn{7}{c}{RPL}                                                                                                                                                      \\
	%			\bf{Network layer}         & \multicolumn{3}{c}{6LoWPAN}             & \multicolumn{4}{c}{IPv4/IPv6}                                                                                                      \\
		\bf{Link layer}            & \multicolumn{7}{c}{IEEE 802.15.4}                                                                                                                                            \\
		\bf{Physical layer}        & \multicolumn{3}{c}{EPCglobal}     & \multicolumn{2}{c}{IEEE 802.15.4} & \multicolumn{2}{c}{Z-Wave}                                 \\\hline
	\end{tabulary}
	\caption{\label{tab:Tablej} Standardization efforts that support the IoT}
\end{table}


\begin{itemize}
	\item Constrained Application Protocol
	\item The IETF Constrained RESTful Environments
	\item CoAP is bound to UDP
%				\item Enable devices with low resources to use RESTful interactions
	\item CoAP can be divided into two sub-layers
		\begin{itemize}
			\item messaging sub-layer
			\item request/response sub-layer
			\begin{itemize}
				\item[a)] Confirmable. 
				\item[b)] Non-confirmable. 
				\item[c)] Piggybacked responses. 
				\item[d)] Separate response
			\end{itemize}
		\end{itemize}
	\item CoAP, as in HTTP, uses methods such as:
		\begin{itemize}
			\item GET, PUT, POST and DELETE to 
			\item Achieve, Create, Retrieve, Update and Delete
		\end{itemize}
		\begin{itemize}
			\item Ex: the GET method can be used by a server to inquire the client’s temperature
		\end{itemize}
\end{itemize}

\begin{table}[h!]
\scriptsize
	\begin{tabulary}{\textwidth}{|C|C|C|C|C|}
	0 1 & 2 3 & 4-7 & 8-15 & 16-31 \\\hline
	Ver & T & OC & CODE & Message ID \\\hline
	\multicolumn{5}{|c|}{Token} \\\hline
	\multicolumn{5}{|c|}{Options} \\\hline
	\multicolumn{5}{|c|}{Payload} \\\hline
	\end{tabulary}
	\caption{\label{tab:CoapPacket}CoAP message format.}
\end{table}

\hspace*{1.2cm}
\begin{minipage}{\textwidth}
\begin{itemize}
	\item[Ver:] is the version of CoAP
	\item[T:] is the type of Transaction
	\item[OC:] is Option count
	\item[Code:] represents the request method (1-10) or response code (40-255).
		\begin{itemize}
			\item Ex: the code for GET, POST, PUT, and DELETE is 1, 2, 3, and 4, respectively.
		\end{itemize}
	\item[Message ID:] is a unique identifier for matching the response.
\end{itemize}
\end{minipage}

%\begin{column}{0.6\textwidth}
	\begin{itemize}
		\item Message Queue Telemetry Transport
		\item Andy Stanford-Clark of IBM and Arlen Nipper of Arcom
			\begin{itemize}
				\item Standardized in 2013 at OASIS
			\end{itemize}
		\item MQTT uses the publish/subscribe pattern to provide transition flexibility and simplicity of implementation
		\item MQTT is built on top of the TCP protocol
		\item MQTT delivers messages through three levels of QoS
		\item Specifications
			\begin{itemize}
				\item MQTT v3.1 and MQTT-SN (MQTT-S or V1.2)
				\item MQTT v3.1 adds broker support for indexing topic names
			\end{itemize}
		\item The publisher acts as a generator of interesting data.
	\end{itemize}
%	\end{column}

\begin{table}[h!]
\scriptsize
	\begin{tabulary}{\textwidth}{|C|C|C|C|}
	0-3          & 4   & 5 6       & 7                   \\\hline
	Message type & DUP & QoS level & Retain              \\\hline
	\multicolumn{4}{|c|}{Remaining length}               \\\hline
	\multicolumn{4}{|c|}{Variable length header}         \\\hline
	\multicolumn{4}{|c|}{Variable length message payload}\\\hline
	\end{tabulary}
	\caption{\label{tab:MqttPacket}MQTT message format.}
\end{table}
\hspace*{1.2cm}
%			\addtolenght{\textwidth}{-3cm}
\begin{minipage}{\textwidth}
\begin{itemize}
\item[Message type:] CONNECT (1), CONNACK (2), PUBLISH (3), SUBSCRIBE (8) and so on
\item[DUP flag:] indicates that the massage is duplicated
\item[QoS Level:] identify the three levels of QoS for delivery assurance of Publish messages
\item[Retain field:] retain the last received Publish message and submit it to new subscribers as a first message
\end{itemize}
\end{minipage}
%			\addtolenght{\textwidth}{+3cm}

\begin{itemize}
	\item Extensible Messaging and Presence Protocol
	\item Developed by the Jabber open source community
	\item An IETF instant messaging standard used for:
		\begin{itemize}
			\item multi-party chatting, voice and telepresence
		\end{itemize}
	\item Connects a client to a server using a XML stanzas
	\item An XML stanza is divided into 3 components:
		\begin{itemize}
			\item message: fills the subject and body fields
			\item presence: notifies customers of status updates
			\item iq (info/query): pairs message senders and receivers
		\end{itemize}
	\item Message stanzas identify:
		\begin{itemize}
			\item the source (from) and destination (to) addresses
			\item types, and IDs of XMPP entities
		\end{itemize}
	\end{itemize}

\begin{itemize}
	\item Advanced Message Queuing Protocol
	\item Communications are handled by two main components
		\begin{itemize}
			\item exchanges: route the messages to appropriate queues.
			\item message queues: Messages can be stored in message queues and then be sent to receivers
		\end{itemize}
	\item It also supports the publish/subscribe communications.
	\item It defines a layer of messaging on top of its transport layer.
	\item AMQP defines two types of messages
		\begin{itemize}
			\item bare massages: supplied by the sender
			\item annotated messages: seen at the receiver
		\end{itemize}
	\item The header in this format conveys the delivery parameters:
		\begin{itemize}
			\item durability, priority, time to live, first acquirer \& delivery count.
		\end{itemize}
	\item AMQP frame format
		\begin{itemize}
			\item[Size] the frame size.
			\item[DOFF] the position of the body inside the frame.
			\item[Type] the format and purpose of the frame.
				\begin{itemize}
					\item Ex: 0x00 show that the frame is an AMQP frame
					\item Ex: 0x01 represents a SASL frame.
				\end{itemize}
			\end{itemize}
		\end{itemize}

\begin{itemize}
	\item Data Distribution Service
	\item Developed by Object Management Group (OMG)
	\item Supports 23 QoS policies:
		\begin{itemize}
			\item like security, urgency, priority, durability, reliability, etc
		\end{itemize}
	\item Relies on a broker-less architecture
		\begin{itemize}
			\item uses multicasting to bring excellent Quality of Service
			\item real-time constraints
		\end{itemize}
	\item DDS architecture defines two layers:
		\begin{itemize}
			\item[DLRL] Data-Local Reconstruction Layer
				\begin{itemize}
					\item serves as the interface to the DCPS functionalities
				\end{itemize}
			\item[DCPS] Data-Centric Publish/Subscribe
				\begin{itemize}
					\item delivering the information to the subscribers
				\end{itemize}
			\end{itemize}
		\item 5 entities are involved with the data flow in the DCPS layer:
			\begin{itemize}
				\item Publisher:disseminates data
				\item DataWriter: used by app to interact with the publisher
				\item Subscriber: receives published data and delivers them to app
				\item DataReader: employed by Subscriber to access received data
				\item Topic: relate DataWriters to DataReaders
			\end{itemize}
		\end{itemize}

\begin{itemize}
	\item No need for manual reconfiguration or extra administration
	\item It is able to run without infrastructure
	\item It is able to continue working if failure happens.
		\item It inquires names by sending an IP multicast message to all the nodes in the local domain
		\begin{itemize}
			\item Clients asks devices that have the given name to reply back
			\item the target machine receives its name and multicasts its IP @
			\item Devices update their cache with the given name and IP @
		\end{itemize}
	\end{itemize}

\begin{itemize}
	\item Requires zero configuration aids to connect machine
	\item It uses mDNS to send DNS packets to specific multicast addresses through UDP
	\item There are two main steps to process Service Discovery:
		\begin{itemize}
			\item finding host names of required services such as printers
			\item pairing IP addresses with their host names using mDNS
		\end{itemize}
	\item Advantages
		\begin{itemize}
			\item IoT needs an architecture without dependency on a configuration mechanism
			\item smart devices can join the platform or leave it without affecting the behavior of the whole system
		\end{itemize}
	\item Drawbacks
		\begin{itemize}
			\item Need for caching DNS entries
		\end{itemize}
	\end{itemize}


%\changefontsizes{5.3pt}
\begin{table}[h!]
\scriptsize
\begin{tabulary}{\textwidth}{L|L|L|L|L}
	\                                  & \bf{LiteOS}                & \bf{Nano-RK}                  & \bf{MANTIS}          & \bf{Contiki} \\\hline
	\bf{Architecture}                  & Monolithic                 & Layered                       & Modular              & Modular \\\hline
	\bf{Scheduling Memory}             & Round Robin                & Monotonic harmonized          & Priority classes     & Interrupts execute w.r.t. \\\hline
	\bf{Network}                       & File                       & Socket abstraction            & At Kernel COMM layer & uIP, Rime \\\hline
	\bf{Virtualization and Completion} & Synchronization primitives & Serialized access  semaphores & Semaphores           & Serialized, Access \\\hline
	\bf{Multi threading}               & \ok                        & \ok                           & \ko                  & \ok \\\hline
	\bf{Dynamic protection}            & \ok                        & \ko                           & \ok                  & \ok \\\hline
	\bf{Memory Stack}                  & \ok                        & \ko                           & \ko                  & \ko \\\hline
\end{tabulary}
\caption{\label{tab:OS} Common operating systems used in IoT environment \cite{al-fuqaha_internet_24}}
\end{table}

\begin{itemize}
	\item Routing over low-power and lossy links (ROLL)
	\item Support minimal routing requirements.
		\begin{itemize}
			\item like multipoint-to-point, point-to-multipoint and point-to-point.
		\end{itemize}
	\item A Destination Oriented Directed Acyclic Graph (DODAG)
		\begin{itemize}
			\item Directed acyclic graph with a single root.
			\item Each node is aware of ts parents 
			\item but not about related children
		\end{itemize}
	\item RPL uses four types of control messages
		\begin{itemize}
			\item DODAG Information Object (DIO)
			\item Destination Advertisement Object (DAO)
			\item DODAG Information Solicitation (DIS)
			\item DAO Acknowledgment (DAO-ACk)
		\end{itemize}
	%				\item RPL routers work under one of two modes:
	%					\begin{itemize}
	%						\item Non-Storing mode
	%						\item Storing modes mode
	%					\end{itemize}
		\end{itemize}


%\changefontsizes{5pt}
\begin{table}[h!]
\scriptsize
	\begin{tabulary}{\textwidth}{L|C|C|C|C|C|C|C|C|C}
	Paper           & Architecture & Availability & Reliability & Mobility & Performance & Management & Scalability & Interoperability & Security\\\hline
	IoT-A           &              &              &             &          &             &            &             &                  &         \\\hline
	IoT@Work        &              &              &             &          &             &            &             &                  &         \\\hline
	EBBITS          &              &              &             &          &             &            &             &                  &         \\\hline
	BETaas          &              &              &             &          &             &            &             &                  &         \\\hline
	CALIPSO         &              &              &             &          &             &            &             &                  &         \\\hline
	VITAL           &              &              &             &          &             &            &             &                  &         \\\hline
	SENSAI          &              &              &             &          &             &            &             &                  &         \\\hline
	RERUM           &              &              &             &          &             &            &             &                  &         \\\hline
	RELEYonIT       &              &              &             &          &             &            &             &                  &         \\\hline
	IoT6            &              &              &             &          &             &            &             &                  &         \\\hline
	OpenIoT         &              &              &             &          &             &            &             &                  &         \\\hline
	Apec IoV        &              &              &             &          &             &            &             &                  &         \\\hline
	Smart Santander &              &              &             &          &             &            &             &                  &         \\\hline
	OMA Device      &              &              &             &          &             &            &             &                  &         \\\hline
	OMA-DM          &              &              &             &          &             &            &             &                  &         \\\hline
	LWM2M           &              &              &             &          &             &            &             &                  &         \\\hline
	NETCONF Light   &              &              &             &          &             &            &             &                  &         \\\hline
	Kura            &              &              &             &          &             &            &             &                  &         \\\hline
	MASH            &              &              &             &          &             &            &             &                  &         \\\hline
	IoT-iCore       &              &              &             &          &             &            &             &                  &         \\\hline
	PROBE-IT        &              &              &             &          &             &            &             &                  &         \\\hline
	OpenIoT         &              &              &             &          &             &            &             &                  &         \\\hline
	LinkSmart       &              &              &             &          &             &            &             &                  &         \\\hline
	IETF SOLACE     &              &              &             &          &             &            &             &                  &         \\\hline
	BUTLER          &              &              &             &          &             &            &             &                  &         \\\hline
	Codo            &              &              &             &          &             &            &             &                  &         \\\hline
	SVELETE         &              &              &             &          &             &            &             &                  &         \\\hline
	
	\end{tabulary}
	\caption{\label{tab:Table54975} An example table.}
\end{table}

\begin{table}[h!]
\scriptsize
	\begin{tabulary}{\textwidth}{L|C|C|C|C}
	\bf{Platform}      & \ \bf{COAP} & \bf{XMPP} & \bf{MQTT}\\\hline
	\bf{Arkessa}       &             &           & \ok      \\\hline
	\bf{Axeda}         &             &           &          \\\hline
	\bf{Etherios}      &             &           &          \\\hline
	\bf{LittleBits}    &             &           &          \\\hline
	\bf{NanoService}   & \ok         &           &          \\\hline
	\bf{Nimbits}       &             & \ok       &          \\\hline
	\bf{Ninja blocks}  &             &           &          \\\hline
	\bf{OnePlateformv} & \ok         & \ok       &          \\\hline
	\bf{RealTime.io}   &             &           &          \\\hline
	\bf{SensorCloud}   &             &           &          \\\hline
	\bf{SmartThings}   &             &           &          \\\hline
	\bf{TempoDB}       &             &           &          \\\hline
	\bf{ThingWorx}     &             &           & \ok      \\\hline
	\bf{Xively}        &             &           & \ok      \\\hline
	\bf{Ubidots}       &             &           & \ok      \\\hline
		\end{tabulary}
	\caption{\label{tab:IoTPlatforms} IoT cloud platforms and their characteristics}
\end{table}


\begin{itemize}
	\item Standard topologies to form IEEE 802.15.4e networks are 
		\begin{itemize}
			\item[Star] contains at least one FFD and some RFDs
			\item[Mesh] contains a PAN coordinator and other nodes communicate with each other
			\item[Cluster] consists of a PAN coordinator, a cluster head and normal nodes.
		\end{itemize}
			\item The IEEE 802.15.4e standard supports 2 types of network nodes
		\begin{itemize}
			\item[FFD] Full function device: serve as a coordinator
				\begin{itemize}
					\item It is responsible for creation, control and maintenance of the net
					\item It store a routing table in their memory and implement a full MAC
				\end{itemize}
			\item[RFD] Reduced function devices: simple nodes with restricted resources
				\begin{itemize}
					\item They can only communicate with a coordinator
					\item They are limited to a star topology
				\end{itemize}
			\end{itemize}
		\end{itemize}

\begin{tabulary}{\textwidth}{|C|C|C|C|C|C|C|C|C|}\hline
	Preamble & PHDR & PHDRCRC & MHDR & FHDR & FPort & Payload & MIC & CRC \\\hline
\end{tabulary}

%%\changefontsizes{5pt}
\begin{table}[h!]
\scriptsize
	\begin{tabular}{l|l|l|l}
	\textbf{Use cases}         &  &  & \\\hline
	Health Monitoring          &  &  & \\\hline
	Water Distribution         &  &  & \\\hline
	Electricity Distribution   &  &  & \\\hline
	Smart Buildings            &  &  & \\\hline
	Intelligent Transportation &  &  & \\\hline
	Surveillance               &  &  & \\\hline
	Environmental Monitoring   &  &  & \\
	\end{tabular}
	\caption{\label{tab:IoTUseCase} Use cases \cite{hancke_role_2012}}
\end{table}


\begin{table}[h!]
\scriptsize
	\begin{tabulary}{\textwidth}{L|L|L|L|L}
		\bf{Routing protocol}  & \bf{Control Cost} & \bf{Link Cost} & \bf{Node Cost} \\\hline
		\bf{OSPF/IS-IS}        & \ko               & \ok            & \ko      \\
		\bf{OLSRv2}            & ?                 & \ok            & \ok      \\
	%		\bf{TBRPF}             & \ko               & \ok            & ?        \\
		\bf{RIP}               & \ok               & ?              & \ko      \\
	%		\bf{AODV}              & \ok               & \ko            & \ko      \\
	%		\bf{DYMO}              & \ok               & ?              & ?        \\
		\bf{DSR}               & \ok               & \ko            & \ko      \\
		\bf{RPL}               & \ok               & \ok            & \ok      \\\hline
	\end{tabulary}
	\caption{\label{tab:routingsComaprison} Routing protocols comparison \cite{_rpl2_}}
\end{table}

\begin{table}
\scriptsize
	\begin{tabulary}{\textwidth}{C|C|C|C|C|C|C|C}
		\textbf{Application protocol} & RestFull & Transport & Publish/Subscribe & Request/Response & Security & QoS & Header size (Byte)\\\hline
		\textbf{COAP}                 & \ok      & UDP       & \ok               & \ok              & DTLS     & \ok & 4           \\\hline
		\textbf{MQTT}                 & \ko      & TCP       & \ok               & \ko              & SSL      & \ok & 2           \\\hline
		\textbf{MQTT-SN}              & \ko      & TCP       & \ok               & \ko              & SSL      & \ok & 2           \\\hline
		\textbf{XMPP}                 & \ko      & TCP       & \ok               & \ok              & SSL      & \ko & -           \\\hline
		\textbf{AMQP}                 & \ko      & TCP       & \ok               & \ko              & SSL      & \ok & 8           \\\hline
		\textbf{DDS}                  & \ko      & UDP TCP   & \ok               & \ko              & SSL DTLS & \ok & -           \\\hline
		\textbf{HTTP}                 & \ok      & TCP       & \ko               & \ok              & SSL      & \ko & -           \\
	\end{tabulary}
	\caption{\label{tab:protocolsComparison} Application protocols comparison}
\end{table}



%\begin{tikzpicture}[>=latex]
%	% the shapes
%	\node[myshape,rectangle split part fill={white,white,white,white,myred}] (shape1) {};
%	\node[myshape,rectangle split part fill={white,white,white,white,white,white,mygreen},below=of shape1] (shape2) {};
%	\node[myshape,rectangle split part fill={white,white,white,white,myblue},below=of shape2] (shape3) {};
%	\node[mytri,left=of shape2](in) {};
%	\node[draw,circle,inner sep=0.6cm,right=of shape2](out) {};
%% the arrows
%	\draw (out.east) -- ++(15pt,0pt) coordinate (end);
%	\foreach \Ancla/\Color in {{north west}/myred,west/mygreen,{south west}/myblue}{
%		\draw[line width=1.5pt,\Color,->] ([xshift=-30pt]in.\Ancla) -- ([xshift=-5pt]in.\Ancla);
%	}
%	\foreach \Valor/\Color in {1/myred,2/mygreen,3/myblue}{
%		\draw[line width=1.5pt,->,\Color,shorten <= 4pt] (in.east) -- (shape\Valor.west);  
%		\draw[line width=1.5pt,\Color,shorten <= 4pt] (shape\Valor.east) -- (out.west) ;  
%	}
%	\foreach \Ancla/\Color in {{north west}/myred,west/mygreen,{south west}/myblue}{
%		\draw[line width=1.5pt,\Color,->] ([xshift=5pt]end|-out.\Ancla) -- ([xshift=30pt]end|-out.\Ancla);
%	}
%	% the labels
%	\node[align=center,anchor=south east]at ([yshift=10pt]in.north west) {Classify \\ arrivals};
%	\foreach \Valor in {1,2,3}{
%		\node[anchor=west,fill=white] at (shape\Valor.east) {$W_{\Valor}$};
%	}
%	\node[anchor=south] at (out.north){Link};
%	\node[anchor=north] at (end|-out.south){Departures};
%	
%\end{tikzpicture}
%\begin{tikzpicture}[>=latex]
%\node at (2.75,-0.75cm) {$\mu$};
%\node[align=center] at (1cm,-2cm) {Waiting \\ Area};
%\node[align=center] at (3cm,-2cm) {Service \\ Node};
%\draw[<-] (0  ,-0.75) -- +(-20pt,0) node[left] {$\lambda$};
%\draw[->] (3.5,-0.75) -- +(20pt,0 );
%\draw (0,0) -- ++(2cm,0) -- ++(0,-1.5cm) -- ++(-2cmx,0);
%\foreach \i in {1,...,4}
%	\draw (2cm-\i*10pt,0) -- +(0,-1.5cm);
%\draw (2.75,-0.75cm) circle [radius=0.75cm];
%\draw [decorate,decoration={brace,amplitude=10pt},xshift=-4pt,yshift=0pt] (0.5,0.5) -- (0.5,5.0) node [black,midway,xshift=-0.6cm] {\footnotesize $P_1$};
%\end{tikzpicture}



\Figure{!htb}{1}{Filtres.png}{Filtres \cite{merdrignac_systeme_2015}}
\Figure{!htb}{1}{classification.png}{classification \cite{merdrignac_systeme_2015}}
\Figure{!htb}{1}{LPWAN.png}{LPWAN connectivity}
\Figure{!htb}{1}{sdn-wise.png}{LPWAN connectivity}
\Figure{!htb}{1}{interoperability-stack.jpg}{Interoperability}
\Figure{!htb}{1}{stat_interoperability.png}{Key barriers in adopting the Industrial Internet \footfullcite{industrialinternetofthings_executive_}}
\Figure{!htb}{1}{wsn-IoT.png}{wsn-IoT}


\begin{bytefield}[bitwidth=2.1em]{16}
\bitheader{0-15}                                                                                                               \\
\begin{rightwordgroup}{802.15.5 Header}                                                                                        \\
\y{4}{Length}     & \y{4}{FCF}             & \y{4}{DSN}               & \y{4}{DST PAN}                                      \\
\y{16}{Destination address}                                                                                                \\
\y{16}{Source address}                                                                                                     \\
\end{rightwordgroup}                                                                                                           \\
\begin{rightwordgroup}{Mesh addressing Header}                                                                                 \\
\y{1}{I}       & \y{1}{O}             & \y{1}{S}               & \y{1}{D}   & \y{4}{Hop Limit}                             \\
\y{16}{Source address}                                                                                                     \\
\y{16}{Destination address}                                                                                                \\
\end{rightwordgroup}                                                                                                           \\
\begin{rightwordgroup}{Fragment Header}                                                                                        \\
\y{1}{I}       & \y{1}{I}             & \y{1}{O}               & \y{2}{rsv} & \y{11}{Datagrame size}                       \\
\y{16}{Datagrame Tag}                                                                                                      \\
\y{8}{Datagrame Offset}                                                                                                    \\
\end{rightwordgroup}                                                                                                           \\
\begin{rightwordgroup}{IPv6 Header}                                                                                            \\
\y{5}{Version} & \y{5}{Traffic Class} & \y{6}{Flow Label}                                                                  \\
\y{8}{Length}  & \y{4}{Next Header}   & \y{4}{Hope limit}                                                                  \\
\y{16}{Source address}                                                                                                     \\
\y{16}{Destination address}                                                                                                \\
\end{rightwordgroup}                                                                                                           \\
\begin{rightwordgroup}{IPv6 Payload}                                                                                           \\
\y{2}{V=2}     & \y{1}{P}             & \y{1}{X}                & \y{4}{CC}  & \y{1}{M}                         & \y{7}{PT}\\
\y{16}{timestamp}                                                                                                          \\
\end{rightwordgroup}                                                                                                           \\
\begin{rightwordgroup}{UDP Header}                                                                                             \\
\y{8}{Source port}     & \y{8}{Destination port}                                                                           \\
\y{8}{Length}                & \y{8}{Checksum}                                                                             \\
\end{rightwordgroup}                                                                                                           \\
\end{bytefield}
\begin{table}[h!]
\begin{center}
	\begin{tabular}{l|l|l|l|l}
	Application & CoAP, MQTT          &                     &                       &            \\\hline
	Transport   & \multicolumn{4}{c}{UDP/TCP}                                                    \\\hline
	Network     & IPv6 RPL            & \multicolumn{3}{c}{IPv4/IPv6}                            \\\hline
	\           & 6LowPan             & \multicolumn{2}{c}{RFC 2464}                & RFC 5072   \\\hline
	MAC         & IEEE 802.15.4       & IEEE 802.11 (Wi-Fi) & IEEE 802.3 (Ethernet) & 2G, 3G, LTE\\\hline
	\           & 2.4GHz, 915, 868MHz & 2.4, 5GHz           &                       &            \\\hline
	\           & DSS, FSK, OFDM      & CSMA/CA             & UTP, FO               &            \\\hline
		\end{tabular}
	\caption{\label{tab:Table} An example table.}
\end{center}
\end{table}



\begin{itemize}
	\item Network selection
	\begin{itemize}
		\item MADM
		\begin{itemize}
			\item Ranking methods
			\item Ranking \& weighted methods
		\end{itemize}
		\item Game theory
		\begin{itemize}
			\item Users vs users
			\item Users vs networks
			\item Networks vs network
		\end{itemize}
		\item Fuzzy logic
		\begin{itemize}
			\item as a score method
			\item another theory
		\end{itemize}
		\item Utility function
		\begin{itemize}
			\item 1
			\item 2
		\end{itemize}
	\end{itemize}
\end{itemize}

\begin{table}[h!]
\scriptsize
	\begin{tabulary}{\textwidth}{L|L}
	\multirow{3}{*}{Naïve modes}           & Instantaneous  \\
	\                                      & Hist. average  \\
	\                                      & Clustering     \\\hline
	\multirow{9}{*}{Parametric models}     & Rarely used    \\
	\                                      & Traffic Models \\
	\                                      & Time Series    \\
	\                                      & Linear regression \\
	\                                      & ARIMA \\
	\                                      & Kalman filtering \\
	\                                      & ATHENA \\
	\                                      & SETAR \\
	\                                      & Gaussian Maximum Likelihood \\\hline

	\multirow{6}{*}{Non-Parametric models} & k-Nearest Neighbor           \\
	\                                      & Locally Weighted Regression \\
	\                                      & Fuzzy Logic    \\
	\                                      & Bayes Network  \\
	\                                      & Neural Network \\
	\                                      & Include temporal/spatial patterns \\
	\end{tabulary}
	\caption{\label{tab:models} Taxonomy of prediction models \cite{_short_2007}}
\end{table}

\begin{table}[h!]
\scriptsize
	\begin{tabular}{l|l|l|l}
	\textbf{Parameters}                                & \textbf{Parameters}     & \textbf{Type} & \textbf{Expected as}  \\\hline
	\multirow{4}{*}{\textbf{Network conditions}}       & network load            & Dynamic       & Minimized  \\
	\                                                  & network coverage        & Static        & Fixed      \\
	\                                                  & network connection time & Dynamic       & Minimized  \\
	\                                                  & available bandwidth     & Dynamic       & Minimized  \\\hline
	\multirow{5}{*}{\textbf{Application requirements}} & throughput              & Dynamic       & Minimized  \\
	\                                                  & delay                   & Dynamic       & Minimized  \\
	\                                                  & jitter                  & Dynamic       & Minimized  \\
	\                                                  & PLR                     & Dynamic       & Minimized  \\
	\                                                  & energy consumption      & Dynamic       & Minimized  \\\hline
	\multirow{2}{*}{\textbf{User preferences}}         & budget                  & Static        & Fixed      \\
	\                                                  & cost                    & Static        & Fixed      \\
	\                                                  & design                  &               &            \\\hline
	\multirow{2}{*}{\textbf{Mobile equipment}}         & energy                  & Dynamic       & Fixed      \\
	\                                                  & mobility                & Dynamic       & Fixed      \\\hline
	\end{tabular}
	\caption{\label{tab:network_selection} Network selection inputs and classification of parameters \cite{bendaoud_network_2019}}
\end{table}


QoS parameters \cite{meshinchi_qosaware_2018} \cite{chowdhury_survey_2018}
\begin{itemize}
	\item Application layer
	\begin{itemize}
		\item Service time
		\item Service availability
		\item Service cost
		\item Service reliability
	\end{itemize}
	\item Network layers
	\begin{itemize}
		\item Bandwidth
		\item Packet loss
		\item Jitter
		\item Delay
	\end{itemize}
	\item Sensing layer
	\begin{itemize}
		\item Data accuracy
		\item Data collection delay
		\item Sampling rate
		\item WSN lifetime
		\item WSN coverage
	\end{itemize}
	\item Sensing layer
	\begin{itemize}
		\item Information accuracy
		\begin{itemize}
			\item Data accuracy
			\item Sensing time accuracy
			\item Spatial accuracy
			\item Reduce data redundancy
			\item Data packaging
		\end{itemize}
		\item Energy compsumption
		\begin{itemize}
			\item Sleep management
			\item Life time management
		\end{itemize}
		\item Coverage
		\begin{itemize}
			\item Sensing area
		\end{itemize}
	\end{itemize}
\end{itemize}

\begin{table}[h!]
\scriptsize
	\begin{tabulary}{\textwidth}{|p{6em}|L|L|L|L|}
	\                           & \textbf{802.15.4}                   & \textbf{802.15.4e}                       & \textbf{802.15.4g}                & \textbf{802.15.4f}                      \\\hline
	\textbf{Frequency}          & 2.4Ghz (DSSS + oQPSK)               & 2.4Ghz (DSSS + oQPSK, CSS+DQPSK )        & 2.4Ghz (DSSS + oQPSK, CSS+DQPSK ) & 2.4Ghz (DSSS + oQPSK,CSS+DQPSK )        \\
	\                           & 868Mhz (DSSS + BPSK)                & 868Mhz (DSSS + BPSK)                     & 868Mhz (DSSS + BPSK)              & 868Mhz (DSSS + BPSK)                    \\
	\                           & 915Mhz (DSSS + BPSK)                & 915Mhz (DSSS + BPSK)                     & 915Mhz (DSSS + BPSK)              & 915Mhz (DSSS + BPSK) 3\sim 10Ghz (BPM+BPSK )\\
	\textbf{Data rate}          & Upto 250kbps                        & Upto 800kbps                             & Up to 800kbps                     &                                         \\
	\textbf{Differences}        & -                                   & Time sync and channel hopping            & Phy Enhancements                  & Mac and Phy Enhancements                \\
	\textbf{Frame Size}         & 127 bytes                           & N/A                                      & Up to 2047 bytes                  & N/A                                     \\
	\textbf{Range}              & 1 – 75+ m                           & 1 – 75+ m                                & Upto 1km                          & N/A                                     \\
	\textbf{Goals}              & General Low-power Sensing/Actuating & Industrial segments                      & Smart utilities                   & Active RFID                             \\
	\textbf{Products}           & Many                                & Few                                      & Connode (6LoWPAN)                 & LeanTegra PowerMote                     \\\hline
	\end{tabulary}
\caption{\label{tab:IEEE_802.15.4_standards} IEEE 802.15.4 standards \cite{sarwar_iot_}}
\end{table}

\begin{table}[h!]
\scriptsize
	\begin{tabulary}{\textwidth}{L|L|L|L|L|L|L}
	\bf{Phy protocol}     & \bf{IEEE 802.15.4} & \bf{BLE}      & \bf{EPCglobal} & \bf{Z-Wave}              & \bf{LTE-M}           & \bf{ZigBee} \\\hline
	\bf{Standard Body}    &                    & IEEE 802.15.1 &                &                          &                      & IEEE 802.15.4, ZigBee Alliance \\\hline
	\bf{Radio band (MHz)} & 868/915/2400       & 2400          & 860-960        & 868/908/2400             & 700-900              & \\\hline
	\bf{MAC address}      & TDMA, CSMA/CA      & TDMA          & ALOHA          & CSMA/CA                  & OFDMA                & \\\hline
	\bf{Data rate (bps)}  & 20/40/250 K        & 1024K         & varies 5-640K  & 40K                      & 1G (up), 500M (down) & \\\hline
	\bf{Throughput}       &                    &               &                & 9.6, 40, 200kbps         &                      & \\\hline
	\bf{Scalability ???}  & 65K nodes          & 5917 slaves   & -              & 232 nodes                & -                    & \\\hline
	\bf{Range}            & 10-20m             & 10-100m       &                &                          &                      & \\\hline
	\bf{Addressing}       & 8|16bit            & 16bit         &                &                          &                      & \\\hline
	\end{tabulary}
	\caption{\label{tab:IoT_cloud} IoT cloud platforms and their characteristics \cite{al-fuqaha_internet_24}}
\end{table}

%\changefontsizes{6pt}
\begin{table}[h!]
\scriptsize
	\begin{tabulary}{\textwidth}{L|L|L|L|L}
	\bf{Characteristics}                           & \bf{6LoWPAN}              & \bf{LoRaWAN}                    & \bf{SigFox}                           & \bf{Narrowband}\\\hline
	\bf{Standar body}                              &                           & LoRa Alliance                   &                                       & 3GPP          \\\hline
	\bf{TX Active Power @ 3V}                      &                           &                                 &                                       &               \\\hline
	\bf{Frequency band}                            & 902-929                   & 902-928                         & 902                                   &               \\
	\bf{\footnotesize{(MHz)}}                      & 868-868.6                 & 863-870 and 434                 & 868                                   &               \\\hline
	\bf{Number of channels}                        & 0016 for 2400             & 80             for 915          & 25                                    &               \\
	\bf{\footnotesize{(channels for MHz)}}         & 0010 for 915              & 10             for 868 and 780  &                                       &               \\
	\                                              & 0001 for 868.3            &                                 &                                       &               \\\hline
	\bf{Channel bandwidth}                         & 0005 for 2400             & 0.125 and 0.50 for 915          & 0.0001-0.0012                         &               \\
	\bf{\footnotesize{(MHz)}}                      & 0002 for 915              & 0.125 and 0.25 for 868 and 780  &                                       &               \\
	\                                              & 0600 for 868.3            &                                 &                                       &               \\\hline
	\bf{Maximum data rate}                         & 0250 for 2400             & 0.00098-0.0219 for 915          & 0.1-0.6                               &               \\
	\bf{\footnotesize{(kbps for MHz)}}             & 0040 for 915              & 0.250-0.05     for 868 and 780  &                                       &               \\
	\                                              & 0020 for 868.3            &                                 &                                       &               \\\hline
	\bf{Channel modulation}                        & QPSK for 2400             & LoRa           for 915          & BPSK and GFSK                         &               \\
	\                                              & BPSK for 915              & LoRa and GFSK  for 868  and 780 &                                       &               \\
	\                                              & BPSK for 868.3            &                                 &                                       &               \\\hline
	\bf{Channel coding}                            & -085 for 2400             & -137                            & -137                                  &               \\
	\bf{\footnotesize{(dBm for MHz)}}              & -092 for 915              &                                 &                                       &               \\
	\                                              & -092 for 868.3            &                                 &                                       &               \\\hline
	\bf{Protocol data unit \footnotesize{(bytes)}} & 6+127                     & x + (19 to 250)                 & 12+ (0 to 12)                         &               \\\hline
	\bf{Channel coding}                            & Direct                    & CSS                             & Ultra                                 &               \\\hline
	\bf{Transmission range}                        & 10-100 m                  & 5-15 km                         & 10-50 km                              &               \\\hline
	\bf{Battery lifetime}                          & 1-2 years                 & <10 years                       & <10 years                             &               \\\hline
	\bf{Standard Body}                             & IETF                      &                                 &                                       &               \\\hline
	\bf{Security}                                  & Access Control List (ACL) &                                 &                                       &               \\\hline
	\bf{Uplink}                                    &                           &                                 & 100bps, 12 bytes/msg, max 140 msg/day &               \\\hline
	\bf{Downlink}                                  &                           &                                 & 8 bytes/msg, max 4 msg/day            &               \\\hline
	\bf{Scalability}                               &                           &                                 &                                       &               \\\hline
	\bf{Proprietary}                               &                           &                                 & \ok                                   &               \\\hline
	\bf{Cost}                                      &                           & High                            &                                       &               \\\hline
	
	\end{tabulary}
	\caption{\label{tab:LPWan_characteristics} LPWan Characteristics \cite{al-kashoash_comparison_2016}}
\end{table}
%\changefontsizes{7pt}
\begin{table}[h!]
\scriptsize
	\begin{tabulary}{\textwidth}{L|L|L|L|L|L}
	\bf{Feature}                     & \bf{Wi-Fi}                                            & \bf{802.11p}                                          & \bf{UMTS}                        & \textbf{LTE}                      & \textbf{LTE-A}                     \\\hline
	\bf{Channel width MHz}           & 20                                                    & 10                                                    & 5                                & 1.4, 3, 5, 10, 15, 20    & <100                 \\\hline
	\bf{Frequency band(s) GHz}       & 2.4 , 5.2                                             & 5.86-5.92                                             & 0.7-2.6                          & 0.7-2.69                 & 0.45-4.99                 \\\hline
	\bf{Bit rate Mb/s}               & 6-54                                                  & 3–27                                                  & 2                                & <300                & <1000                \\\hline
	\bf{Range km}                    & <0.1                                             & <1                                               & <10                         & <30                 & <30                  \\\hline
	\bf{Capacity}                    & Medium                                                & Medium                                                & \ko                              & \ok                      & \ok                       \\\hline
	\bf{Coverage}                    & Intermittent                                          & Intermittent                                          & Ubiquitous                       & Ubiquitous               & Ubiquitous                \\\hline
	\bf{Mobility support km/h}       & \ko                                                   & Medium                                                & \ok                              & <350                & <350                 \\\hline
	\bf{QoS support}                 & EDCA \scriptsize{Enhanced Distributed Channel Access} & EDCA \scriptsize{Enhanced Distributed Channel Access} & QoS classes and bearer selection & QCI and bearer selection & QCI and bearer selection  \\\hline
	\bf{Broadcast/multicast support} & Native broadcast                                      & Native broadcast                                      & Through MBMS                     & Through eMBMS            & Through eMBMS             \\\hline
	\bf{V2I support}                 & \ok                                                   & \ok                                                   & \ok                              & \ok                      & \ok                       \\\hline
	\bf{V2V support}                 & Native (ad hoc)                                       & Native (ad hoc)                                       & \ko                              & \ko                      & Through D2D               \\\hline
	\bf{Market penetration}          & \ok                                                   & \ko                                                   & \ok                              & \ok                      & \ok                       \\\hline
	\bf{Data rate}                   & <640 kbps                                        & 250 kbps                                              & 106–424 kbps                              & \ok                      & \ok                       \\\hline
	\end{tabulary}
	\caption{\label{tab:Tableppp} An example table.}
\end{table}


\begin{itemize}
	\item[\cite{qin_software_2014}] Many studies have identified \green{SDN} as a potential solution to the WSN challenges,
	as well as a model for \red{heterogeneous} integration.
	\item[\cite{qin_software_2014}] This \red{shortfall} can be resolved by using the \green{SDN approach.}
	\item[\cite{kobo_survey_2017}] \green{SDN} also enhances better control of \red{heterogeneous} network infrastructures.
	\item[\cite{kobo_survey_2017}] Anadiotis et al. define a \green{SDN operating system for IoT} that integrates SDN based WSN \textbf{(SDN-WISE)}.
		This experiment shows how \red{heterogeneity} between different kinds of SDN networks can be achieved.
	\item[\cite{kobo_survey_2017}] In cellular networks,
		OpenRoads presents an approach of introducing \green{SDN} based \red{heterogeneity} in wireless networks for operators.
	\item[\cite{ndiaye_software_2017}] There has been a plethora of (industrial) studies \green{synergising SDN in IoT}.
			The major characteristics of IoT are low latency,wireless access, mobility and \red{heterogeneity}.
	\item[\cite{ndiaye_software_2017}] Thus a bottom-up approach application of \green{SDN} to the realisation of \red{heterogeneous IoT} is suggested.
	\item[\cite{ndiaye_software_2017}] Perhaps a more complete IoT architecture is proposed,
			where the authors apply \green{SDN} principles in IoT \red{heterogeneous} networks.
	\item[\cite{bera_softwaredefined_2017}] it provides the \green{SDWSN} with a proper model of network management,
			especially considering the potential of \red{heterogeneity} in SDWSN.
	\item[\cite{bera_softwaredefined_2017}] We conjecture that the \green{SDN paradigm} is a good candidate to solve the \red{heterogeneity} in IoT.

\end{itemize}


\begin{table}[h!]
\scriptsize
	\begin{tabulary}{\columnwidth}{L|L|C|C|C|C|C}
	\textbf{Management architecture}                 & \textbf{Management feature}            & \textbf{Controller configuration} & \textbf{Traffic Control} & \textbf{Configuration and monitoring} & \textbf{Scapability and localization} & \textbf{Communication management}\\\hline
	\textbf{\cite{luo_sensor_2012} Sensor Open Flow} & SDN support protocol                   & Distributed                       & in/out-band              & \ok                                   & \ok                                   & \ok                              \\\hline
	\textbf{\cite{costanzo_software_2012} SDWN}          & Duty sycling, aggregation, routing     & Centralized                       & in-band                  & \ok                                   &                                       & \\\hline
	\textbf{\cite{galluccio_sdnwise_2015} SDN-WISE}  & Programming simplicity and aggregation & Distributed                       & in-band                  &                                       & \ok                                   & \\\hline
	\textbf{\cite{degante_smart_2014} Smart}        & Efficiency in resource allocation      & Distributed                       & in-band                  &                                       & \ok                                   & \\\hline
	\textbf{SDCSN}                                   & Network reliability and QoS            & Distributed                       & in-band                  &                                       & \ok                                   & \\\hline
	\textbf{TinySDN}                                 & In-band-traffic control                & Distributed                       & in-band                  &                                       & \ok                                   & \\\hline
	\textbf{Virtual Overlay}                         & Network flexibility                    & Distributed                       & in-band                  &                                       & \ok                                   & \\\hline
	\textbf{Context based}                           & Network scalability and performance    & Distributed                       & in-band                  &                                       & \ok                                   & \\\hline
	\textbf{CRLB}                                    & Node localization                      & Centralized                       & in-band                  &                                       &                                       & \\\hline
	\textbf{Multi-hope}                              & Traffic and energy control             & Centralized                       & in-band                  &                                       &                                       & \ok                              \\\hline
	\textbf{Tiny-SDN}                                & Network task measurement               & -                                 & in-band                  &                                       &                                       & \\
	\end{tabulary}
	\caption{\label{tab:Table} SDN-based network and topology management architectures. \cite{ndiaye_software_2017}}
\end{table}

\begin{table}
\scriptsize
	\begin{tabulary}{\textwidth}{L|C|C|C|C|C|C|C}
		\bf{Application protocol}                                                 & DDS                                     & CoAP                              & AMQP                              & MQTT                                & MQTT-SN & XMPP & HTTP\\\hline
		\bf{Service discovery}     & \multicolumn{3}{c}{mDNS}                & \multicolumn{4}{c}{DNS-SD}                                                                                                         \\
		\bf{Network layer}         & \multicolumn{7}{c}{RPL}                                                                                                                                                      \\
	%			\bf{Network layer}         & \multicolumn{3}{c}{6LoWPAN}             & \multicolumn{4}{c}{IPv4/IPv6}                                                                                                      \\
		\bf{Link layer}            & \multicolumn{7}{c}{IEEE 802.15.4}                                                                                                                                            \\
		\bf{Physical layer}        & \multicolumn{3}{c}{EPCglobal}     & \multicolumn{2}{c}{IEEE 802.15.4} & \multicolumn{2}{c}{Z-Wave}                                 \\\hline
	\end{tabulary}
	\caption{\label{tab:Tablej} Standardization efforts that support the IoT}
\end{table}

\begin{table}[h!]
\scriptsize
\begin{tabulary}{\textwidth}{L|L|L|L|L}
	\                                  & \bf{LiteOS}                & \bf{Nano-RK}                  & \bf{MANTIS}          & \bf{Contiki} \\\hline
	\bf{Architecture}                  & Monolithic                 & Layered                       & Modular              & Modular \\\hline
	\bf{Scheduling Memory}             & Round Robin                & Monotonic harmonized          & Priority classes     & Interrupts execute w.r.t. \\\hline
	\bf{Network}                       & File                       & Socket abstraction            & At Kernel COMM layer & uIP, Rime \\\hline
	\bf{Virtualization and Completion} & Synchronization primitives & Serialized access  semaphores & Semaphores           & Serialized, Access \\\hline
	\bf{Multi threading}               & \ok                        & \ok                           & \ko                  & \ok \\\hline
	\bf{Dynamic protection}            & \ok                        & \ko                           & \ok                  & \ok \\\hline
	\bf{Memory Stack}                  & \ok                        & \ko                           & \ko                  & \ko \\\hline
\end{tabulary}
\caption{\label{tab:OS} Common operating systems used in IoT environment \cite{al-fuqaha_internet_24}}
\end{table}

\begin{itemize}
	\item Routing over low-power and lossy links (ROLL)
	\item Support minimal routing requirements.
	\begin{itemize}
		\item like multipoint-to-point, point-to-multipoint and point-to-point.
	\end{itemize}
	\item A Destination Oriented Directed Acyclic Graph (DODAG)
	\begin{itemize}
		\item Directed acyclic graph with a single root.
		\item Each node is aware of ts parents 
		\item but not about related children
	\end{itemize}
	\item RPL uses four types of control messages
	\begin{itemize}
		\item DODAG Information Object (DIO)
		\item Destination Advertisement Object (DAO)
		\item DODAG Information Solicitation (DIS)
		\item DAO Acknowledgment (DAO-ACk)
	\end{itemize}
	%				\item RPL routers work under one of two modes:
	%					\begin{itemize}
	%						\item Non-Storing mode
	%						\item Storing modes mode
	%					\end{itemize}
\end{itemize}

%\changefontsizes{5pt}
\begin{table}[h!]
\scriptsize
	\begin{tabulary}{\textwidth}{L|C|C|C|C|C|C|C|C|C}
	Paper           & Architecture & Availability & Reliability & Mobility & Performance & Management & Scalability & Interoperability & Security\\\hline
	IoT-A           &              &              &             &          &             &            &             &                  &         \\\hline
	IoT@Work        &              &              &             &          &             &            &             &                  &         \\\hline
	EBBITS          &              &              &             &          &             &            &             &                  &         \\\hline
	BETaas          &              &              &             &          &             &            &             &                  &         \\\hline
	CALIPSO         &              &              &             &          &             &            &             &                  &         \\\hline
	VITAL           &              &              &             &          &             &            &             &                  &         \\\hline
	SENSAI          &              &              &             &          &             &            &             &                  &         \\\hline
	RERUM           &              &              &             &          &             &            &             &                  &         \\\hline
	RELEYonIT       &              &              &             &          &             &            &             &                  &         \\\hline
	IoT6            &              &              &             &          &             &            &             &                  &         \\\hline
	OpenIoT         &              &              &             &          &             &            &             &                  &         \\\hline
	Apec IoV        &              &              &             &          &             &            &             &                  &         \\\hline
	Smart Santander &              &              &             &          &             &            &             &                  &         \\\hline
	OMA Device      &              &              &             &          &             &            &             &                  &         \\\hline
	OMA-DM          &              &              &             &          &             &            &             &                  &         \\\hline
	LWM2M           &              &              &             &          &             &            &             &                  &         \\\hline
	NETCONF Light   &              &              &             &          &             &            &             &                  &         \\\hline
	Kura            &              &              &             &          &             &            &             &                  &         \\\hline
	MASH            &              &              &             &          &             &            &             &                  &         \\\hline
	IoT-iCore       &              &              &             &          &             &            &             &                  &         \\\hline
	PROBE-IT        &              &              &             &          &             &            &             &                  &         \\\hline
	OpenIoT         &              &              &             &          &             &            &             &                  &         \\\hline
	LinkSmart       &              &              &             &          &             &            &             &                  &         \\\hline
	IETF SOLACE     &              &              &             &          &             &            &             &                  &         \\\hline
	BUTLER          &              &              &             &          &             &            &             &                  &         \\\hline
	Codo            &              &              &             &          &             &            &             &                  &         \\\hline
	SVELETE         &              &              &             &          &             &            &             &                  &         \\\hline
	
	\end{tabulary}
	\caption{\label{tab:Table54975} An example table.}
\end{table}

\begin{table}[h!]
\scriptsize
	\begin{tabulary}{\textwidth}{L|C|C|C|C}
	\bf{Platform}      & \ \bf{COAP} & \bf{XMPP} & \bf{MQTT}\\\hline
	\bf{Arkessa}       &             &           & \ok      \\\hline
	\bf{Axeda}         &             &           &          \\\hline
	\bf{Etherios}      &             &           &          \\\hline
	\bf{LittleBits}    &             &           &          \\\hline
	\bf{NanoService}   & \ok         &           &          \\\hline
	\bf{Nimbits}       &             & \ok       &          \\\hline
	\bf{Ninja blocks}  &             &           &          \\\hline
	\bf{OnePlateformv} & \ok         & \ok       &          \\\hline
	\bf{RealTime.io}   &             &           &          \\\hline
	\bf{SensorCloud}   &             &           &          \\\hline
	\bf{SmartThings}   &             &           &          \\\hline
	\bf{TempoDB}       &             &           &          \\\hline
	\bf{ThingWorx}     &             &           & \ok      \\\hline
	\bf{Xively}        &             &           & \ok      \\\hline
	\bf{Ubidots}       &             &           & \ok      \\\hline
	\end{tabulary}
	\caption{\label{tab:IoTPlatforms} IoT cloud platforms and their characteristics}
\end{table}


\begin{itemize}
	\item Standard topologies to form IEEE 802.15.4e networks are 
	\begin{itemize}
		\item[Star] contains at least one FFD and some RFDs
		\item[Mesh] contains a PAN coordinator and other nodes communicate with each other
		\item[Cluster] consists of a PAN coordinator, a cluster head and normal nodes.
	\end{itemize}
	\item The IEEE 802.15.4e standard supports 2 types of network nodes
	\begin{itemize}
		\item[FFD] Full function device: serve as a coordinator
		\begin{itemize}
			\item It is responsible for creation, control and maintenance of the net
			\item It store a routing table in their memory and implement a full MAC
		\end{itemize}
		\item[RFD] Reduced function devices: simple nodes with restricted resources
		\begin{itemize}
			\item They can only communicate with a coordinator
			\item They are limited to a star topology
		\end{itemize}
	\end{itemize}
\end{itemize}

\begin{tabulary}{\textwidth}{|C|C|C|C|C|C|C|C|C|}\hline
	Preamble & PHDR & PHDRCRC & MHDR & FHDR & FPort & Payload & MIC & CRC \\\hline
\end{tabulary}

%%\changefontsizes{5pt}
\begin{table}[h!]
\scriptsize
	\begin{tabular}{l|l|l|l}
	\textbf{Use cases}         &  &  & \\\hline
	Health Monitoring          &  &  & \\\hline
	Water Distribution         &  &  & \\\hline
	Electricity Distribution   &  &  & \\\hline
	Smart Buildings            &  &  & \\\hline
	Intelligent Transportation &  &  & \\\hline
	Surveillance               &  &  & \\\hline
	Environmental Monitoring   &  &  & \\
	\end{tabular}
	\caption{\label{tab:IoTUseCase} Use cases \cite{hancke_role_2012}}
\end{table}

\begin{table}[h!]
\scriptsize
	\begin{tabulary}{\textwidth}{L|L|L|L|L}
		\bf{Routing protocol}  & \bf{Control Cost} & \bf{Link Cost} & \bf{Node Cost} \\\hline
		\bf{OSPF/IS-IS}        & \ko               & \ok            & \ko      \\
		\bf{OLSRv2}            & ?                 & \ok            & \ok      \\
	%		\bf{TBRPF}             & \ko               & \ok            & ?        \\
		\bf{RIP}               & \ok               & ?              & \ko      \\
	%		\bf{AODV}              & \ok               & \ko            & \ko      \\
	%		\bf{DYMO}              & \ok               & ?              & ?        \\
		\bf{DSR}               & \ok               & \ko            & \ko      \\
		\bf{RPL}               & \ok               & \ok            & \ok      \\\hline
	\end{tabulary}
	\caption{\label{tab:routingsComaprison} Routing protocols comparison \cite{_rpl2_}}
\end{table}

\begin{table}
\scriptsize
	\begin{tabulary}{\textwidth}{C|C|C|C|C|C|C|C}
		\textbf{Application protocol} & RestFull & Transport & Publish/Subscribe & Request/Response & Security & QoS & Header size (Byte)\\\hline
		\textbf{COAP}                 & \ok      & UDP       & \ok               & \ok              & DTLS     & \ok & 4           \\\hline
		\textbf{MQTT}                 & \ko      & TCP       & \ok               & \ko              & SSL      & \ok & 2           \\\hline
		\textbf{MQTT-SN}              & \ko      & TCP       & \ok               & \ko              & SSL      & \ok & 2           \\\hline
		\textbf{XMPP}                 & \ko      & TCP       & \ok               & \ok              & SSL      & \ko & -           \\\hline
		\textbf{AMQP}                 & \ko      & TCP       & \ok               & \ko              & SSL      & \ok & 8           \\\hline
		\textbf{DDS}                  & \ko      & UDP TCP   & \ok               & \ko              & SSL DTLS & \ok & -           \\\hline
		\textbf{HTTP}                 & \ok      & TCP       & \ko               & \ok              & SSL      & \ko & -           \\
	\end{tabulary}
	\caption{\label{tab:protocolsComparison} Application protocols comparison}
\end{table}



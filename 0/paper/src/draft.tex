
\Itemize{
	\item Transaction \& contract layer
	\item Validation layer (forward validation request)
	\item Block Generation Layer (PoW,PoC, PoA PoS, PBFT)
	\item Distribution Layer
}

Consensus algorithms
\Itemize{
	\item Proof of Work (PoW)
	\item Proof of Capacity (PoC)
	\item Proof of Authority (PoA)
	\item Proof of Stake (PoS)
	\item Proof of Bizantine Fault Tolerant (PBFT)
}

\begin{table}[h!]
%\scriptsize
	\begin{tabular}{l|l|l|l}
	\textbf{Parameters}                                & \textbf{Parameters}     & \textbf{Type} & \textbf{Expected as}  \\\hline
	\multirow{4}{*}{\textbf{Network conditions}}       & network load            & Dynamic       & Minimized  \\
	\                                                  & network coverage        & Static        & Fixed      \\
	\                                                  & network connection time & Dynamic       & Minimized  \\
	\                                                  & available bandwidth     & Dynamic       & Minimized  \\\hline
	\multirow{5}{*}{\textbf{Application requirements}} & throughput              & Dynamic       & Minimized  \\
	\                                                  & delay                   & Dynamic       & Minimized  \\
	\                                                  & jitter                  & Dynamic       & Minimized  \\
	\                                                  & PLR                     & Dynamic       & Minimized  \\
	\                                                  & energy consumption      & Dynamic       & Minimized  \\\hline
	\multirow{2}{*}{\textbf{User preferences}}         & budget                  & Static        & Fixed      \\
	\                                                  & cost                    & Static        & Fixed      \\
	\                                                  & design                  &               &            \\\hline
	\multirow{2}{*}{\textbf{Mobile equipment}}         & energy                  & Dynamic       & Fixed      \\
	\                                                  & mobility                & Dynamic       & Fixed      \\\hline
	\end{tabular}
	\caption{\label{tab:network_selection} Network selection inputs and classification of parameters \cite{bendaoud_network_2019}}
\end{table}



\begin{table}[h!]
%\scriptsize
	\begin{tabulary}{\textwidth}{L|L|L}
	\textbf{Application layer} & \textbf{Network layer} & \textbf{Sensing layer }        \\\hline
	Service time               & Bandwidth              & \textbf{Energy consumption}                  \\
	Service availability       & Packet loss            & Sleep management          \\
	Service cost               & Jitter                 & Life time management                  \\
	Service reliability        & Delay                  & \textbf{Coverage}                   \\
	\                          & Availability           & Sensing area              \\
	\                          &                        & \textbf{Information accuracy}  \\
	\                          &                        & Data accuracy                  \\
	\                          &                        & Sensing time accuracy          \\
	\                          &                        & Spatial accuracy               \\
	\                          &                        & Reduce data redundancy         \\
	\                          &                        & Data packaging                 \\
	\                          &                        &                                \\\hline
	\                          &                        & Sampling rate                               \\\hline
	\                          &                        & Bit rate error                               \\\hline
	\                          &                        &                                \\\hline
	\                          &                        &                                \\\hline
	\                          &                        &                                \\\hline
	\                          &                        &                                \\\hline

	\end{tabulary}
\caption{\label{tab:} QoS parameters \cite{meshinchi_qosaware_2018} \cite{chowdhury_survey_2018}}
\end{table}










\begin{table}[h!]
%\scriptsize
	\begin{tabulary}{\textwidth}{L|L}
	\textbf{Maximize}         & \textbf{Minimize}             \\\hline
	(\textbf{T}) Throughput            & (\textbf{RT}) Response Time            \\
	(\textbf{F}) Fairness              & (\textbf{LT}) Latency                  \\
	(\textbf{R}) Reliability           & (\textbf{J}) Jitter                    \\
	(\textbf{IA}) Information Accuracy & (\textbf{TF})  Traffic                 \\
	(\textbf{Cov}) Coverage of IoT     & (\textbf{AWT}) Average Waiting Time    \\
	(\textbf{NL}) Network Life         & (\textbf{D})  Delay                    \\
	(\textbf{RU}))Resource Utilization & (\textbf{L}) Load                      \\
	\                         & (\textbf{EC}) Energy Consumption       \\
	\                         & (\textbf{BP})  Blocking Probability    \\
	\                         & (\textbf{CCI})  Co-channel Interference\\
	\                         & (\textbf{SC}) Service Cost             \\
	\                         & (\textbf{ST})   Service Time           \\
	\end{tabulary}
\caption{\label{tab:scheduling} Objectives of IoT resource scheduling}
\end{table}






Network selection

Service selection

Gateway selection

Input:
	

Method:
	Ranking machine learning

Output:
	Ranked list of gateway



\Figure{!htb}{1}{lorawan_parameters.png}{LoraWan Parameters}

\ac{NY}

\begin{table}[h!]
\scriptsize
	\begin{tabulary}{\textwidth}{L|L|L}
		\textbf{Plan de controle}   & \textbf{Plan de gestion}   & 	\textbf{Plan de doonées}  \\\hline
	Controle d'admission      & Controle et supervision de QoS & Controle du trafic         \\
	Réservation de ressources & Gestion de contrats            & Façonnage du trafic        \\
	Routage                   & QoS mapping                    & Controle de congestion     \\
	Signalisation             & Politique de QoS               & Classification de paquets  \\
	\                         &                                & Marquage de paquets        \\
	\                         &                                & Ordonnancements des paquets\\
	\                         &                                & Gestion de files d'attente \\
	\end{tabulary}
	\caption{\label{tab:qos} An example table.}
\end{table}

%\begin{tikzpicture}[>=latex]
%	% the shapes
%	\node[myshape,rectangle split part fill={white,white,white,white,myred}] (shape1) {};
%	\node[myshape,rectangle split part fill={white,white,white,white,white,white,mygreen},below=of shape1] (shape2) {};
%	\node[myshape,rectangle split part fill={white,white,white,white,myblue},below=of shape2] (shape3) {};
%	\node[mytri,left=of shape2](in) {};
%	\node[draw,circle,inner sep=0.6cm,right=of shape2](out) {};
%	
%	% the labels
%	\node[align=center,anchor=south east]at ([yshift=10pt]in.north west) {Classify \\ arrivals};
%	\foreach \Valor in {1,2,3}{
%		\node[anchor=west,fill=white] at (shape\Valor.east) {$W_{\Valor}$};
%	}
%	\node[anchor=south] at (out.north){Link};
%	\node[anchor=north] at (end|-out.south){Departures};

%% the arrows
%	\draw (out.east) -- ++(15pt,0pt) coordinate (end);
%	\foreach \Ancla/\Color in {{north west}/myred,west/mygreen,{south west}/myblue}{
%		\draw[line width=1.5pt,\Color,->] ([xshift=-30pt]in.\Ancla) -- ([xshift=-5pt]in.\Ancla);
%	}
%	\foreach \Valor/\Color in {1/myred,2/mygreen,3/myblue}{
%		\draw[line width=1.5pt,->,\Color,shorten <= 4pt] (in.east) -- (shape\Valor.west);  
%		\draw[line width=1.5pt,\Color,shorten <= 4pt] (shape\Valor.east) -- (out.west) ;  
%	}
%	\foreach \Ancla/\Color in {{north west}/myred,west/mygreen,{south west}/myblue}{
%		\draw[line width=1.5pt,\Color,->] ([xshift=5pt]end|-out.\Ancla) -- ([xshift=30pt]end|-out.\Ancla);
%	}
%\end{tikzpicture}

%\begin{tikzpicture}[>=latex]
%	\node at (2.75,-0.75cm) {$\mu$};
%	\node[align=center] at (1cm,-2cm) {Waiting \\ Area};
%	\node[align=center] at (3cm,-2cm) {Service \\ Node};
%	
%	\draw[<-] (0  ,-0.75) -- +(-20pt,0) node[left] {$\lambda$};
%	\draw[->] (3.5,-0.75) -- +(20pt,0 );
%	\draw (0,0) -- ++(2cm,0) -- ++(0,-1.5cm) -- ++(-2cmx,0);
%	\foreach \i in {1,...,4}
%		\draw (2cm-\i*10pt,0) -- +(0,-1.5cm);
%	\draw (2.75,-0.75cm) circle [radius=0.75cm];
%	\draw [decorate,decoration={brace,amplitude=10pt},xshift=-4pt,yshift=0pt] (0.5,0.5) -- (0.5,5.0) node [black,midway,xshift=-0.6cm] {\footnotesize $P_1$};
%\end{tikzpicture}

\FigureH{!htb}{.45}{sensors.png}{Sensors diversity}{data_analysis.jpg}{Data diversity}{sensor2.png}{Sensors \& data diversity}
\Figure{!htb}{.7}{LPWAN.png}{Communication diversity}

\Figure{!htb}{1}{Filtres.png}{Filtres \cite{merdrignac_systeme_2015}}
\Figure{!htb}{1}{classification.png}{classification \cite{merdrignac_systeme_2015}}
\Figure{!htb}{1}{sdn-wise.png}{LPWAN connectivity}
\Figure{!htb}{1}{interoperability-stack.jpg}{Interoperability}
\Figure{!htb}{1}{stat_interoperability.png}{Key barriers in adopting the Industrial Internet \footfullcite{industrialinternetofthings_executive_}}
\Figure{!htb}{1}{wsn-IoT.png}{wsn-IoT}


\begin{bytefield}[bitwidth=2.1em]{16}
\bitheader{0-15}                                                                                                               \\
\begin{rightwordgroup}{802.15.5 Header}                                                                                        \\
\y{4}{Length}     & \y{4}{FCF}             & \y{4}{DSN}               & \y{4}{DST PAN}                                      \\
\y{16}{Destination address}                                                                                                \\
\y{16}{Source address}                                                                                                     \\
\end{rightwordgroup}                                                                                                           \\

\begin{rightwordgroup}{Mesh addressing Header}                                                                                 \\
\y{1}{I}       & \y{1}{O}             & \y{1}{S}               & \y{1}{D}   & \y{4}{Hop Limit}                             \\
\y{16}{Source address}                                                                                                     \\
\y{16}{Destination address}                                                                                                \\
\end{rightwordgroup}                                                                                                           \\

\begin{rightwordgroup}{Fragment Header}                                                                                        \\
\y{1}{I}       & \y{1}{I}             & \y{1}{O}               & \y{2}{rsv} & \y{11}{Datagrame size}                       \\
\y{16}{Datagrame Tag}                                                                                                      \\
\y{8}{Datagrame Offset}                                                                                                    \\
\end{rightwordgroup}                                                                                                           \\

\begin{rightwordgroup}{IPv6 Header}                                                                                            \\
\y{5}{Version} & \y{5}{Traffic Class} & \y{6}{Flow Label}                                                                  \\
\y{8}{Length}  & \y{4}{Next Header}   & \y{4}{Hope limit}                                                                  \\
\y{16}{Source address}                                                                                                     \\
\y{16}{Destination address}                                                                                                \\
\end{rightwordgroup}                                                                                                           \\

\begin{rightwordgroup}{IPv6 Payload}                                                                                           \\
\y{2}{V=2}     & \y{1}{P}             & \y{1}{X}                & \y{4}{CC}  & \y{1}{M}                         & \y{7}{PT}\\
\y{16}{timestamp}                                                                                                          \\
\end{rightwordgroup}                                                                                                           \\

\begin{rightwordgroup}{UDP Header}                                                                                             \\
\y{8}{Source port}     & \y{8}{Destination port}                                                                           \\
\y{8}{Length}                & \y{8}{Checksum}                                                                             \\
\end{rightwordgroup}                                                                                                           \\
\end{bytefield}



\begin{table}[h!]
\scriptsize
	\begin{tabulary}{\textwidth}{L|L}
	\multirow{3}{*}{Naïve modes}           & Instantaneous  \\
	\                                      & Hist. average  \\
	\                                      & Clustering     \\\hline
	\multirow{9}{*}{Parametric models}     & Rarely used    \\
	\                                      & Traffic Models \\
	\                                      & Time Series    \\
	\                                      & Linear regression \\
	\                                      & ARIMA \\
	\                                      & Kalman filtering \\
	\                                      & ATHENA \\
	\                                      & SETAR \\
	\                                      & Gaussian Maximum Likelihood \\\hline

	\multirow{6}{*}{Non-Parametric models} & k-Nearest Neighbor           \\
	\                                      & Locally Weighted Regression \\
	\                                      & Fuzzy Logic    \\
	\                                      & Bayes Network  \\
	\                                      & Neural Network \\
	\                                      & Include temporal/spatial patterns \\
	\end{tabulary}
	\caption{\label{tab:models} Taxonomy of prediction models \cite{_short_2007}}
\end{table}


\begin{itemize}
	\item[\cite{qin_software_2014}] Many studies have identified \green{SDN} as a potential solution to the WSN challenges,
	as well as a model for \red{heterogeneous} integration.
	\item[\cite{qin_software_2014}] This \red{shortfall} can be resolved by using the \green{SDN approach.}
	\item[\cite{kobo_survey_2017}] \green{SDN} also enhances better control of \red{heterogeneous} network infrastructures.
	\item[\cite{kobo_survey_2017}] Anadiotis et al. define a \green{SDN operating system for IoT} that integrates SDN based WSN \textbf{(SDN-WISE)}.
		This experiment shows how \red{heterogeneity} between different kinds of SDN networks can be achieved.
	\item[\cite{kobo_survey_2017}] In cellular networks,
		OpenRoads presents an approach of introducing \green{SDN} based \red{heterogeneity} in wireless networks for operators.
	\item[\cite{ndiaye_software_2017}] There has been a plethora of (industrial) studies \green{synergising SDN in IoT}.
			The major characteristics of IoT are low latency,wireless access, mobility and \red{heterogeneity}.
	\item[\cite{ndiaye_software_2017}] Thus a bottom-up approach application of \green{SDN} to the realisation of \red{heterogeneous IoT} is suggested.
	\item[\cite{ndiaye_software_2017}] Perhaps a more complete IoT architecture is proposed,
			where the authors apply \green{SDN} principles in IoT \red{heterogeneous} networks.
	\item[\cite{bera_softwaredefined_2017}] it provides the \green{SDWSN} with a proper model of network management,
			especially considering the potential of \red{heterogeneity} in SDWSN.
	\item[\cite{bera_softwaredefined_2017}] We conjecture that the \green{SDN paradigm} is a good candidate to solve the \red{heterogeneity} in IoT.
\end{itemize}


\begin{table}[h!]
\scriptsize
	\begin{tabulary}{\columnwidth}{L|L|C|C|C|C|C}
	\textbf{Management architecture}                 & \textbf{Management feature}            & \textbf{Controller configuration} & \textbf{Traffic Control} & \textbf{Configuration and monitoring} & \textbf{Scapability and localization} & \textbf{Communication management}\\\hline
	\textbf{\cite{luo_sensor_2012} Sensor Open Flow} & SDN support protocol                   & Distributed                       & in/out-band              & \ok                                   & \ok                                   & \ok                              \\\hline
	\textbf{\cite{costanzo_software_2012} SDWN}      & Duty sycling, aggregation, routing     & Centralized                       & in-band                  & \ok                                   &                                       & \\\hline
	\textbf{\cite{galluccio_sdnwise_2015} SDN-WISE}  & Programming simplicity and aggregation & Distributed                       & in-band                  &                                       & \ok                                   & \\\hline
	\textbf{\cite{degante_smart_2014} Smart}         & Efficiency in resource allocation      & Distributed                       & in-band                  &                                       & \ok                                   & \\\hline
	\textbf{SDCSN}                                   & Network reliability and QoS            & Distributed                       & in-band                  &                                       & \ok                                   & \\\hline
	\textbf{TinySDN}                                 & In-band-traffic control                & Distributed                       & in-band                  &                                       & \ok                                   & \\\hline
	\textbf{Virtual Overlay}                         & Network flexibility                    & Distributed                       & in-band                  &                                       & \ok                                   & \\\hline
	\textbf{Context based}                           & Network scalability and performance    & Distributed                       & in-band                  &                                       & \ok                                   & \\\hline
	\textbf{CRLB}                                    & Node localization                      & Centralized                       & in-band                  &                                       &                                       & \\\hline
	\textbf{Multi-hope}                              & Traffic and energy control             & Centralized                       & in-band                  &                                       &                                       & \ok                              \\\hline
	\textbf{Tiny-SDN}                                & Network task measurement               & -                                 & in-band                  &                                       &                                       & \\
	\end{tabulary}
	\caption{\label{tab:Tableuy} SDN-based network and topology management architectures. \cite{ndiaye_software_2017}}
\end{table}

\begin{table}[h!]
\scriptsize
	\begin{tabular}{l|l|l|l|l}
	Application & CoAP, MQTT          &                     &                       &            \\\hline
	Transport   & \multicolumn{4}{c}{UDP/TCP}                                                    \\\hline
	Network     & IPv6 RPL            & \multicolumn{3}{c}{IPv4/IPv6}                            \\\hline
	\           & 6LowPan             & \multicolumn{2}{c}{RFC 2464}                & RFC 5072   \\\hline
	MAC         & IEEE 802.15.4       & IEEE 802.11 (Wi-Fi) & IEEE 802.3 (Ethernet) & 2G, 3G, LTE\\\hline
	\           & 2.4GHz, 915, 868MHz & 2.4, 5GHz           &                       &            \\\hline
	\           & DSS, FSK, OFDM      & CSMA/CA             & UTP, FO               &            \\\hline
	\end{tabular}
	\caption{\label{tab:Table} An example table.}
\end{table}

\begin{table}
\scriptsize
	\begin{tabulary}{\textwidth}{L|C|C|C|C|C|C|C}
		\bf{Application protocol}                                                 & DDS                                     & CoAP                              & AMQP                              & MQTT                                & MQTT-SN & XMPP & HTTP\\\hline
		\bf{Service discovery}     & \multicolumn{3}{c}{mDNS}                & \multicolumn{4}{c}{DNS-SD}                                                                                                         \\
		\bf{Network layer}         & \multicolumn{7}{c}{RPL}                                                                                                                                                      \\
	%			\bf{Network layer}         & \multicolumn{3}{c}{6LoWPAN}             & \multicolumn{4}{c}{IPv4/IPv6}                                                                                                      \\
		\bf{Link layer}            & \multicolumn{7}{c}{IEEE 802.15.4}                                                                                                                                            \\
		\bf{Physical layer}        & \multicolumn{3}{c}{EPCglobal}     & \multicolumn{2}{c}{IEEE 802.15.4} & \multicolumn{2}{c}{Z-Wave}                                 \\\hline
	\end{tabulary}
	\caption{\label{tab:Tablej} Standardization efforts that support the IoT}
\end{table}

\begin{table}[h!]
\scriptsize
\begin{tabulary}{\textwidth}{L|L|L|L|L}
	\                                  & \bf{LiteOS}                & \bf{Nano-RK}                  & \bf{MANTIS}          & \bf{Contiki} \\\hline
	\bf{Architecture}                  & Monolithic                 & Layered                       & Modular              & Modular \\\hline
	\bf{Scheduling Memory}             & Round Robin                & Monotonic harmonized          & Priority classes     & Interrupts execute w.r.t. \\\hline
	\bf{Network}                       & File                       & Socket abstraction            & At Kernel COMM layer & uIP, Rime \\\hline
	\bf{Virtualization and Completion} & Synchronization primitives & Serialized access  semaphores & Semaphores           & Serialized, Access \\\hline
	\bf{Multi threading}               & \ok                        & \ok                           & \ko                  & \ok \\\hline
	\bf{Dynamic protection}            & \ok                        & \ko                           & \ok                  & \ok \\\hline
	\bf{Memory Stack}                  & \ok                        & \ko                           & \ko                  & \ko \\\hline
\end{tabulary}
\caption{\label{tab:OS} Common operating systems used in IoT environment \cite{al-fuqaha_internet_24}}
\end{table}









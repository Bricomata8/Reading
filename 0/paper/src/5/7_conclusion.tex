\section{Conclusion} \label{sec:Conclusion}

% Restate the main challenges
The main challenge of this work is to investigate the possibility of using genetic algorithm to model the selection of lora configuration that satisfy the applications requirements.
% Restate the main contribution
Our main contribution was to develope 3 applications that requires different level of QoS.
such as text transmition, souns transmission and image transmmission. We used a low cost lora gateway on raspberrypi and builded 2 arduino boards equiped with 2 lora Tranceivers.
The gateway capture effect is based on the SX1276 transceiver.
% Restate the main findings
To measue the accuracy of applying genetic algorithm in the edge computing to select the best lora configuration we used both simulation and real enviroment tesbeds.
Our simulations compare the performance of each configuration selection 


homogeneous and heterogeneous deployments as a function of the number of nodes and traffic intensity.
First,
	we have analyzed homogeneous deployments for different SFs from SF6 to SF12.
Simulations show better performance for the SF6 deployment but reduced cell coverage.
Second,
	we have compared heterogeneous and homogeneous deployments:
	the Heterogeneous deployment that selects its LoRa configuration according to its link budget results in the best PDR and throughput,
	as well as the lowest average power consumption compared to other deployments,
	for a different number of nodes and different traffic intensity.
The results clearly show the benefits of heterogeneity for large scale network deployments and the need for adaptive SF allocation strategies.
% Future challenges current bad state
As a future work,
	we plan to validate our appproach by using both simulation and real enviroment testbeds. 
We plane to study the performance of applying our approach in terms of PDR,
	throughput,
	and power consumption.
The module will optimally select the configuration according to the scenario criteria (e.g.,
	high data rate,
	energy efficiency,
	or network congestion) and the radio environment (e.g.,
	link budget,
	level of interference,
	device mobility).


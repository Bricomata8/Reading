\begin{abstract}

%Cause1
The exponential usage of messaging services for communication raises many questions in privacy fields.
%Cause2
Privacy issues in messaging services strongly depend on the graph-theoretical properties of users' interactions representing the real friendships between users.
% Problem
One of the most important issues of privacy is that users may disclose information of other users beyond the scope of the interaction,
	without realizing that such information could be aggregated to reveal sensitive information.
% Motivation
Our. 
% Goal
Our goal is to.
% Challenges
Determining vulnerable interactions from non-vulnerable ones is difficult due to the lack of awareness mechanisms.
% Approach1
To address this problem,
	we analyze the topological relationships with the level of trust between users to notify each of them about their vulnerable social interactions.
% Approach2
Particularly,
	we analyze the impact of trusting vulnerable friends in infecting other users' privacy concerns by modeling a new vulnerability contagion process.
% Results1
Simulation results show that over-trusting vulnerable users speeds the vulnerability diffusion process through the network.
% Results2
Furthermore,
	vulnerable users with high reputation level lead to a high convergence level of infection,
	this means that the vulnerability contagion process infects the biggest number of users when vulnerable users get a high level of trust from their interlocutors.
% Validation
We validate our approach by using real messages from Enron dataset.

\end{abstract}

%%%%%%%%%%%%%%%%%%%%%%%%%%%%%%%%%%%%%%

% % Cpntext, Needs, Requirements, Constraints by Statistics (2-3 lines)
% The need of Low power wide area network (LPWAN) networks increased quickly these last years. 
% The main factor is that IoT devices require low power consumption to transmit data in a wide area.
% Lora, Sigfox and NB-IoT are the most known technologies that satisfy these requirements.
% Applications like smart building and smart environment are one of hundreds use cases that need to be deployed with such technologies.
% Unlike Sigfox and NB-IoT, Lora is more open for academic research because the specification that governs how the network is managed is relatively open.
% LoRa is a wireless modulation technique that uses Chirp Spread Spectrum (CSS) in combination with Pulse-Position Modulation (PPM).
% The transmission could be configured with 4 parameters: Spreading factor (SF), Transmission power (Tx), Coding rate (CR) and Bandwidth (Bw), to achieve better performance.
% % Clear Problem (4-5 lines)
% The problem addressed in this work is ...
% Thus heterogeneous network deployments and Spreading Factor (SF) allocation strategies need to be studied.
% In this paper,
% 	we investigate the performance of homogeneous networks (i.e.
% when all the nodes select the same LoRa configuration) and heterogeneous networks (i.e.
% when each node selects its LoRa configuration according to its link budget or their needs) for large scale deployments (up to 10000 nodes per gateway).
% %Motivation
% Our work is motivated by 
% %Goal
% Our goal is to 
% % Challenges and Difficulties (1-2 lines)
% ... is a challenging problem 
% The main LPWAN research directions are about large scale networks to support massive number of devices,
% 	interference issues,
% 	link optimization and adaptability.
% % Clear Contribution (3-4 lines)
% For that purpose we have developed a LoRa Module,
% 	based on improved WSNet simulator,
% 	including a spectrum usage abstraction,
% 	the co-channel rejection due to the quasi-orthogonality of SFs and the gateway capture effect.
% % Experimentation & results (4-5 lines)
% Simulation results show the performance comparison in terms of reliability,
% 	network capacity and power consumption for homogeneous and heterogeneous deployments as a function of the number of nodes and the traffic intensity.
% The comparison shows the benefits of the heterogeneous deployment where each node selects its configuration according to its link budget.







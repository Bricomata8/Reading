\section{Context \& motivation}


The exponential growth of 5G networks and the development of IoT that will greatly come with it,
	would considerably raise the number of Smart Cities applications.
The aim of such technology is mainly to improve the comfort and the safety of users through wireless IoT networks.
Wireless Sensor Networks (WSNs) are the source of sensed data of cities things,
	i.e.
roads,
	cars,
	pedestrians,
	houses,
	parking,
	etc.
The cloud is the entity that collects the sensed data and allows users and machines to do data analysis and improve services.
For Smart Cities,
	one objective is improving the welfare of citizens as well as its safety getting real-time information about the city infrastructure.
One application would be the transportation systems,
	and traffic lights control having as an objective avoids congestion and dangerous situations.
A static cycle of traffic lights has a direct impact on traffic jams.
The long period at red or green light could impact the fluidity of the city traffic.
The Internet of Things (IoT) would give an answer to the required interoperability between heterogeneous wireless networks.
Our objective is to model,
	prototype and evaluate a traffic control system.
Indeed,
	different infrastructures have different purposes and technologies,
	this means that it is not possible to state communication between two infrastructures following a Device-to-Device approach.
However,
	thinking of an indirect or Device-to-Cloud communication between infrastructures seems useful when every connected system has its own technologies,
	e.g.
Zigbee,
	LoRa,
	SigFox,
	ITS-G5.
Consequently,
	IP stack would be the suitable mediator for interconnecting these networks.
It removes the barriers of rigid standard specifications of the hardware despite the overhead of the extra network configuration.
Furthermore,
	we want to have a scalable solution not limited only on the traffic light management system.
We can deploy sensors and actuators to measure noise or air pollution via panels or roads and offer new services,
	e.g.
where and when jogging is better.
To implement our Urban Traffic Light Control based on an IoT network architecture (IoT-UTLC),
	we setting a real IEEE 802.15.4 WSN devices that would act as actuators and sensors.
All these small traffic light devices are driven by a Border Router (BR) which is a gateway to the Internet.
This BR 

\section{Methodology and contributions}

\section{Organization of the thesis}





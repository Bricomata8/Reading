\chapter{Network}
%\changefontsizes{pt}

\section{protocols} % (fold)
\label{sec:protocols}

% section protocols (end)
\begin{table}[h!]
\scriptsize
	\begin{tabulary}{\textwidth}{L|L|L|L|L}
		\bf{Routing protocol}  & \bf{Control Cost} & \bf{Link Cost} & \bf{Node Cost} \\\hline
		\bf{OSPF/IS-IS}        & \ko               & \ok            & \ko      \\
		\bf{OLSRv2}            & ?                 & \ok            & \ok      \\
	%		\bf{TBRPF}             & \ko               & \ok            & ?        \\
		\bf{RIP}               & \ok               & ?              & \ko      \\
	%		\bf{AODV}              & \ok               & \ko            & \ko      \\
	%		\bf{DYMO}              & \ok               & ?              & ?        \\
		\bf{DSR}               & \ok               & \ko            & \ko      \\
		\bf{RPL}               & \ok               & \ok            & \ok      \\\hline
	\end{tabulary}
	\caption{\label{tab:routingsComaprson} Routing protocols comparison \cite{_rpl2_}}
\end{table}


\begin{itemize}
	\item Routing over low-power and lossy links (ROLL)
	\item Support minimal routing requirements.
	\begin{itemize}
		\item like multipoint-to-point, point-to-multipoint and point-to-point.
	\end{itemize}
	\item A Destination Oriented Directed Acyclic Graph (DODAG)
	\begin{itemize}
		\item Directed acyclic graph with a single root.
		\item Each node is aware of ts parents 
		\item but not about related children
	\end{itemize}
	\item RPL uses four types of control messages
	\begin{itemize}
		\item DODAG Information Object (DIO)
		\item Destination Advertisement Object (DAO)
		\item DODAG Information Solicitation (DIS)
		\item DAO Acknowledgment (DAO-ACk)
	\end{itemize}
	%				\item RPL routers work under one of two modes:
	%					\begin{itemize}
	%						\item Non-Storing mode
	%						\item Storing modes mode
	%					\end{itemize}
\end{itemize}


\begin{itemize}
	\item Standard topologies to form IEEE 802.15.4e networks are 
	\begin{itemize}
		\item[Star] contains at least one FFD and some RFDs
		\item[Mesh] contains a PAN coordinator and other nodes communicate with each other
		\item[Cluster] consists of a PAN coordinator, a cluster head and normal nodes.
	\end{itemize}
	\item The IEEE 802.15.4e standard supports 2 types of network nodes
	\begin{itemize}
		\item[FFD] Full function device: serve as a coordinator
		\begin{itemize}
			\item It is responsible for creation, control and maintenance of the net
			\item It store a routing table in their memory and implement a full MAC
		\end{itemize}
		\item[RFD] Reduced function devices: simple nodes with restricted resources
		\begin{itemize}
			\item They can only communicate with a coordinator
			\item They are limited to a star topology
		\end{itemize}
	\end{itemize}
\end{itemize}
\begin{tabulary}{\textwidth}{|C|C|C|C|C|C|C|C|C|}\hline
	Preamble & PHDR & PHDRCRC & MHDR & FHDR & FPort & Payload & MIC & CRC \\\hline
\end{tabulary}



\begin{table}[h!]
\scriptsize
	\begin{tabulary}{\textwidth}{L|L|L|L|L}
		\bf{Routing protocol}  & \bf{Control Cost} & \bf{Link Cost} & \bf{Node Cost} \\\hline
		\bf{OSPF/IS-IS}        & \ko               & \ok            & \ko      \\
		\bf{OLSRv2}            & ?                 & \ok            & \ok      \\
	%		\bf{TBRPF}             & \ko               & \ok            & ?        \\
		\bf{RIP}               & \ok               & ?              & \ko      \\
	%		\bf{AODV}              & \ok               & \ko            & \ko      \\
	%		\bf{DYMO}              & \ok               & ?              & ?        \\
		\bf{DSR}               & \ok               & \ko            & \ko      \\
		\bf{RPL}               & \ok               & \ok            & \ok      \\\hline
	\end{tabulary}
	\caption{\label{tab:routingsComaprison} Routing protocols comparison \cite{_rpl2_}}
\end{table}







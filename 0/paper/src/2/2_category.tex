\section{QoS Metrics}
\label{cha:metrics}


\begin{table}[h!]
\centering
\scriptsize
	\begin{tabulary}{\textwidth}{L|L|L|L}
	\                    & \textbf{Board}  & \textbf{GW} & \textbf{API}  \\\hline
	\textbf{Application} &                 &             &               \\\hline
	\textbf{Network}     &                 &             &               \\\hline
	\textbf{MAC}         &                 &             &               \\\hline
	\textbf{Radio}       &                 &             &               \\\hline
	\end{tabulary}
\caption{\label{tab:ygxps} }
\end{table}


Requirements:
\Itemize{
\item Latency/ Reliability Requirements
\item Message Size Requirements
\item Frequency Requirements
\item Range Requirements
\item Speed Requirements
\item Security Requirements
}

\begin{table}[h!]
%\scriptsize
	\begin{tabular}{l|l|l|l}
	\textbf{Metrics}                                & \textbf{Parameters}     & \textbf{Type} & \textbf{Expected as}  \\\hline
	\multirow{4}{*}{\textbf{Network conditions}}       & network load            & Dynamic       & Minimized  \\
	\                                                  & network coverage        & Static        & Fixed      \\
	\                                                  & network connection time & Dynamic       & Minimized  \\
	\                                                  & available bandwidth     & Dynamic       & Minimized  \\\hline
	\multirow{5}{*}{\textbf{Application requirements}} & throughput              & Dynamic       & Minimized  \\
	\                                                  & delay                   & Dynamic       & Minimized  \\
	\                                                  & jitter                  & Dynamic       & Minimized  \\
	\                                                  & PLR                     & Dynamic       & Minimized  \\
	\                                                  & energy consumption      & Dynamic       & Minimized  \\\hline
	\multirow{2}{*}{\textbf{User preferences}}         & budget                  & Static        & Fixed      \\
	\                                                  & cost                    & Static        & Fixed      \\
	\                                                  & design                  &               &            \\\hline
	\multirow{2}{*}{\textbf{Mobile equipment}}         & energy                  & Dynamic       & Fixed      \\
	\                                                  & mobility                & Dynamic       & Fixed      \\\hline
	\end{tabular}
	\caption{\label{tab:network_selection} Network selection inputs and classification of parameters \cite{bendaoud_network_2019}}
\end{table}

\begin{table}[h!]
%\scriptsize
	\begin{tabulary}{\textwidth}{L|L|L}
	\textbf{Application layer} & \textbf{Network layer} & \textbf{Sensing layer }        \\\hline
	Service time               & Bandwidth              & \textbf{Energy consumption}                  \\
	Service availability       & Packet loss            & Sleep management          \\
	Service cost               & Jitter                 & Life time management                  \\
	Service reliability        & Delay                  & \textbf{Coverage}                   \\
	\                          & Availability           & Sensing area              \\
	\                          &                        & \textbf{Information accuracy}  \\
	\                          &                        & Data accuracy                  \\
	\                          &                        & Sensing time accuracy          \\
	\                          &                        & Spatial accuracy               \\
	\                          &                        & Reduce data redundancy         \\
	\                          &                        & Data packaging                 \\
	\                          &                        &                                \\\hline
	\                          &                        & Sampling rate                               \\\hline
	\                          &                        & Bit rate error                               \\\hline
	\                          &                        &                                \\\hline
	\                          &                        &                                \\\hline
	\                          &                        &                                \\\hline
	\                          &                        &                                \\\hline

	\end{tabulary}
\caption{\label{tab:} QoS parameters \cite{meshinchi_qosaware_2018} \cite{chowdhury_survey_2018}}
\end{table}

\begin{table}[h!]
\scriptsize
	\begin{tabulary}{\textwidth}{L|L|L}
		\textbf{Plan de controle}   & \textbf{Plan de gestion}   & 	\textbf{Plan de doonées}  \\\hline
	Controle d'admission      & Controle et supervision de QoS & Controle du trafic         \\
	Réservation de ressources & Gestion de contrats            & Façonnage du trafic        \\
	Routage                   & QoS mapping                    & Controle de congestion     \\
	Signalisation             & Politique de QoS               & Classification de paquets  \\
	\                         &                                & Marquage de paquets        \\
	\                         &                                & Ordonnancements des paquets\\
	\                         &                                & Gestion de files d'attente \\
	\end{tabulary}
	\caption{\label{tab:qos} An example table.}
\end{table}

\begin{table}[h!]
%\scriptsize
	\begin{tabulary}{\textwidth}{L|L}
	\textbf{Maximize (Reward)}         & \textbf{Minimize (Cost)}             \\\hline
	(\textbf{T}) Throughput            & (\textbf{RT}) Response Time            \\
	(\textbf{F}) Fairness              & (\textbf{LT}) Latency                  \\
	(\textbf{R}) Reliability           & (\textbf{J}) Jitter                    \\
	(\textbf{IA}) Information Accuracy & (\textbf{TF})  Traffic                 \\
	(\textbf{Cov}) Coverage of IoT     & (\textbf{AWT}) Average Waiting Time    \\
	(\textbf{NL}) Network Life         & (\textbf{D})  Delay                    \\
	(\textbf{RU}) Resource Utilization & (\textbf{L}) Load                      \\
	\                         & (\textbf{EC}) Energy Consumption       \\
	\                         & (\textbf{BP})  Blocking Probability    \\
	\                         & (\textbf{CCI})  Co-channel Interference\\
	\                         & (\textbf{SC}) Service Cost             \\
	\                         & (\textbf{ST})   Service Time           \\
	\end{tabulary}
\caption{\label{tab:scheduling} Objectives of IoT resource scheduling}
\end{table}

\section{Validation}



\begin{table}[h!]
\scriptsize
	\begin{tabulary}{\textwidth}{L|L}
	\multirow{3}{*}{Naïve modes}           & Instantaneous  \\
	\                                      & Hist. average  \\
	\                                      & Clustering     \\\hline
	\multirow{9}{*}{Parametric models}     & Rarely used    \\
	\                                      & Traffic Models \\
	\                                      & Time Series    \\
	\                                      & Linear regression \\
	\                                      & ARIMA \\
	\                                      & Kalman filtering \\
	\                                      & ATHENA \\
	\                                      & SETAR \\
	\                                      & Gaussian Maximum Likelihood \\\hline

	\multirow{6}{*}{Non-Parametric models} & k-Nearest Neighbor           \\
	\                                      & Locally Weighted Regression \\
	\                                      & Fuzzy Logic    \\
	\                                      & Bayes Network  \\
	\                                      & Neural Network \\
	\                                      & Include temporal/spatial patterns \\
	\end{tabulary}
	\caption{\label{tab:models} Taxonomy of prediction models \cite{_short_2007}}
\end{table}    


\begin{align}
\ac{MSE}=         & \frac{1}{n} \sum_{i=1}^{n}\left(p_{i}-r_{i}\right)^{2} \\
\ac{RMSE}=        & \sqrt{\frac{1}{n} \sum_{i=1}^{n}\left(p_{i}-r_{i}\right)^{2}} \\
\ac{MAE}=         & \frac{1}{n} \sum_{i=1}^{n}\left|p_{i}-r_{i}\right| \\
Recall=      & \frac{T P}{T P+F N} \\
Precision=   & \frac{T P}{T P+F P} \\
F1_{-}Score= & \frac{2 \times \text { Precision } \times \text { Recall}}{\text { precision }+\text {recall}} \\
\ac{TPR} =        & \frac{T P}{T P+F N} \\
\ac{FPR} =        & \frac{F P}{F P+T N} \\
\ac{ROC}=& (TPR,FPR)\\
Novelty=     & \sum_{i \in L} \frac{\log _{2} P_{i}}{n} \text { where } \quad P_{i}=\frac{n-r a n k_{i}}{n-1} \\
Serendipity= & \frac{1}{n} \sum_{i \in n} \max \left(P_{\text {user}}-P_{U}, 0\right) \times r e l_{i} \\
diversity=   & \frac{a}{c} \sum_{i=1}^{c} \frac{1}{n} \sum_{j=1}^{n} i_{j} \\
Coverage=    & 100 \times \frac{u}{U} \\
Stability=   & \frac{1}{P_{2}} \sum_{i \in P_{2}}\left|P_{2, i}-P_{1, i}\right| \\
DCG=         & r e l_{1}+\sum_{i=2}^{\text {pos}} \frac{r e l_{i}}{\log _{2} i} \\
IDCG=        & r e l_{1}+\sum_{i=2}^{|h|-1} \frac{r e l_{i}}{\log _{2} i} \\
\ac{NDCG}=        & \frac{D C G}{I D C G} \\
\end{align}




Gateway selection
Input:
Method:
	Ranking machine learning
Output:
	Ranked list of gateway







% \FigureH{!htb}{.45}{sensors.png}{Sensors diversity}{data_analysis.jpg}{Data diversity}{sensor2.png}{Sensors \& data diversity}

\Figure{!htb}{1}{Filtres.png}{Filtres \cite{merdrignac_systeme_2015}}
\Figure{!htb}{1}{classification.png}{classification \cite{merdrignac_systeme_2015}}
\Figure{!htb}{1}{interoperability-stack.jpg}{Interoperability}





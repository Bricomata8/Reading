
\chapter{State of the art \cite{bregell_hardware_2015}} % (fold)


\section{IoT Uses cases} % (fold)
\label{sec:section_name}

\subsection{Transportation and logistics}
\subsection{Healthcare}
\subsection{Smart environnement}
\subsection{personal and social}
\subsection{Futuristic}
\subsection{Summary and discussion}



\Figure{!htb}{1}{market.jpg}{}

\Figure{!htb}{1}{standard.png}{802.15.4 use cases \cite{sarwar_iot_}}



% section section_name (end)

%%\changefontsizes{5pt}
\begin{table}[h!]
\scriptsize
	\begin{tabular}{l|l|l|l}
	\textbf{Use cases}         &  &  & \\\hline
	Health Monitoring          &  &  & \\\hline
	Water Distribution         &  &  & \\\hline
	Electricity Distribution   &  &  & \\\hline
	Smart Buildings            &  &  & \\\hline
	Intelligent Transportation &  &  & \\\hline
	Surveillance               &  &  & \\\hline
	Environmental Monitoring   &  &  & \\
	\end{tabular}
	\caption{\label{tab:IoTUseCase} Use cases \cite{hancke_role_2012}}
\end{table}


\begin{table}[h!]
\scriptsize
	\begin{tabulary}{\textwidth}{L|C|C|C|C|C}
	Callenges-Applications  & Gids & EHealth & Transportations & Cities & \textbf{Building}\\\hline
	Ressources cinstraints  & +           & +++     & -               & ++           & +            \\\hline
	Mobility                & +           & ++      & +++             & +++          & -            \\\hline
	\textbf{Heterogeneity}  & ++          & ++      & ++              & +++          & +            \\\hline
	Scalability             & +++         & ++      & +++             & +++          & ++           \\\hline
	QoS cinstraints         & ++          & ++      & +++             & +++          & +++          \\\hline
	Data management         & ++          & +       & +++             & +++          & ++           \\\hline
	Lack of standardization & ++          & ++      & ++              & ++           & +++          \\\hline
	Amount of attacks       & +           & +       & +++             & +++          & +++          \\\hline
	Safety                  & ++          & ++      & +++             & ++           & +++          \\\hline
	\end{tabulary}
\caption{\label{tab:iot_challenges} Main IoT challenges\cite{kouicem_internet_2018}}
\end{table}


voir \cite{rizzi_evaluation_2017}
\begin{table}[h!]
\scriptsize
	\begin{tabulary}{\textwidth}{L|C|C|L}
	\textbf{Use Case}                      & \textbf{Packet rate (λ) [packet/day]} & \textbf{Minimum success rate (Ps,min )} & \textbf{Grouping}\\\hline
	\textbf{Wearables}                     & 10                           &        90                        & \multirow{5}{*}{Group A PL = 10/20B} \\
	\textbf{Smoke Detectors}               & 2                            &        90                        &         \\
	\textbf{Smart Grid}                    & 10                           &        90                        &         \\
	\textbf{White Goods}                   & 3                            &        90                        &         \\
	\textbf{Waste Management}              & 24                           &        90                        &         \\\hline
	\textbf{VIP/Pet Tracking}              & 48                           &        90                        & \multirow{9}{*}{Group B PL = 50B}        \\
	\textbf{Smart Bicycle}                 & 192                          &        90                        &         \\
	\textbf{Animal Tracking}               & 100                          &        90                        &         \\
	\textbf{Environmental Monitoring}      & 5                            &        90                        &         \\
	\textbf{Asset Tracking}                & 100                          &        90                        &         \\
	\textbf{Smart Parking}                 & 60                           &        90                        &         \\
	\textbf{Alarms/Actuators}              & 5                            &        90                        &         \\
	\textbf{Home Automation}               & 5                            &        90                        &         \\
	\textbf{Machinery Control}             & 100                          &        90                        &         \\\hline
	\textbf{Water/Gas Metering}            & 8                            &        90                        & \multirow{9}{*}{Group C PL = 100/200B}        \\
	\textbf{Environmental Data Collection} & 24                           &        90                        &         \\
	\textbf{Medical Assisted Living}       & 8                            &        90                        &         \\
	\textbf{Microgeneration}               & 2                            &        90                        &         \\
	\textbf{Safety Monitoring}             & 2                            &        90                        &         \\
	\textbf{Propane Tank Monitoring}       & 2                            &        90                        &         \\
	\textbf{Stationary Monitoring}         & 4                            &        90                        &         \\
	\textbf{Urban Lighting}                & 5                            &        90                        &         \\
	\textbf{Vending Machines Payment}      & 100                          &        90                        &         \\\hline
	\textbf{Vending Machines General}      & 1                            &        90                        & Group D PL = 1KB        \\
	\end{tabulary}
\caption{\label{tab:zzes}A PPLICATION REQUIREMENTS FOR THE USE CASES OF INTEREST\cite{feltrin_lorawan_2018}.}
\end{table}

Smart systems in smart cities \cite{alba_intelligent_2016}
\begin{itemize}
	\item Smart Mobility
	\item Smart semaphores controle
	\item Smart Red Swarm
	\item Smart panels
	\item Smart bus scheduling
	\item Smart EV management
	\item Smart surface parking
	\item Smart signs
	\item Smart energy systems
	\item Smart lighting
	\item Smart water jet systems
	\item Smart residuals gathering
	\item Smart building construction
	\item Smart tourism
	\item Smart QRinfo
	\item Smart monitoring
	\item Smart hawkeye
\end{itemize}

\begin{table}[h!]
\scriptsize
	\begin{tabular}{l|l|l|l|l|l|l|l}

	\bf{Application protocol}& DDS                                     & CoAP                              & AMQP                              & MQTT                                & MQTT-SN & XMPP & HTTP\\\hline
	\bf{Service discovery}   & \multicolumn{3}{c}{mDNS}                & \multicolumn{4}{c}{DNS-SD}                                                                                                         \\
	\bf{Transport}           & \multicolumn{7}{c}{UDP/TCP}                                                    \\
	Network                  & \multicolumn{2}{c}{IPv6 RPL}            & \multicolumn{5}{c}{IPv4/IPv6}                            \\\hline
	\                        & \multicolumn{2}{c}{6LowPan}             & \multicolumn{4}{c}{RFC 2464}                & RFC 5072   \\\hline
	MAC                      & \multicolumn{2}{c}{IEEE 802.15.4}       & \multicolumn{2}{c}{IEEE 802.11 (Wi-Fi)} & \multicolumn{2}{c}{IEEE 802.3 (Ethernet)} & 2G, 3G, LTE\\\hline
	\                        & \multicolumn{2}{c}{2.4GHz, 915, 868MHz} & \multicolumn{2}{c}{2.4, 5GHz}           &         &              &            \\\hline
	\                        & \multicolumn{2}{c}{DSS, FSK, OFDM }     & \multicolumn{2}{c}{CSMA/CA}             & CUTP, FO               &            \\\hline
	\end{tabular}
	\caption{\label{tab:Table} Standardization efforts that support the IoT}
\end{table}






\section{IoT Cloud Platforms, cayenne ?} % (fold)
\label{subsec:cloud_app}

% section cloud_app (end)

%\changefontsizes{5pt}
\begin{table}[h!]
\scriptsize
	\begin{tabulary}{\textwidth}{L|C|C|C|C|C|C|C|C|C}
	Paper           & Architecture & Availability & Reliability & Mobility & Performance & Management & Scalability & Interoperability & Security\\\hline
	IoT-A           &              &              &             &          &             &            &             &                  &         \\\hline
	IoT@Work        &              &              &             &          &             &            &             &                  &         \\\hline
	EBBITS          &              &              &             &          &             &            &             &                  &         \\\hline
	BETaas          &              &              &             &          &             &            &             &                  &         \\\hline
	CALIPSO         &              &              &             &          &             &            &             &                  &         \\\hline
	VITAL           &              &              &             &          &             &            &             &                  &         \\\hline
	SENSAI          &              &              &             &          &             &            &             &                  &         \\\hline
	RERUM           &              &              &             &          &             &            &             &                  &         \\\hline
	RELEYonIT       &              &              &             &          &             &            &             &                  &         \\\hline
	IoT6            &              &              &             &          &             &            &             &                  &         \\\hline
	OpenIoT         &              &              &             &          &             &            &             &                  &         \\\hline
	Apec IoV        &              &              &             &          &             &            &             &                  &         \\\hline
	Smart Santander &              &              &             &          &             &            &             &                  &         \\\hline
	OMA Device      &              &              &             &          &             &            &             &                  &         \\\hline
	OMA-DM          &              &              &             &          &             &            &             &                  &         \\\hline
	LWM2M           &              &              &             &          &             &            &             &                  &         \\\hline
	NETCONF Light   &              &              &             &          &             &            &             &                  &         \\\hline
	Kura            &              &              &             &          &             &            &             &                  &         \\\hline
	MASH            &              &              &             &          &             &            &             &                  &         \\\hline
	IoT-iCore       &              &              &             &          &             &            &             &                  &         \\\hline
	PROBE-IT        &              &              &             &          &             &            &             &                  &         \\\hline
	OpenIoT         &              &              &             &          &             &            &             &                  &         \\\hline
	LinkSmart       &              &              &             &          &             &            &             &                  &         \\\hline
	IETF SOLACE     &              &              &             &          &             &            &             &                  &         \\\hline
	BUTLER          &              &              &             &          &             &            &             &                  &         \\\hline
	Codo            &              &              &             &          &             &            &             &                  &         \\\hline
	SVELETE         &              &              &             &          &             &            &             &                  &         \\\hline
		\end{tabulary}
	\caption{\label{tab:Table54975} An example table.}
\end{table}


\begin{table}[h!]
\scriptsize
	\begin{tabulary}{\textwidth}{L|C|C|C|C}
	\bf{Platform}      & \ \bf{COAP} & \bf{XMPP} & \bf{MQTT}\\\hline
	\bf{Arkessa}       &             &           & \ok      \\\hline
	\bf{Axeda}         &             &           &          \\\hline
	\bf{Etherios}      &             &           &          \\\hline
	\bf{LittleBits}    &             &           &          \\\hline
	\bf{NanoService}   & \ok         &           &          \\\hline
	\bf{Nimbits}       &             & \ok       &          \\\hline
	\bf{Ninja blocks}  &             &           &          \\\hline
	\bf{OnePlateformv} & \ok         & \ok       &          \\\hline
	\bf{RealTime.io}   &             &           &          \\\hline
	\bf{SensorCloud}   &             &           &          \\\hline
	\bf{SmartThings}   &             &           &          \\\hline
	\bf{TempoDB}       &             &           &          \\\hline
	\bf{ThingWorx}     &             &           & \ok      \\\hline
	\bf{Xively}        &             &           & \ok      \\\hline
	\bf{Ubidots}       &             &           & \ok      \\\hline
	\end{tabulary}
	\caption{\label{tab:IoTPlatforms} IoT cloud platforms and their characteristics}
\end{table}



\section{IoT Sensors platforms}

\subsection{SDN platforms}

Sensor OpenFlow [20,21]
SDWN [60]
Smart [14]
SDN-WISE [78]
SDCSN [88]
TinySDN [69,118]
Virtual Overlay [59,87,90]
Multi-task [122]
SDWSN-RL [123]
Wireless power transfer [126]
Function alternation [65]
Statistical machine learning [24]
Context-based [91,92]
Soft-WSN [9]

\begin{itemize}
	\item[\cite{qin_software_2014}] Many studies have identified \green{SDN} as a potential solution to the WSN challenges,
	as well as a model for \red{heterogeneous} integration.
	\item[\cite{qin_software_2014}] This \red{shortfall} can be resolved by using the \green{SDN approach.}
	\item[\cite{kobo_survey_2017}] \green{SDN} also enhances better control of \red{heterogeneous} network infrastructures.
	\item[\cite{kobo_survey_2017}] Anadiotis et al. define a \green{SDN operating system for IoT} that integrates SDN based WSN \textbf{(SDN-WISE)}.
		This experiment shows how \red{heterogeneity} between different kinds of SDN networks can be achieved.
	\item[\cite{kobo_survey_2017}] In cellular networks,
		OpenRoads presents an approach of introducing \green{SDN} based \red{heterogeneity} in wireless networks for operators.
	\item[\cite{ndiaye_software_2017}] There has been a plethora of (industrial) studies \green{synergising SDN in IoT}.
			The major characteristics of IoT are low latency,wireless access, mobility and \red{heterogeneity}.
	\item[\cite{ndiaye_software_2017}] Thus a bottom-up approach application of \green{SDN} to the realisation of \red{heterogeneous IoT} is suggested.
	\item[\cite{ndiaye_software_2017}] Perhaps a more complete IoT architecture is proposed,
			where the authors apply \green{SDN} principles in IoT \red{heterogeneous} networks.
	\item[\cite{bera_softwaredefined_2017}] it provides the \green{SDWSN} with a proper model of network management,
			especially considering the potential of \red{heterogeneity} in SDWSN.
	\item[\cite{bera_softwaredefined_2017}] We conjecture that the \green{SDN paradigm} is a good candidate to solve the \red{heterogeneity} in IoT.
\end{itemize}


\begin{table}[h!]
\scriptsize
	\begin{tabulary}{\columnwidth}{L|L|C|C|C|C|C}
	\textbf{Management architecture}                 & \textbf{Management feature}            & \textbf{Controller configuration} & \textbf{Traffic Control} & \textbf{Configuration and monitoring} & \textbf{Scapability and localization} & \textbf{Communication management}\\\hline
	\textbf{\cite{luo_sensor_2012} Sensor Open Flow} & SDN support protocol                   & Distributed                       & in/out-band              & \ok                                   & \ok                                   & \ok                              \\\hline
	\textbf{\cite{costanzo_software_2012} SDWN}      & Duty sycling, aggregation, routing     & Centralized                       & in-band                  & \ok                                   &                                       & \\\hline
	\textbf{\cite{galluccio_sdnwise_2015} SDN-WISE}  & Programming simplicity and aggregation & Distributed                       & in-band                  &                                       & \ok                                   & \\\hline
	\textbf{\cite{degante_smart_2014} Smart}         & Efficiency in resource allocation      & Distributed                       & in-band                  &                                       & \ok                                   & \\\hline
	\textbf{SDCSN}                                   & Network reliability and QoS            & Distributed                       & in-band                  &                                       & \ok                                   & \\\hline
	\textbf{TinySDN}                                 & In-band-traffic control                & Distributed                       & in-band                  &                                       & \ok                                   & \\\hline
	\textbf{Virtual Overlay}                         & Network flexibility                    & Distributed                       & in-band                  &                                       & \ok                                   & \\\hline
	\textbf{Context based}                           & Network scalability and performance    & Distributed                       & in-band                  &                                       & \ok                                   & \\\hline
	\textbf{CRLB}                                    & Node localization                      & Centralized                       & in-band                  &                                       &                                       & \\\hline
	\textbf{Multi-hope}                              & Traffic and energy control             & Centralized                       & in-band                  &                                       &                                       & \ok                              \\\hline
	\textbf{Tiny-SDN}                                & Network task measurement               & -                                 & in-band                  &                                       &                                       & \\
	\end{tabulary}
	\caption{\label{tab:Tableuy} SDN-based network and topology management architectures. \cite{ndiaye_software_2017}}
\end{table}


\Figure{!htb}{1}{sdn-wise.png}{LPWAN connectivity}




% section other (end)

\begin{itemize}
	\item Network selection
	\begin{itemize}
		\item MADM
		\begin{itemize}
			\item Ranking methods
			\item Ranking \& weighted methods
		\end{itemize}
		\item Game theory
		\begin{itemize}
			\item Users vs users
			\item Users vs networks
			\item Networks vs network
		\end{itemize}
		\item Fuzzy logic
		\begin{itemize}
			\item as a score method
			\item another theory
		\end{itemize}
		\item Utility function
		\begin{itemize}
			\item 1
			\item 2
		\end{itemize}
	\end{itemize}
\end{itemize}




\subsection{Operating systems}

{LPWAN}
\Figure{}{1}{os-timeline.png}{}

The operating system is the foundation of the IoT technology as it provides the functions for the connectivity between the nodes.
However,
	different types of nodes need different levels of OS complexity;
	a passive node generally only needs the communication stack and is not in need of any threading capabilities,
	as the program can handle all logic in one function.
Active nodes and border routers need to have a much more complex OS,
	as they need to be able to handle several running threads or processes,
	e.g.
routing,
	data collection and interrupts.
To qualify as an OS suitable for the IoT,
	it needs to meet the basic requirements:
	• Low Random-access memory (RAM) footprint 
	• Low Read-only memory (ROM) footprint 
	• Multi-tasking • Power management (PM) 
	• Soft real-time These requirements are directly bound to the type of hardware designed for the IoT.
As this type of hardware in general needs to have a small form factor and a long battery life,
	the on-board memory is usually limited to keep down size and energy consumption.
Also,
	because of the limited amount of memory,
	the implementation of threads is usually a challenging task,
	as context switching needs to store thread or process variables to memory.
The size of the memory also directly affects the energy consumption,
	as memory in general is very power hungry during accesses.
To be able reduce the energy consumption,
	the OS needs some kind of power management.
The power management does not only let the OS turn on and off peripherals such as flash memory,
	I/O,
	and sensors,
	but also puts the MCU itself in different power modes.
As the nodes can be used to control and monitor consumer devices,
	either a hard or soft real-time OS is required.
Otherwise,
	actions requiring a close to instantaneous reaction might be indefinitely delayed.
Hard real-time means that the OS scheduler can guarantee latency and execution time,
	whereas Soft real-time means that latency and execution time is seen as real-time but can not be guaranteed by the scheduler.
Operating systems that meet the above requirements are compared in table 2.1 and 2.2.

\begin{table}[h!]
\scriptsize
\begin{tabular}{l|l|c|l|c|c|l|l} % Application protocol
	\bf{OS}      & \bf{Architecture} & \bf{Multi}     & \bf{Scheduling} & \bf{Dynamic}    & \bf{Memory}     & \bf{Network}  & \bf{Virtualization} \\
	\            &                   & \bf{threading} &                 & \bf{Memory}     & \bf{protection} & \bf{Stack} & \bf{and Completion}   \\\hline
	\bf{Contiki/Contiki-ng} & Modular& \ok            & Interrupts      & \ok             & \ko             & uIP          & Serialized          \\
	\            &                   &                & execute w.r.t.  &                 &                 & Rime         & Access              \\\hline
	\bf{MANTIS}  & Modular           & \ko            & Priority        & \ok             & \ko             & At Kernel    & Semaphores.         \\
	\            &                   &                & classes         &                 &                 & COMM layer   &                     \\\hline
	\bf{Nano-RK} & Layered           & \ok            & Monotonic       & \ko             & \ko             & Socket       & Serialized access   \\
	\            &                   &                & harmonized      &                 &                 & abstraction  & semaphores          \\\hline
	\bf{LiteOS}  & Monolithic        & \ok            & Round Robin     & \ok             & \ok             & File         & Synchronization     \\
	\            &                   &                &                 &                 &                 &              & primitives          \\\hline
\end{tabular}
\caption{\label{tab:OS} Common operating systems used in IoT environment \cite{al-fuqaha_internet_24}}
\end{table}

\subsubsection{Contiki}

Contiki is a embedded operating system developed for IoT written in C [12].
It supports a broad range of MCUs and has drivers for various transceivers.
The OS does not only support TCP/IPv4 and IPv6 with the uIP stack [9],
	but also has support for the 6LoWPAN stack and its own stack called RIME.
It supports threading with a thread system called Photothreads [13].
The threads are stack-less and thus use only two bytes of memory per thread;
	however,
	each thread is bound to one function and it only has permission to control its own execution.
Included in Contiki,
	there is a range of applications such as a HTTP,
	Constrained Application Protocol (CoAP),
	FTP,
	and DHCP servers,
	as well as other useful programs and tools.
These applications can be included in a project and can run simultaneously with the help of Photothreads.
The limitations to what applications can be run is the amount of RAM and ROM the target MCU provides.
A standard system with IPv6 networking needs about 10 kB RAM and 30 kB ROM but as applications are added the requirements tend to grow.

\subsubsection{RIOT}

RIOT is a open source embedded operating system supported by Freie Universität Berlin,
	INIRA,
	and Hamburg University of Applied Sciences [14].
The kernel is written in C but the upper layers support C++ as well.
As the project originates from a project with real-time and reliability requirements,
	the kernel supports hard real-time multi-tasking scheduling.
One of the goals of the project is to make the OS completely POSIX compliant.
Overhead for multi-threading is minimal with less than 25 bytes per thread.
Both IPv6 and 6LoWPAN is supported together with UDP,
	TCP,
	and IPv6 Routing Protocol for Low-Power and Lossy Networks (RPL);
	and CoAP and Concise Binary Object Representation (CBOR) are available as application level communication protocols.

\subsubsection{TinyOS}

TinyOS is written in Network Embedded Systems C (nesC) which is a variant of C [15].
nesC does not have any dynamic memory allocation and all program paths are available at compile-time.
This is manageable thanks to the structure of the language;
	it uses modules and interfaces instead of functions [16].
The modules use and provide interfaces and are interconnected with configurations;
	this procedure makes up the structure of the program.
Multitasking is implemented in two ways:
	trough tasks and events.
Tasks,
	which focus on computation,
	are non-preemptive,
	and run until completion.
In contrast,
	events which focus on external events i.e.
interrupts,
	are preemptive,
	and have separate start and stop functions.
The OS has full support for both 6LoWPAN and RPL,
	and also have libraries for CoAP.


\subsubsection{freeRTOS}

One of the more popular and widely known operating systems is freeRTOS [17].
Written in C with only a few source files,
	it is a simple but powerful OS,
	easy to overview and extend.
It features two modes of scheduling,
	pre-emptive and co-operative,
	which may be selected according to the requirements of the application.
Two types of multitasking are featured:
	one is a lightweight Co-routine type,
	which has a shared stack for lower RAM usage and is thus aimed to be used on very small devices;
	the other is simply called Task,
	has its own stack and can therefore be fully pre-empted.
Tasks also support priorities which are used together with the pre-emptive scheduler.
The communication methods supported out-of-the-box are TCP and UDP.

\subsubsection{Summary and conclusion}


\begin{table}[h!]
\scriptsize
\begin{tabulary}{\textwidth}{L|L|L|L|L}
	\                                  & \bf{LiteOS}                & \bf{Nano-RK}                  & \bf{MANTIS}          & \bf{Contiki} \\\hline
	\bf{Architecture}                  & Monolithic                 & Layered                       & Modular              & Modular \\\hline
	\bf{Scheduling Memory}             & Round Robin                & Monotonic harmonized          & Priority classes     & Interrupts execute w.r.t. \\\hline
	\bf{Network}                       & File                       & Socket abstraction            & At Kernel COMM layer & uIP, Rime \\\hline
	\bf{Virtualization and Completion} & Synchronization primitives & Serialized access  semaphores & Semaphores           & Serialized, Access \\\hline
	\bf{Multi threading}               & \ok                        & \ok                           & \ko                  & \ok \\\hline
	\bf{Dynamic protection}            & \ok                        & \ko                           & \ok                  & \ok \\\hline
	\bf{Memory Stack}                  & \ok                        & \ko                           & \ko                  & \ko \\\hline
\end{tabulary}
\caption{\label{tab:OS} Common operating systems used in IoT environment \cite{al-fuqaha_internet_24}}
\end{table}



\subsection{Hardware platform}

\Figure{}{1}{riot-boards.png}{}

Even though the hardware is in one sense the tool that the OS uses to make IoT possible,
	it is still very important to select a platform that is future-proof and extensible.
To be regarded as an extensible platform,
	the hardware needs to have I/O connections that can be used by external peripherals.
Amongst the candidate interfaces are Serial Peripheral Interface (SPI),
	Inter-Integrated Circuit (I 2 C),
	and Controller Area Network (CAN).
These interfaces allow developers to attach custom-made PCBs with sensors for monitoring or actuators for controlling the environment.
The best practice is to implement an extension socket with a well-known form factor.
A future-proof device is specified as a device that will be as attractive in the future as it is today.
For hardware,
	this is very hard to achieve as there is constant development that follows Moore’s Law [4];
	however,
	the most important aspects are:
	the age of the chip,
	its expected remaining lifetime,
	and number of current implementations i.e.
its popularity.
If a device is widely used by consumers,
	the lifetime of the product is likely to be extended.
One last thing to take into consideration is the product family;
	if the chip belongs to a family with several members the transition to a newer chip is usually easier.

\subsubsection{OpenMote}

OpenMote is based on the Ti CC2538 System on Chip (SoC),
	which combines an ARM Cortex-M3 with a IEEE 802.15.4 transceiver in one chip [18, 19].
The board follows the XBee form factor for easier extensibility,
	which is used to connect the core board to either the OpenBattery or OpenBase extension boards [20, 21].
It originates from the CC2538DK which was used by Thingsquare to demo their Mist IoT solution [22].
Hence,
	the board has full support for Contiki,
	which is the foundation of Thingsquare.
It can run both as a battery-powered sensor board and as a border router,
	depending on what extension board it is attached to,
	e.g OpenBattery or OpenBase.
Furthermore,
	the board has limited support but ongoing development for RIOT and also full support for freeRTOS.

\subsubsection{MSB430-H}

The Modular Sensor Board 430-H from Freie Universität Berlin was designed for their ScatterWeb project [23].
As the university also hosts the RIOT project,
	the decision to support RIOT was natural.
The main board has a Ti MSP430F1612 MCU [24],
	a \textbf{Ti CC1100 transceiver},
	and a battery slot for dual AA batteries;
	it also includes a SHT11 temperature and humidity sensor and a MMA7260Q accelerometer to speed up early development.
All GPIO pins and buses are connected to external pins for extensibility.
Other modules with new peripherals can then be added by making a PCB that matches the external pin layout.

\subsubsection{Zolertia}

As many other Wireless Sensor Network (WSN) products,
	the Zolertia Z1 builds upon the MSP430 MCU [25, 26].
The communication is managed by the Ti CC2420 which operates in the 2.4 GHz band.
The platform includes two sensors:
	the SHT11 temperature and humidity sensor and the MMA7600Q accelerometer.
Extensibility is ensured with:
	two connections designed especially for external sensors,
	an external connector with USB,
	Universal asynchronous \textbf{receiver/transmitter (UART)},
	SPI,
	and I 2 C.


\section{IoT Communication protocols}


\subsection{Application}


\subsubsection{LwM2M}
\subsubsection{CBOR}
\subsubsection{DTLS}
\subsubsection{OSCOAP}

\subsubsection{CoAP}

\begin{itemize}
	\item Constrained Application Protocol
	\item The IETF Constrained RESTful Environments
	\item CoAP is bound to UDP
%	\item Enable devices with low resources to use RESTful interactions
	\item CoAP can be divided into two sub-layers
		\begin{itemize}
			\item messaging sub-layer
			\item request/response sub-layer
			\begin{itemize}
				\item[a)] Confirmable. 
				\item[b)] Non-confirmable. 
				\item[c)] Piggybacked responses. 
				\item[d)] Separate response
			\end{itemize}
		\end{itemize}
	\item CoAP, as in HTTP, uses methods such as:
	\begin{itemize}
		\item GET, PUT, POST and DELETE to 
		\item Achieve, Create, Retrieve, Update and Delete
	\end{itemize}
	\begin{itemize}
		\item Ex: the GET method can be used by a server to inquire the client’s temperature
	\end{itemize}
\end{itemize}

\begin{bytefield}[bitwidth=1em]{32}
	\bitheader{0-31}                                                     \\
	\begin{rightwordgroup}{CoAP Header}                                  \\
	\y{2}{Ver} & \y{2}{T} & \y{4}{TKL} & \y{8}{Code} & \y{16}{Message ID}\\
	\y{32}{Token}                                                        \\
	\y{32}{Options}                                                      \\
	\y{8}{11111111}  \y{24}{Payload}                                     \\
	\end{rightwordgroup}                                                 \\
\end{bytefield}
\begin{itemize}
	\item[\textbf{Ver:}] is the version of CoAP
	\item[\textbf{T:}] is the type of Transaction
	\item[\textbf{TKL:}] Token length
	\item[\textbf{Code:}] represents the request method (1-10) or response code (40-255).
		\begin{itemize}
			\item Ex: the code for GET, POST, PUT, and DELETE is 1, 2, 3, and 4, respectively.
		\end{itemize}
	\item[\textbf{Message ID:}] is a unique identifier for matching the response.
	\item[\textbf{Token:}] Optional response matching token.
\end{itemize}
	
\subsubsection{MQTT}
\begin{itemize}
	\item Message Queue Telemetry Transport
	\item Andy Stanford-Clark of IBM and Arlen Nipper of Arcom
		\begin{itemize}
			\item Standardized in 2013 at OASIS
		\end{itemize}
	\item MQTT uses the publish/subscribe pattern to provide transition flexibility and simplicity of implementation
	\item MQTT is built on top of the TCP protocol
	\item MQTT delivers messages through three levels of QoS
	\item Specifications
		\begin{itemize}
			\item MQTT v3.1 and MQTT-SN (MQTT-S or V1.2)
			\item MQTT v3.1 adds broker support for indexing topic names
		\end{itemize}
	\item The publisher acts as a generator of interesting data.
\end{itemize}

\begin{bytefield}[bitwidth=4em]{8}
	\bitheader{0-7}                                                     \\
	\begin{rightwordgroup}{CoAP Header}                                  \\
	\y{4}{Message Type} & \y{1}{UDP} & \y{2}{QoS Level} & \y{1}{Retain}  \\
	\y{8}{Remaining length}                                                        \\
	\y{8}{Variable length header}                                                      \\
	\y{8}{Variable length message payload}                                     \\
	\end{rightwordgroup}                                                 \\
\end{bytefield}
\begin{itemize}
\item[\textbf{Message type:}] CONNECT (1), CONNACK (2), PUBLISH (3), SUBSCRIBE (8) and so on
\item[\textbf{DUP flag:}] indicates that the massage is duplicated
\item[\textbf{QoS Level:}] identify the three levels of QoS for delivery assurance of Publish messages
\item[\textbf{Retain field:}] retain the last received Publish message and submit it to new subscribers as a first message
\end{itemize}

\subsubsection{XMPP}

\begin{itemize}
	\item Extensible Messaging and Presence Protocol
	\item Developed by the Jabber open source community
	\item An IETF instant messaging standard used for:
	\begin{itemize}
		\item multi-party chatting, voice and telepresence
	\end{itemize}
	\item Connects a client to a server using a XML stanzas
	\item An XML stanza is divided into 3 components:
	\begin{itemize}
		\item message: fills the subject and body fields
		\item presence: notifies customers of status updates
		\item iq (info/query): pairs message senders and receivers
	\end{itemize}
	\item Message stanzas identify:
	\begin{itemize}
		\item the source (from) and destination (to) addresses
		\item types, and IDs of XMPP entities
	\end{itemize}
\end{itemize}

\subsubsection{AMQP}

\begin{itemize}
	\item Advanced Message Queuing Protocol
	\item Communications are handled by two main components
	\begin{itemize}
		\item exchanges: route the messages to appropriate queues.
		\item message queues: Messages can be stored in message queues and then be sent to receivers
	\end{itemize}
	\item It also supports the publish/subscribe communications.
	\item It defines a layer of messaging on top of its transport layer.
	\item AMQP defines two types of messages
	\begin{itemize}
		\item bare massages: supplied by the sender
		\item annotated messages: seen at the receiver
	\end{itemize}
	\item The header in this format conveys the delivery parameters:
	\begin{itemize}
		\item durability, priority, time to live, first acquirer \& delivery count.
	\end{itemize}
	\item AMQP frame format
	\begin{itemize}
		\item[Size] the frame size.
		\item[DOFF] the position of the body inside the frame.
		\item[Type] the format and purpose of the frame.
		\begin{itemize}
			\item Ex: 0x00 show that the frame is an AMQP frame
			\item Ex: 0x01 represents a SASL frame.
		\end{itemize}
	\end{itemize}
\end{itemize}

\subsubsection{DDS}

\begin{itemize}
	\item Data Distribution Service
	\item Developed by Object Management Group (OMG)
	\item Supports 23 QoS policies:
	\begin{itemize}
		\item like security, urgency, priority, durability, reliability, etc
	\end{itemize}
	\item Relies on a broker-less architecture
	\begin{itemize}
		\item uses multicasting to bring excellent Quality of Service
		\item real-time constraints
	\end{itemize}
	\item DDS architecture defines two layers:
	\begin{itemize}
		\item[DLRL] Data-Local Reconstruction Layer
		\begin{itemize}
			\item serves as the interface to the DCPS functionalities
		\end{itemize}
		\item[DCPS] Data-Centric Publish/Subscribe
		\begin{itemize}
			\item delivering the information to the subscribers
		\end{itemize}
	\end{itemize}
	\item 5 entities are involved with the data flow in the DCPS layer:
	\begin{itemize}
		\item Publisher:disseminates data
		\item DataWriter: used by app to interact with the publisher
		\item Subscriber: receives published data and delivers them to app
		\item DataReader: employed by Subscriber to access received data
		\item Topic: relate DataWriters to DataReaders
	\end{itemize}
\end{itemize}
\begin{itemize}
	\item No need for manual reconfiguration or extra administration
	\item It is able to run without infrastructure
	\item It is able to continue working if failure happens.
	\item It inquires names by sending an IP multicast message to all the nodes in the local domain
	\begin{itemize}
		\item Clients asks devices that have the given name to reply back
		\item the target machine receives its name and multicasts its IP @
		\item Devices update their cache with the given name and IP @
	\end{itemize}
\end{itemize}

\subsubsection{mDNS}

\begin{itemize}
	\item Requires zero configuration aids to connect machine
	\item It uses mDNS to send DNS packets to specific multicast addresses through UDP
	\item There are two main steps to process Service Discovery:
	\begin{itemize}
		\item finding host names of required services such as printers
		\item pairing IP addresses with their host names using mDNS
	\end{itemize}
	\item Advantages
	\begin{itemize}
		\item IoT needs an architecture without dependency on a configuration mechanism
		\item smart devices can join the platform or leave it without affecting the behavior of the whole system
	\end{itemize}
	\item Drawbacks
	\begin{itemize}
		\item Need for caching DNS entries
	\end{itemize}
\end{itemize}


\begin{table}
\scriptsize
	\begin{tabulary}{\textwidth}{C|C|C|C|C|C|C|C}
		\textbf{Application protocol} & RestFull & Transport & Publish/Subscribe & Request/Response & Security & QoS & Header size (Byte)\\\hline
		\textbf{COAP}                 & \ok      & UDP       & \ok               & \ok              & DTLS     & \ok & 4           \\\hline
		\textbf{MQTT}                 & \ko      & TCP       & \ok               & \ko              & SSL      & \ok & 2           \\\hline
		\textbf{MQTT-SN}              & \ko      & TCP       & \ok               & \ko              & SSL      & \ok & 2           \\\hline
		\textbf{XMPP}                 & \ko      & TCP       & \ok               & \ok              & SSL      & \ko & -           \\\hline
		\textbf{AMQP}                 & \ko      & TCP       & \ok               & \ko              & SSL      & \ok & 8           \\\hline
		\textbf{DDS}                  & \ko      & UDP TCP   & \ok               & \ko              & SSL DTLS & \ok & -           \\\hline
		\textbf{HTTP}                 & \ok      & TCP       & \ko               & \ok              & SSL      & \ko & -           \\
	\end{tabulary}
	\caption{\label{tab:protocolsComparisoniu} Application protocols comparison}
\end{table}

\subsection{Network}
%\changefontsizes{pt}


\subsubsection{6TiSCH}
\subsubsection{OLSRv2}
\subsubsection{AODVv2}
\subsubsection{LoRaWAN}
\subsubsection{RPL}
\subsubsection{6LowPan}


% section protocols (end)
\begin{table}[h!]
\scriptsize
	\begin{tabulary}{\textwidth}{L|L|L|L|L}
		\bf{Routing protocol}  & \bf{Control Cost} & \bf{Link Cost} & \bf{Node Cost} \\\hline
		\bf{OSPF/IS-IS}        & \ko               & \ok            & \ko      \\
		\bf{OLSRv2}            & ?                 & \ok            & \ok      \\
	%		\bf{TBRPF}             & \ko               & \ok            & ?        \\
		\bf{RIP}               & \ok               & ?              & \ko      \\
	%		\bf{AODV}              & \ok               & \ko            & \ko      \\
	%		\bf{DYMO}              & \ok               & ?              & ?        \\
		\bf{DSR}               & \ok               & \ko            & \ko      \\
		\bf{RPL}               & \ok               & \ok            & \ok      \\\hline
	\end{tabulary}
	\caption{\label{tab:routingsComaprson} Routing protocols comparison \cite{_rpl2_}}
\end{table}


\begin{itemize}
	\item Routing over low-power and lossy links (ROLL)
	\item Support minimal routing requirements.
	\begin{itemize}
		\item like multipoint-to-point, point-to-multipoint and point-to-point.
	\end{itemize}
	\item A Destination Oriented Directed Acyclic Graph (DODAG)
	\begin{itemize}
		\item Directed acyclic graph with a single root.
		\item Each node is aware of ts parents 
		\item but not about related children
	\end{itemize}
	\item RPL uses four types of control messages
	\begin{itemize}
		\item DODAG Information Object (DIO)
		\item Destination Advertisement Object (DAO)
		\item DODAG Information Solicitation (DIS)
		\item DAO Acknowledgment (DAO-ACk)
	\end{itemize}
	%				\item RPL routers work under one of two modes:
	%					\begin{itemize}
	%						\item Non-Storing mode
	%						\item Storing modes mode
	%					\end{itemize}
\end{itemize}


\begin{itemize}
	\item Standard topologies to form IEEE 802.15.4e networks are 
	\begin{itemize}
		\item[Star] contains at least one FFD and some RFDs
		\item[Mesh] contains a PAN coordinator and other nodes communicate with each other
		\item[Cluster] consists of a PAN coordinator, a cluster head and normal nodes.
	\end{itemize}
	\item The IEEE 802.15.4e standard supports 2 types of network nodes
	\begin{itemize}
		\item[FFD] Full function device: serve as a coordinator
		\begin{itemize}
			\item It is responsible for creation, control and maintenance of the net
			\item It store a routing table in their memory and implement a full MAC
		\end{itemize}
		\item[RFD] Reduced function devices: simple nodes with restricted resources
		\begin{itemize}
			\item They can only communicate with a coordinator
			\item They are limited to a star topology
		\end{itemize}
	\end{itemize}
\end{itemize}
\begin{tabulary}{\textwidth}{|C|C|C|C|C|C|C|C|C|}\hline
	Preamble & PHDR & PHDRCRC & MHDR & FHDR & FPort & Payload & MIC & CRC \\\hline
\end{tabulary}



\begin{table}[h!]
\scriptsize
	\begin{tabulary}{\textwidth}{L|L|L|L|L}
		\bf{Routing protocol}  & \bf{Control Cost} & \bf{Link Cost} & \bf{Node Cost} \\\hline
		\bf{OSPF/IS-IS}        & \ko               & \ok            & \ko      \\
		\bf{OLSRv2}            & ?                 & \ok            & \ok      \\
	%		\bf{TBRPF}             & \ko               & \ok            & ?        \\
		\bf{RIP}               & \ok               & ?              & \ko      \\
	%		\bf{AODV}              & \ok               & \ko            & \ko      \\
	%		\bf{DYMO}              & \ok               & ?              & ?        \\
		\bf{DSR}               & \ok               & \ko            & \ko      \\
		\bf{RPL}               & \ok               & \ok            & \ok      \\\hline
	\end{tabulary}
	\caption{\label{tab:routingsComaprison} Routing protocols comparison \cite{_rpl2_}}
\end{table}


\subsection{LLC/MAC/Radio}

% \Figure{!htb}{.5}{riot-gnrc-stack.svg}
\Figure{!htb}{.5}{riot-network-stack.png}{LPWAN}

\Figure{!htb}{1}{iot-protocols-overview.png}{LPWAN}


\Figure{!htb}{1}{LPWAN.pdf}{LPWAN}


Several different wireless communication protocols,
	such as Wireless LAN (WLAN),
	BLE, 6LoWPAN,
	and ZigBee may be suitable for IoT applications.
They all operate in the 2.4GHz frequency band and this,
	together with the limited output power in this band,
	means that they all have a similar range.
The main differences are located in the MAC,
	PHY,
	and network layer.
WLAN is much too power hungry as seen in table 2.6 and is only listed as a reference for the comparisons.


\subsubsection{Divers}
\paragraph{IPLC}
\paragraph{BACnet}
\paragraph{Z-WAze}
\paragraph{Bluetooth LE}
BLE is developed to be backwards compatible with Bluetooth,
	but with lower data rate and power consumption [28].
Featuring a data rate of 1Mbit/s with a peak current consumption less than 15mA,
	it is a very efficient protocol for small amounts of data.
Each frame can be transmitted 47bytes in 1Mbit/s = 376Mus;
	thanks to the short transmission time,
	the transceivers consumes less power as the transceiver can be in receive mode or completely off most of the time.
BLE uses its own addressing methods and as the MAC frame size (figure 2.6) is only 39bytes,
	thus IPv6 addressing is not possible.

Starting from Bluetooth version 4.2,
	there is support for IPv6 addressing with the Internet Protocol Support Profile;
	the new version allows the BLE frame to be variable between 2 257 bytes.
The network set-up is controlled by the standard Bluetooth methods,
	whereas IPv6 addressing is handled by 6LoWPAN as specified in IPv6 over Bluetooth Low Energy [29].


\subsubsection{SigFox}

\subsubsection{IETF}
\paragraph{6LoWPAN} 
is a relatively new protocol that is maintained by the Internet
Engineering Task Force (IETF) [7, 6].
The purpose of the protocol is to enable IPv6 traffic over a IEEE 802.15.4 network with as low overhead as possible;
	this is achieved by compressing the IPv6 and UDP header.
A full size IPv6 + UDP header is 40+8 bytes which is tild 38\% of a IEEE 802.15.4
frame,
	but with the header compression this overhead can be reduced to 7 bytes,
	thus reducing the overhead to tild 5\%,
	as seen in figures 2.3 and 2.4.


\subsubsection{3GPP}
\paragraph{NB-IoT}
\paragraph{EC-GSM}
\paragraph{e-MTC}


\subsubsection{IEEE}
\paragraph{IEEE 802.11}

\paragraph{IEEE 802.15.4}
The IEEE 802.15.4 standard defines the PHY and MAC layers for wireless communication [6].
It is designed to use as little transmission time as possible but still have a decent payload,
	while consuming as little power as possible.
Each frame starts with a preamble and a start frame delimiter;
	it then continues with the MAC frame length and the MAC frame itself as seen in figure 2.2.
The overhead for each PHY packet is only 4+1+1 133 tild 4.5\%;
	when
using the maximum transmission speed of 250kbit/s,
	each frame can be sent 133byte in 250kbit/s = 4.265ms.
Furthermore,
	it can also operate in the 868MHz and 915MHz bands,
	maintaining the 250kbit/s transmission rate by using Offset quadrature phase-shift keying (O-QPSK).

Several network layer protocols are implemented on top of IEEE 802.15.4.
The two that will be examined are 6LoWPAN and ZigBEE.

\paragraph{ZigBee} 
is a communication standard initially developed for home automation
networks; it has several different protocols designed for specific areas such
as lighting, remote control, or health care [27, 6]. Each of these protocols
uses their own addressing with different overhead; however, there is also the
possibility of direct IPv6 addressing. Then, the overhead is the same as for
uncompressed 6LoWPAN, as seen in figure 2.5.

A new standard called ZigBee 3.0 aims to bring all these standards together under one roof to simplify the integration into IoT.
The release date of this standard is set to Q4 2015.

\subsubsection{LoaraWAN}
%Low Power Wide Area Networks
\Figure{h}{1}{lora_stack.png}{uhuhuh}

LoRa (Long Range) is a proprietary spread spectrum
modulation technique by Semtech. It is a derivative of Chirp
Spread Spectrum (CSS). The LoRa physical layer may be
used with any MAC layer; however, LoRaWAN is the cur-
rently proposed MAC which operates a network in a simple
star topology. 

As LoRa is capable to transmit over very long distances
it was decided that LoRaWAN only needs to support a star
topology. Nodes transmit directly to a gateway which is pow-
ered and connected to a backbone infrastructure. Gateways
are powerful devices with powerful radios capable to receive
and decode multiple concurrent transmissions (up to 50).
Three classes of node devices are defined: (1) Class A end-
devices: The node transmits to the gateway when needed.
After transmission the node opens a receive window to ob-
tain queued messages from the gateway. (2) Class B end-
devices with scheduled receive slots: The node behaves like
a Class A node with additional receive windows at sched-
uled times. Gateway beacons are used for time synchroni-
sation of end-devices. (3) Class C end-devices with maxi-
mal receive slots: these nodes are continuous listening which
makes them unsuitable for battery powered operations.

\clearpage

\begin{figure}
\begin{bytefield}[bitwidth=1.2em]{32}
	\bitheader{0-31}                                                  \\

	\begin{rightwordgroup}{MAC}
	\y{32}{Args}                                       \\
	\y{24}{FRM Payload (encrypted)} 	\y{8}{CID}         \\
	\y{8}{FPort}     & \y{24}{FRM Payload (encrypted)} \\
	\y{32}{FOpts: 0..120} \\
	\y{1}{1}     & \y{1}{2}             & \y{1}{3}               & \y{1}{4}       & \y{4}{FOptsLen}             & \y{16}{FCnt}               & \y{8}{FOpts: 0..120}    \\
	\y{32}{Dev Addr}
	\end{rightwordgroup}
\\
	\begin{rightwordgroup}{Phy}
	\y{32}{MIC}                                                       \\
	\y{32}{MAC Payload}                                               \\
	\y{3}{MType}   & \y{3}{RFU} & \y{2}{Major}  & \y{24}{MAC Payload} 
	\end{rightwordgroup}
%	FHDR \y{4}{Dev Addr: 32}     & \y{4}{FCtrl: 8}             & \y{4}{FCnt: 16}               & \y{4}{FOpts: 0..120}                                      \\
%	MHDR \y{4}{MType: 3}     & \y{4}{RFU: 3}             & \y{4}{Major: 2}                                  \\
%	\begin{rightwordgroup}{802.15.5 Header}                                                                                        \\
%		\y{1}{ADR}     & \y{1}{AdrAckReq}             & \y{1}{ACK}               & \y{1}{FPending}       & \y{4}{FOptsLen}                                 \\
%		\y{4}{ADR}     & \y{4}{MAC Payload}             & \y{4}{ACK}               & \y{4}{RFU}            & \y{4}{FOptsLen}                            \\
%	\end{rightwordgroup}
%	\y{16}{MACCommand_1: 8..40}        & \y{16}{MACCommand_n: 8..40}                                            \\
%	\y{8}{CID: 8}         & \y{32}{Args: 0..32}                                                                                             \\
\end{bytefield}
\caption{LoRaWAN frame format.\cite{augustin_study_2016}}\label{fig:jhjh}
\end{figure}


\Itemize{
	\item[DevAddr:] the short address of the device
	\item[FPort:] a multiplexing port field
	\Itemize{
		\item 0: the payload contains only MAC commands
	}
	\item[FOptsLen:]
	\item[FCnt:] frame counter
	\item[MIC:] is a cryptographic message integrity code
	\Itemize{
		\item computed over the fields MHDR, FHDR, FPort and the encrypted FRMPayload.
	}
	\item[MType:] is the message type (uplink or a downlink)
	\Itemize{
		\item whether or not it is a confirmed message (reqst ack)
	}
	\item[Major:] is the LoRaWAN version; currently, only a value of zero is valid
	\item[ADR/Ack] control the data rate adaptation mechanism by the network server
	\item[ACK] acknowledges the last received frame
	\item[B] indicates that the network server has additional data to send
	\item[FOptsLen] is the length of the FOpts field in bytes
	\item[FOpts] is used to piggyback MAC commands on a data message
	\item[CID] is the MAC command identifier
	\item[Args] are the optional arguments of the commands
	\item[FRMPL] is the payload, which is encrypted using AES with a key length of 128 bits
}

\clearpage
\paragraph{ALIANCE}


\subparagraph{Class-A}

\subsubparagraph{Uplink}
LoRa Server supports Class-A devices.
In Class-A a device is always in sleep mode,
	unless it has something to transmit.
Only after an uplink transmission by the device,
	LoRa Server is able to schedule a downlink transmission.
Received frames are de-duplicated (in case it has been received by multiple gateways),
	after which the mac-layer is handled by LoRa Server and the encrypted application-playload is forwarded to the application server.

\subsubparagraph{Downlink}
LoRa Server persists a downlink device-queue for to which the application-server can enqueue downlink payloads.
Once a receive window occurs,
	LoRa Server will transmit the first downlink payload to the device.

\subsubparagraph{Confirmed data}
LoRa Server sends an acknowledgement to the application-server as soon one is received from the device.
When the next uplink transmission does not contain an acknowledgement,
	a nACK is sent to the application-server.

\textbf{Note:}
	After a device (re)activation the device-queue is flushed.

\Figure{!htb}{.8}{a.png}{Class A}



\subparagraph{Class-B}
LoRa Server supports Class-B devices.
A Class-B device synchronizes its internal clock using Class-B beacons emitted by the gateway,
	this process is also called a “beacon lock”.
Once in the state of a beacon lock,
	the device negotioates its ping-interval.
LoRa Server is then able to schedule downlink transmissions on each occuring ping-interval.

\subsubparagraph{Downlink}
LoRa Server persists all downlink payloads in its device-queue.
When the device has acquired a beacon lock,
	it will schedule the payload for the next free ping-slot in the queue.
When adding payloads to the queue when a beacon lock has not yet been acquired,
	LoRa Server will update all device-queue to be scheduled on the next free ping-slot once the device has acquired the beacon lock.

\subsubparagraph{Confirmed data}
LoRa Server sends an acknowledgement to the application-server as soon one is received from the device.
Until the frame has timed out,
	LoRa Server will wait with the transmission of the next downlink Class-B payload.

\textbf{Note:} The timeout of a confirmed Class-B downlink can be configured through the device-profile.
This should be set to a value less than the interval between two ping-slots.

\subsubparagraph{Requirements}

\begin{itemize}
	\item[Device]
	The device must be able to operate in Class-B mode.
This feature has been tested against the develop branch of the Semtech LoRaMac-node source.
	\item[Gateway]
	The gateway must have a GNSS based time-source and must use at least the Semtech packet-forwarder version 4.0.1 or higher.
It also requires LoRa Gateway Bridge 2.2.0 or higher.

\end{itemize}

\Figure{!htb}{1}{b.png}{Class B}

\subparagraph{Class-C}

\subsubparagraph{Downlink}
LoRa Server supports Class-C devices and uses the same Class-A downlink device-queue for Class-C downlink transmissions.
The application-server can enqueue one or multiple downlink payloads and LoRa Server will transmit these (semi) immediately to the device,
	making sure no overlap exists in case of multiple Class-C transmissions.

\subsubparagraph{Confirmed data}
LoRa Server sends an acknowledgement to the application-server as soon one is received from the device.
Until the frame has timed out,
	LoRa Server will wait with the transmission of the next downlink Class-C payload.

\textbf{Note:} The timeout of a confirmed Class-C downlink can be configured through the device-profile.

\Figure{!htb}{1}{c.png}{Class C}

% \subsubsection{Summary and discussion}


\paragraph{SEMTECH}

\Figure{!htb}{1}{lorawan_parameters.png}{LoraWan Parameters}


LoRa has three configurable parameters:
\Itemize{
	\item[BW] \textbf{Bandwidth:}
	Bandwidth (BW) is the range of frequencies in the trans-
mission band. Higher BW gives a higher data rate (thus
shorter time on air), but a lower sensitivity (due to integration
of additional noise). A lower BW gives a higher sensitivity,
but a lower data rate. Lower BW also requires more accurate
crystals (less ppm). Data is send out at a chip rate equal to
the bandwidth. So, a bandwidth of 125 kHz corresponds to a
chip rate of 125 kcps. The SX1272 has three programmable
bandwidth settings: 500 kHz, 250 kHz and 125 kHz. The
Semtech SX1272 can be programmed in the range of 7.8 kHz
to 500 kHz, though bandwidths lower than 62.5 kHz requires
a temperature compensated crystal oscillator (TCXO).
	\item[CF] \textbf{Carrier Frequency:}
	Carrier Frequency (CF) is the centre frequency used for
the transmission band. For the SX1272 it is in the range
of 860 MHz to 1020 MHz, programmable in steps of 61 Hz.
The alternative radio chip Semtech SX1276 allows adjust-
ment from 137 MHz to 1020 MHz.
	\item[CR] \textbf{Coding Rate:}
	Coding Rate (CR) is the FEC rate used by the LoRa mo-
dem and offers protection against bursts of interference. A
higher CR offers more protection, but increases time on air.
Radios with different CR (and same CF/SF/BW), can still
communicate with each other. CR of the payload is stored in
the header of the packet, which is always encoded at 4/8.
	\item[SF] \textbf{Spreading Factor:}
	SF is the ratio between the symbol rate and chip rate.
A higher spreading factor increases the Signal to Noise Ra-
tio (SNR), and thus sensitivity and range, but also increases
the air time of the packet. The number of chips per symbol is
calculated as 2 sf . For example, with an SF of 12 (SF12) 4096
chips/symbol are used. Each increase in SF halves the trans-
mission rate and, hence, doubles transmission duration and
ultimately energy consumption. Spreading factor can be se-
lected from 6 to 12. SF6, with the highest rate transmission,
is a special case and requires special operations. For exam-
ple, implicit headers are required. Radio communications
with different SF are orthogonal to each other and network
separation using different SF is possible.
	\item[PL] \textbf{Path Loss:}
	\item[ToA] \textbf{Time on Air:}
}

Addition:
\Itemize{
	\item Payload (PL)
	\item Signal-to-noise ratio (SNR)
	\item Signal-to-Interference Ratio (SIR)
	\item Packet delivery ratio (PDR)
	\item Tw Power (Tx Power)
	\item Bit error rate (BER)
	\item Packet Reception Ratio (PRR)
}

signal-to-interference-plus-noise ratio (SINR)
signal-to-noise ratio (SNR)


\begin{align}
\textbf{T}_{\textbf{s}}     =& \frac{2^{\red{SF}}}{\green{BW}_{[Hz]}}\\
\ac{SR}_{\textbf{[sps]}}    =& \frac{\green{BW}}{2^{\red{SF}}}\\
\ac{DR}_{\textbf{[bps]}}    =& \red{SF}*\frac{\green{BW}_{[Hz]}}{2^{\red{SF}}}*\yellow{CR}\\
\ac{BR}_{\textbf{[bps]}}    =& \red{SF} * \frac{\frac{4}{4+\yellow{CR}}}{\frac{2 \red{SF}}{\green{BW}}}\\
\ac{Sen}_{\textbf{[dBm]}}   =& -174+10 \log _{10} \green{BW} + NF + SNR\\
\ac{SNR}_{\textbf{[dB]}}    =& 20.log(\frac{S}{N})\\
\ac{BER}_{\textbf{[bps]}}   =& \frac{8}{15}.\frac{1}{16}.\sum{k=2}{16}{-1^{k}(\frac{16}{k})e^{20.SINR(\frac{1}{k}-1)}}\\
\ac{BER}_{\textbf{[bps]}}   =& 10^{\alpha .e^{\ \beta \tiny\ac{SNR}}}\\
\ac{PER}_{\textbf{[pps]}}   =& 1-(1-BER)^{n_{bits}}\\
\ac{PRR}                    =& (1-\ac{BER})^{L}\\
\end{align}



\begin{align}
\ac{RSSI} = & Tx_{power}.\frac{Rayleigh_{power}}{PL}\\
\textbf{LoRa}=                                       & \frac{2^{S F}}{B W}\left((N P+4.25)+\left(S W+\max \left(\left\lceil\frac{8 P L-4 S F+28+16 C R C-20 I H}{4(S F-2 D E)}\right](C R+4), 0\right)\right)\right)\\
\textbf{Lora}=                                       & n_{s}=8+max([\frac{8PL-4SF+8+CRC+H}{4*(SF-DE)}]*\frac{4}{CR})\\
\textbf{Lora}=                                       & \frac{1}{R_{s}} \Bigg(n_{preamble} + \Bigg(SW + max\Bigg(\Bigg[\frac{8PL - 4SF + 28 + 16CRC -20IH}{4(SF-2DE)}\Bigg](CR+4),0\Bigg)\Bigg)\Bigg)\\
\textbf{GFSK}=                                       & \frac{8}{R_{GFSK}}\Bigg(L_{preamble} + SW + PL + 2CRC\Bigg)\\
\textbf{GFSK}=                                       & \frac{8}{D R}(N P+S W+P L+2 C R C)\\
\end{align}


\begin{table}[h!]
\scriptsize
	\begin{tabular}{|c|c?c|c|c|c|c|c?c|c|c|c|c|c?c|c|c|c|c|c?}
	\textbf{SF} &                      & 07 & 08 & 09 & 10 & 11 & 12 & 07 & 08 & 09 & 10 & 11 & 12 & 07 & 08 & 09 & 10 & 11 & 12\\\hline
	\           & \textbf{BW}          & \multicolumn{6}{c|}{125}     & \multicolumn{6}{c|}{250}     & \multicolumn{6}{c|}{500} \\\cline{1-20}
	07          & \multirow{6}{*}{125} & x  &    &    &    &    &    &    &    & x  &    &    &    &    &    &    &    & x  &   \\\cline{3-20}
	08          &                      &    & x  &    &    &    &    &    &    &    & x  &    &    &    &    &    &    &    & x \\\cline{3-20}
	09          &                      &    &    & x  &    &    &    &    &    &    &    & x  &    &    &    &    &    &    &   \\\cline{3-20}
	10          &                      &    &    &    & x  &    &    &    &    &    &    &    & x  &    &    &    &    &    &   \\\cline{3-20}
	11          &                      &    &    &    &    & x  &    &    &    &    &    &    &    &    &    &    &    &    &   \\\cline{3-20}
	12          &                      &    &    &    &    &    & x  &    &    &    &    &    &    &    &    &    &    &    &   \\\cmidrule[1pt]{1-20}

	07          & \multirow{6}{*}{250} &    &    &    &    &    &    & x  &    &    &    &    &    &    &    & x  &    &    &   \\\cline{3-20}
	08          &                      &    &    &    &    &    &    &    & x  &    &    &    &    &    &    &    & x  &    &   \\\cline{3-20}
	09          &                      & x  &    &    &    &    &    &    &    & x  &    &    &    &    &    &    &    & x  &   \\\cline{3-20}
	10          &                      &    & x  &    &    &    &    &    &    &    & x  &    &    &    &    &    &    &    & x \\\cline{3-20}
	11          &                      &    &    & x  &    &    &    &    &    &    &    & x  &    &    &    &    &    &    &   \\\cline{3-20}
	12          &                      &    &    &    & x  &    &    &    &    &    &    &    & x  &    &    &    &    &    &   \\\cmidrule[1pt]{1-20}

	07          & \multirow{6}{*}{500} &    &    &    &    &    &    &    &    &    &    &    &    & x  &    &    &    &    &   \\\cline{3-20}
	08          &                      &    &    &    &    &    &    &    &    &    &    &    &    &    & x  &    &    &    &   \\\cline{3-20}
	09          &                      &    &    &    &    &    &    & x  &    &    &    &    &    &    &    & x  &    &    &   \\\cline{3-20}
	10          &                      &    &    &    &    &    &    &    & x  &    &    &    &    &    &    &    & x  &    &   \\\cline{3-20}
	11          &                      & x  &    &    &    &    &    &    &    & x  &    &    &    &    &    &    &    & x  &   \\\cline{3-20}
	12          &                      &    & x  &    &    &    &    &    &    &    & x  &    &    &    &    &    &    &    & x \\\cmidrule[1pt]{1-20}
	\end{tabular}
\caption{\label{tab:uyuy} uyuyuy}
\end{table}


\begin{table}[h!]
\scriptsize
	\begin{tabulary}{\textwidth}{L|L|L|L|L|L}
	\              & LoRa[5]     & SigFox[6]         & NB-IoT [7] & Z-Wave[8] & Wi-Fi[9]   \\\hline
	Cost           & 3−5e        & 2−5e              & 10−20e     & 8−12e     & < 2e       \\\hline
	\ac{DR}      & <50 kbps    & <100 bps          & <200 kbps  & <40 kbps  & <300 Mbps  \\\hline
	Autonomy       & <10 years   & <10 years         & <10 years  & <2 years  & <10 days   \\\hline
	Range (urban)  & <5 km       & <10 km            & <1 km      & <100 m    & <40 m      \\\hline
	Modulation     & CSS         & BPSK              & QPSK       & FSK       & BPSK/QAM   \\\hline
	\ac{BW}      & 125/250 kHz & 100 Hz            & 200 kHz    & 300 kHz   & 20/40 MHz  \\\hline
	Frequency (EU) & 868 MHz     & 868 MHz           & LTE bands  & 868 MHz   & 2.4/5.0 GHz\\\hline
	Spectrum Cost  & Free        & Free              & Very High  & Free      & Free       \\\hline
	Max. msg/day   & Unlimited   & 140(↑), 4(↓)      & Unlimited  & Unlimited & Unlimited  \\\hline
	Max. payload   & 243 bytes   & 12(↑), 8(↓) bytes & 1600 bytes & 64 bytes  & 64 KB      \\\hline
	\end{tabulary}
\caption{\label{tab:terdjfy} Wireless technologies commonly used in smart buildings \cite{lopes_design_2019}}
\end{table}

\ac{BS}


%\changefontsizes{6pt}
\begin{table}[!ht]
\scriptsize
	\begin{tabulary}{\textwidth}{L|L|L|L|L|L|L}
	\bf{Characteristics}               & \bf{6LoWPAN}   & \bf{LoRaWAN}                    & \bf{SigFox}   & \bf{NB-IoT} & \textbf{INGENU} & \textbf{TELENSA}\\\hline
	\bf{Proprietary}                   &                &                                 & \ok           &             &                 &                 \\\hline
	\bf{Standar}                       & IETF           & LoRa Alliance                   &               & 3GPP        &                 &                 \\\hline
	\multirow{2}{*}{$\ac{CF}_{[MHz]}$} & 902-929        & 902-928                         & 902           &             &                 &                 \\
	\                                  & 868-868.6      & 863-870 and 434                 & 868           &             &                 &                 \\\hline
	\multirow{3}{*}{$\bf{Channels }$}  & 0016 for 2400  & 80             for 915          & 25            &             &                 &                 \\
	\                                  & 0010 for 915   & 10             for 868 and 780  &               &             &                 &                 \\
	\                                  & 0001 for 868.3 &                                 &               &             &                 &                 \\\hline
	\multirow{3}{*}{$\ac{BW}_{[MHz]}$} & 0005 for 2400  & 0.125 and 0.50 for 915          & 0.0001-0.0012 &             &                 &                 \\
	\                                  & 0002 for 915   & 0.125 and 0.25 for 868 and 780  &               &             &                 &                 \\
	\                                  & 0600 for 868.3 &                                 &               &             &                 &                 \\\hline
	\multirow{3}{*}{$\ac{DR}_{[kbps]}$}& 0250 for 2400  & 0.00098-0.0219 for 915          & 0.1-0.6       &             &                 &                 \\
	\                                  & 0040 for 915   & 0.250-0.05     for 868 and 780  &               &             &                 &                 \\
	\                                  & 0020 for 868.3 &                                 &               &             &                 &                 \\\hline
	\multirow{3}{*}{$\bf{Modulation}$} & QPSK for 2400  & LoRa           for 915          & BPSK and GFSK & QSPSK       &                 &                 \\
	\                                  & BPSK for 915   & LoRa and GFSK  for 868  and 780 &               &             &                 &                 \\
	\                                  & BPSK for 868.3 & \ac{CSS}                        &               &             &                 &                 \\
	\                                  &                & unslotted ALOHA                 &unslotted ALOHA&             & CDMA-like       &                 \\\hline
	\multirow{3}{*}{$\ac{CR}_{[dBm]}$} & -085 for 2400  & -137                            & -137          &             &                 &                 \\
	\                                  & -092 for 915   &                                 &               &             &                 &                 \\
	\                                  & -092 for 868.3 &                                 &               &             &                 &                 \\\hline
	Topology                           &                & Star, Stars                     & Star          &             & Star, Tree      & Star            \\\hline
	\ac{ADR}                           &                &       \ok                       & \ko           &             & \ok             & \ko             \\\hline
	\ac{PL}                            &                & <250B (depends on SF)           & 12B(UL),8B(DL)&             & 10KB            &                 \\\hline
	Handover                           &                & Multi \ac{BS}                   & Multi \ac{BS} &             &                 &                 \\\hline
	Security                           &                & AES 128b                        & \ko           &             & 16B hash, AES 256b&               \\\hline
	\ac{LS}                            &                & \ok                             & \ko           &             & \ko             & \ko             \\\hline
	\ac{FEC}                           &                & AES 128b                        & \ko           &             & \ok             & \ok             \\\hline
	\bf{Range}                         & 10-100 m       & 5-15 km                         & 10-50 km      & 1Km         &                 &                 \\\hline
	\bf{Battery lifetime}              & 1-2 years      & <10 years                       & <10 years     & <10 years   &                 &                 \\\hline
	\bf{Uplink}                        &                &                                 & 100bps        &             &                 &                 \\\hline
	\bf{Downlink}                      &                &                                 & 8 bytes/msg   &             &                 &                 \\\hline
	\bf{Cost}                          &                & 35e                             & 25e           & 1020e       &                 &                 \\\hline
	\bf{max msg/day}                   &                & Unlimited                       & 140(UL),4(DL) & Unlimited   &                 &                 \\\hline
	\bf{max Payload}                   &                & 243B                            & 12(UL),8(DL)  & 1600B       &                 &                 \\\hline

	\end{tabulary}
	\caption{\label{tab:LPWan_characteristics} LPWan Characteristics \cite{al-kashoash_comparison_2016}}
\end{table}

\begin{table}[h!]
\scriptsize
	\begin{tabulary}{\textwidth}{L|L|L|L|L}
	\                                            & \textbf{SIGFOX}                               & \textbf{LORAWAN}                                              & \textbf{INGENU}                                 & \textbf{TELENSA} \\\hline                                                              
	\textbf{Modulation}                          & UNB DBPSK( UL ), GFSK( DL )                   & CSS                                                           & RPMA-DSSS( UL ), CDMA( DL )                     & UNB 2-FSK \\\hline                                                             
	\textbf{Band}                                & S UB -GH Z ISM:EU (868MHz), US(902MHz)        & S UB -GH Z ISM:EU (433MHz 868MHz), US (915MHz), Asia (430MHz) & ISM 2.4GHz                                      & S UB -GH Z bands including ISM:EU (868MHz), US (915MHz), Asia (430MHz) \\\hline
	\textbf{\ac{DR}}                             & 100 bps( UL ), 600 bps( DL )                  & 0.3-37.5 kbps (L O Ra), 50 kbps (FSK)                         & 78kbps ( UL ), 19.5 kbps( DL )                  & 62.5 bps( UL ), 500 bps( DL ) \\\hline                                         
	\textbf{Range}                               & 10 km ( URBAN ), 50 km ( RURAL )              & 5 km( URBAN ), 15 km ( RURAL )                                & 15 km ( URBAN )                                 & 1 km ( URBAN ) \\\hline                                                        
	\textbf{Channels}                            & 360 channels                                  & 10 in EU, 64+8( UL ) and 8( DL ) in US plus multiple SFs      & 40 1MHz channels, up to 1200signals per channel & multiple channels \\\hline                                                     
	\end{tabulary}
\caption{\label{tab:edesf} \cite{raza_low_22}}
\end{table}


\begin{table}[h!]
\scriptsize
	\begin{tabulary}{\textwidth}{L|L|L|L|L|L|L}
	\textbf{Standard}                & \textbf{802.15.4k}                  & \textbf{802.15.4g}      & \textbf{Weightless-W}      & \textbf{Weightless-N}                   & \textbf{Weightless-P}      & \textbf{DASH 7 Alliance}\\\hline
	\textbf{Modulation}              & DSSS, FSK                           & MR-[FSK, OFDMA, OQPSK]  & 16-QAM, BPSK, QPSK, DBPSK  & UNB DBPSK                               & GMSK, offset-QPSK          & GFSK\\\hline
	\textbf{\ac{BW}}                    & ISM S UB -GH Z, 2.4GHz              & ISM S UB -GH Z, 2.4GHz  & TV white spaces 470-790MHz & ISM S UB -GH Z EU (868MHz), US (915MHz) & S UB -GH Z ISM or licensed & UB -GH Z 433MHz, 868MHz, 915MHz\\\hline
	\textbf{\ac{DR}}                 & 1.5 bps-128 kbps                    & 4.8 kbps-800 kbps       & 1 kbps-10 Mbps             & 30 kbps-100 kbps                        & 200 bps-100kbps            & 9.6,55.6,166.7 kbps\\\hline
	\textbf{Range}                   & 5 km ( URBAN )                      & up to several kms       & 5 km ( URBAN )             & 3 km ( URBAN )                          & 2 km ( URBAN )             & 0-5 km ( URBAN )\\\hline
	\textbf{MAC}                     & CSMA/CA, CSMA/CA or A LOHA with PCA & CSMA/CA                 & TDMA/FDMA                  & slotted A LOHA                          & TDMA/FDMA                  & CSMA/CA \\\hline
	\textbf{Topology}                & star                                & tar, mesh, peer-to-peer & star                       & star                                    & star                       & tree, star\\\hline
	\textbf{\ac{PL}}                 & 2047B                               & 2047B                   & >10B                       & 20B                                     & >10B                       & 256B \\\hline
	\textbf{Security}                & AES 128b                            & AES 128b                & AES 128b                   & AES 128b                                & AES 128/256b               & AES 128b \\\hline
	\textbf{Forward error correction}& \ok                                 & \ok                     & \ok                        & \ko                                     & \ok                        & \ok\\\hline
	\end{tabulary}
\caption{\label{tab:uyuy} \cite{raza_low_22}}
\end{table}

\begin{table}[h!]
\scriptsize
	\begin{tabulary}{\textwidth}{L|L|L|L|L|L|L}
	\bf{Phy protocol}     & \bf{IEEE 802.15.4} & \bf{BLE}      & \bf{EPCglobal} & \bf{Z-Wave}              & \bf{LTE-M}           & \bf{ZigBee} \\\hline
	\bf{Standard}    &                    & IEEE 802.15.1 &                &                          &                      & IEEE 802.15.4, ZigBee Alliance \\\hline
	\bf{\ac{BW}(MHz)} & 868/915/2400       & 2400          & 860-960        & 868/908/2400             & 700-900              & \\\hline
	\bf{MAC}      & TDMA, CSMA/CA      & TDMA          & ALOHA          & CSMA/CA                  & OFDMA                & \\\hline
	\bf{\ac{DR} (bps)}  & 20/40/250 K        & 1024K         & varies 5-640K  & 40K                      & 1G (up), 500M (down) & \\\hline
	\bf{Throughput}       &                    &               &                & 9.6, 40, 200kbps         &                      & \\\hline
	\bf{Scalability}  & 65K nodes          & 5917 slaves   & -              & 232 nodes                & -                    & \\\hline
	\bf{Range}            & 10-20m             & 10-100m       &                &                          &                      & \\\hline
	\bf{Addressing}       & 8|16bit            & 16bit         &                &                          &                      & \\\hline
	\end{tabulary}
	\caption{\label{tab:IoT_cloud} IoT cloud platforms and their characteristics \cite{al-fuqaha_internet_24}}
\end{table}


\begin{table}[h!]
\scriptsize
	\begin{tabulary}{\textwidth}{|p{6em}|L|L|L|L|}
	\                           & \textbf{802.15.4}                   & \textbf{802.15.4e}                       & \textbf{802.15.4g}                & \textbf{802.15.4f}                      \\\hline
	\textbf{\ac{CF}}          & 2.4Ghz (DSSS + oQPSK)               & 2.4Ghz (DSSS + oQPSK, CSS+DQPSK )        & 2.4Ghz (DSSS + oQPSK, CSS+DQPSK ) & 2.4Ghz (DSSS + oQPSK,CSS+DQPSK )        \\
	\                           & 868Mhz (DSSS + BPSK)                & 868Mhz (DSSS + BPSK)                     & 868Mhz (DSSS + BPSK)              & 868Mhz (DSSS + BPSK)                    \\
	\                           & 915Mhz (DSSS + BPSK)                & 915Mhz (DSSS + BPSK)                     & 915Mhz (DSSS + BPSK)              & 915Mhz (DSSS + BPSK) 3\sim 10Ghz (BPM+BPSK )\\
	\textbf{\ac{DR}}          & Upto 250kbps                        & Upto 800kbps                             & Up to 800kbps                     &                                         \\
	\textbf{Differences}        & -                                   & Time sync and channel hopping            & Phy Enhancements                  & Mac and Phy Enhancements                \\
	\textbf{\ac{PL}}         & 127 bytes                           & N/A                                      & Up to 2047 bytes                  & N/A                                     \\
	\textbf{Range}              & 1 – 75+ m                           & 1 – 75+ m                                & Upto 1km                          & N/A                                     \\
	\textbf{Goals}              & General Low-power Sensing/Actuating & Industrial segments                      & Smart utilities                   & Active RFID                             \\
	\textbf{Products}           & Many                                & Few                                      & Connode (6LoWPAN)                 & LeanTegra PowerMote                     \\\hline
	\end{tabulary}
\caption{\label{tab:IEEE_802.15.4_standards} IEEE 802.15.4 standards \cite{sarwar_iot_2015}}
\end{table}

\begin{table}[h!]
\scriptsize
	\begin{tabulary}{\textwidth}{L|L|L|L|L}
	\textbf{SF} & Sensitivity[dBm] & \ac{DR}[kb/s] &  & \\\hline
	6           &    -118              &     9.38            &  & \\\hline
	7           &    -123              &                 &  & \\\hline
	8           &    -126              &                 &  & \\\hline
	9           &    -129              &                 &  & \\\hline
	10          &    -132              &                 &  & \\\hline
	11          &    -134.5              &                 &  & \\\hline
	12          &    -137              &                 &  & \\\hline
	\end{tabulary}
\caption{\label{tab:1} hghg}
\end{table}

\begin{table}[h!]
\scriptsize
	\begin{tabular}{l|lll|lll|lll|l}
	\textbf{\ac{SF}/\ac{BW}} & \multicolumn{3}{l}{\textbf{125kHz}}                   & \multicolumn{3}{l}{\textbf{250kHz} }             & \multicolumn{3}{l}{\textbf{500kHz} }             & \\\hline
	\textbf{-}     & Sensitivity [dBm]                  & Bit Rate [kb/s]&   & Sensitivity                        & Data Rate &   & Sensitivity                        & Data Rate &   & \\
	\textbf{6}     & -118                               &                &   & -115                               &           &   & -111                               &           &   & \\
	\textbf{7}     & -123                               & 5.468          &   & -120                               &           &   & -116                               &           &   & \\
	\textbf{8}     & -126                               & 3.125          &   & -123                               &           &   & -119                               &           &   & \\
	\textbf{9}     & -129                               & 1.757          &   & -125                               &           &   & -122                               &           &   & \\
	\textbf{10}    & -132                               & 0.976          &   & -128                               &           &   & -125                               &           &   & \\
	\textbf{11}    & -133                               & 0.537          &   & -130                               &           &   & -128                               &           &   & \\
	\textbf{12}    & -136                               & 0.293          &   & -133                               &           &   & -130                               &           &   & \\\hline
	\end{tabular}
\caption{\label{tab:EE} Receiver sensitivity [dBm]}
\end{table}




\cite{_evaluation_} Nous avons vu en effet plus haut qu’il a été démontré que la méthode CSMA est plus efficace pour le traitement des faibles trafics,
	tandis que TDMA est nettement plus appropriée pour supporter les trafics intensesj.




\begin{table}[h!]
\scriptsize
	\begin{tabular}{l|l|l|l|l|l|l}
	\textbf{\ac{DR}}  & \multicolumn{3}{l}{\textbf{Modulation}} &                                 \multicolumn{2}{l}{\textbf{Max transmission unit}}                          & \textbf{\ac{BR}}  \\\hline
	\                 & \textbf{\ac{SF}}                          & \textbf{\ac{BW} [kHz]} & \textbf{\ac{CR}} & \textbf{Total [B]}                                  & \textbf{Payload [B]} & x kbit/s \\\hline
	0                 & 12                                        & 125               & 4/6         & 64                                                  & 51                   & 0.25     \\\hline
	1                 & 11                                        & 125               & 4/6         & 64                                                  & 51                   & 0.44     \\\hline
	2                 & 10                                        & 125               & 4/5         & 64                                                  & 51                   & 0.98     \\\hline
	3                 & 9                                         & 125               & 4/5         & 128                                                 & 115                  & 1.76     \\\hline
	4                 & 8                                         & 125               & 4/5         & 255                                                 & 242                  & 3.125    \\\hline
	5                 & 7                                         & 125               & 4/5         & 255                                                 & 242                  & 5.47     \\\hline
	6                 & 7                                         & 125               & 4/5         & 255                                                 & 242                  & 11       \\\hline
	7                 &                                           & 125               & 4/5         & 255                                                 & 242                  & 50       \\\hline
	\end{tabular}
\caption{\label{tab:llo}oioioi }
\end{table}

%\section{Lora modules}
%\begin{table}[h!]
%\scriptsize
%	\begin{tabulary}{\textwidth}{L|L|L|C|C|C|C|C}
%	Ref                                & \textbf{Module}                                 & \textbf{Frequency       MHz} & \textbf{Tx power}       & \textbf{Rx power}     & \textbf{Sensitivity}      & \textbf{Channels} & \textbf{Distance}      \\\hline
%	\multirow{2}{*}{\cite{_waspmote_}} & \multirow{2}{*}{\textbf{Semtech SX1272}}        & 863-870 (EU)                 & \multirow{2}{*}{14 dBm} & \multirow{2}{*}{ dBm} & \multirow{2}{*}{-134 dBm} & 8                 & \multirow{2}{*}{22+ km}\\
%	\                                &                                                   & 902-928 (US)                 &                         &                       &                           & 13                & \\\hline
%	\multirow{2}{*}{\cite{_waspmote_}} & \multirow{2}{*}{\textbf{rn2483}}                &                              & \multirow{2}{*}{}       & \multirow{2}{*}{}     & \multirow{2}{*}{}         &                   & \multirow{2}{*}{}\\
%	\                                  &                                                 &                              &                         &                       &                           &                   & \\\hline
%	\multirow{2}{*}{\cite{_waspmote_}} & \multirow{2}{*}{\textbf{rn2903}}                &                              & \multirow{2}{*}{}       & \multirow{2}{*}{}     & \multirow{2}{*}{}         &                   & \multirow{2}{*}{}\\
%	\                                  &                                                 &                              &                         &                       &                           &                   & \\\hline
%	\multirow{2}{*}{\cite{_waspmote_}} & \multirow{2}{*}{\textbf{rak811}}                &                              & \multirow{2}{*}{}       & \multirow{2}{*}{}     & \multirow{2}{*}{}         &                   & \multirow{2}{*}{}\\
%	\                                  &                                                 &                              &                         &                       &                           &                   & \\\hline
%	\multirow{2}{*}{\cite{_waspmote_}} & \multirow{2}{*}{\textbf{Semtech sx1276}}        &                              & \multirow{2}{*}{}       & \multirow{2}{*}{}     & \multirow{2}{*}{}         &                   & \multirow{2}{*}{}\\
%	\                                  &                                                 &                              &                         &                       &                           &                   & \\\hline
%	\multirow{2}{*}{\cite{_waspmote_}} & \multirow{2}{*}{\textbf{rfm95}}                 &                              & \multirow{2}{*}{}       & \multirow{2}{*}{}     & \multirow{2}{*}{}         &                   & \multirow{2}{*}{}\\
%	\                                  &                                                 &                              &                         &                       &                           &                   & \\\hline
%	\multirow{2}{*}{\cite{_waspmote_}} & \multirow{2}{*}{\textbf{CMWX1ZZABZ-078}}        &                              & \multirow{2}{*}{}       & \multirow{2}{*}{}     & \multirow{2}{*}{}         &                   & \multirow{2}{*}{}\\
%	\                                  &                                                 &                              &                         &                       &                           &                   & \\\hline
%	\multirow{2}{*}{\cite{_waspmote_}} & \multirow{2}{*}{\textbf{LoPy4}}                 &                              & \multirow{2}{*}{}       & \multirow{2}{*}{}     & \multirow{2}{*}{}         &                   & \multirow{2}{*}{}\\
%	\                                  &                                                 &                              &                         &                       &                           &                   & \\\hline
%	\multirow{2}{*}{\cite{_waspmote_}} & \multirow{2}{*}{\textbf{mDot}}                  &                              & \multirow{2}{*}{}       & \multirow{2}{*}{}     & \multirow{2}{*}{}         &                   & \multirow{2}{*}{}\\
%	\                                  &                                                 &                              &                         &                       &                           &                   & \\\hline
%	\multirow{2}{*}{\cite{_waspmote_}} & \multirow{2}{*}{\textbf{xDot}}                  &                              & \multirow{2}{*}{}       & \multirow{2}{*}{}     & \multirow{2}{*}{}         &                   & \multirow{2}{*}{}\\
%	\                                  &                                                 &                              &                         &                       &                           &                   & \\\hline
%	\multirow{2}{*}{\cite{_waspmote_}} & \multirow{2}{*}{\textbf{Laird RM192}}           &                              & \multirow{2}{*}{}       & \multirow{2}{*}{}     & \multirow{2}{*}{}         &                   & \multirow{2}{*}{}\\
%	\                                  &                                                 &                              &                         &                       &                           &                   & \\\hline
%	\multirow{2}{*}{\cite{_waspmote_}} & \multirow{2}{*}{\textbf{Laird RM186}}           &                              & \multirow{2}{*}{}       & \multirow{2}{*}{}     & \multirow{2}{*}{}         &                   & \multirow{2}{*}{}\\
%	\                                  &                                                 &                              &                         &                       &                           &                   & \\\hline
%	\multirow{2}{*}{\cite{_waspmote_}} & \multirow{2}{*}{\textbf{CMWX1ZZABZ-078}}        &                              & \multirow{2}{*}{}       & \multirow{2}{*}{}     & \multirow{2}{*}{}         &                   & \multirow{2}{*}{}\\
%	\                                  &                                                 &                              &                         &                       &                           &                   & \\\hline
%	\multirow{2}{*}{\cite{_waspmote_}} & \multirow{2}{*}{\textbf{Also Laird RM1xx}}      &                              & \multirow{2}{*}{}       & \multirow{2}{*}{}     & \multirow{2}{*}{}         &                   & \multirow{2}{*}{}\\
%	\                                  &                                                 &                              &                         &                       &                           &                   & \\\hline
%	\multirow{2}{*}{\cite{_waspmote_}} & \multirow{2}{*}{\textbf{iMST iM88x/iM98x}}      &                              & \multirow{2}{*}{}       & \multirow{2}{*}{}     & \multirow{2}{*}{}         &                   & \multirow{2}{*}{}\\
%	\                                  &                                                 &                              &                         &                       &                           &                   & \\\hline
%	\multirow{2}{*}{\cite{_waspmote_}} & \multirow{2}{*}{\textbf{Mic SAM RN34/35}}       &                              & \multirow{2}{*}{}       & \multirow{2}{*}{}     & \multirow{2}{*}{}         &                   & \multirow{2}{*}{}\\
%	\                                  &                                                 &                              &                         &                       &                           &                   & \\\hline
%	\multirow{2}{*}{\cite{_waspmote_}} & \multirow{2}{*}{\textbf{Semtech SX1278}}        &                              & \multirow{2}{*}{}       & \multirow{2}{*}{}     & \multirow{2}{*}{}         &                   & \multirow{2}{*}{}\\
%	\                                  &                                                 &                              &                         &                       &                           &                   & \\\hline
%	\end{tabulary}
%\caption{\label{tab:sx1276} }
%\end{table}

% \changefontsizes{7pt}
\begin{table}[h!]
\scriptsize
	\begin{tabulary}{\textwidth}{L|L|L|L|L|L}
	\bf{Feature}                     & \bf{Wi-Fi}                                            & \bf{802.11p}                                          & \bf{UMTS}                        & \textbf{LTE}                      & \textbf{LTE-A}  \\\hline
	\bf{Channel MHz}           & 20                                                    & 10                                                    & 5                                & 1.4, 3, 5, 10, 15, 20    & <100                 \\\hline
	\bf{Frequency band(s) GHz}       & 2.4 , 5.2                                             & 5.86-5.92                                             & 0.7-2.6                          & 0.7-2.69                 & 0.45-4.99                 \\\hline
	\bf{\ac{BR} Mb/s}               & 6-54                                                  & 3–27                                                  & 2                                & <300                & <1000                \\\hline
	\bf{Range km}                    & <0.1                                                  & <1                                               & <10                         & <30                 & <30                  \\\hline
	\bf{Capacity}                    & Medium                                                & Medium                                                & \ko                              & \ok                      & \ok                       \\\hline
	\bf{Coverage}                    & Intermittent                                          & Intermittent                                          & Ubiquitous                       & Ubiquitous               & Ubiquitous                \\\hline
	\bf{Mobility support km/h}       & \ko                                                   & Medium                                                & \ok                              & <350                & <350                 \\\hline
	\bf{QoS support}                 & EDCA \scriptsize{Enhanced Distributed Channel Access} & EDCA \scriptsize{Enhanced Distributed Channel Access} & QoS classes and bearer selection & QCI and bearer selection & QCI and bearer selection  \\\hline
	\bf{Broadcast/multicast support} & Native broadcast                                      & Native broadcast                                      & Through MBMS                     & Through eMBMS            & Through eMBMS             \\\hline
	\bf{V2I support}                 & \ok                                                   & \ok                                                   & \ok                              & \ok                      & \ok                       \\\hline
	\bf{V2V support}                 & Native (ad hoc)                                       & Native (ad hoc)                                       & \ko                              & \ko                      & Through D2D               \\\hline
	\bf{Market penetration}          & \ok                                                   & \ko                                                   & \ok                              & \ok                      & \ok                       \\\hline
	\bf{\ac{DR}}                   & <640 kbps                                        & 250 kbps                                              & 106–424 kbps                              & \ok                      & \ok                       \\\hline
	\end{tabulary}
	\caption{\label{tab:Tableppp} An example table.}
\end{table}




\subsection{MAC}

\begin{table}[h!]
\scriptsize
	\begin{tabulary}{\textwidth}{L|L|L|L|L}
	Channel based & FDMA                          & OFDMA WDMA SC-FDMA                              &  & \\\hline
	\             & TDMA                          & MF-TDMA STDMA                                   &  & \\\hline
	\             & CDMA                          & W-CDMA TD-CDMA TD-SCDMA DS-CDMA FH-CDMA MC-CDMA &  & \\\hline
	\             & SDMA                          & HC-SDMA                                         &  & \\\hline                                               &  & \\\hline
	Packet-based  & Collision recovery            & ALOHA Slotted ALOHA R-ALOHA AX.25 CSMA/CD       &  & \\\hline
	\             & Collision avoidance           & MACA MACAW CSMA CSMA/CA DCF PCF HCF CSMA/CARP   &  & \\\hline
	\             & Collision-free                & Token ring Token bus MS-ALOHA                   &  & \\\hline
	Duplexing methods  & Delay and disruption tolerant & MANET VANET DTN Dynamic Source Routing          &  & \\\hline

	\end{tabulary}
\caption{\label{tab:} }
\end{table}

\subsubsection{Sharing the channel}
\paragraph{TDMA}
\paragraph{FDMA}
\paragraph{CDMA}
\paragraph{TSMA}


\subsubsection{Transmitting information}
\paragraph{TFDM}
\paragraph{TDSSS}
\paragraph{TFHSS}


\subsection{Radio}

\subsubsection{Analog modulation}

\paragraph{AM}
\paragraph{FM}
\paragraph{PM}
\paragraph{QAM}
\paragraph{SM}
\paragraph{SSB}
         

\subsubsection{Digital modulation}

\paragraph{ASK}
\paragraph{APSK} 
\paragraph{CPM} 
\paragraph{FSK} 
\paragraph{MFSK} 
\paragraph{MSK} 
\paragraph{OOK} 
\paragraph{PPM} 
\paragraph{PSK} 
\paragraph{QAM} 
\paragraph{SC-FDE} 
\paragraph{TCM WDM}

\subsubsection{Hierarchical modulation}

\paragraph{QAM} 
\paragraph{WDM}

\subsubsection{Spread spectrum}

\paragraph{SS}
\paragraph{DSSS} 
\paragraph{FHSS} 
\paragraph{THSS}


\subsection{Security}

\subsubsection{Blockchain}

Blockchain Layers
\begin{itemize}
	\item Transaction \& contract layer
	\item Validation layer (forward validation request)
	\item Block Generation Layer (PoW,PoC, PoA PoS, PBFT)
	\item Distribution Layer
	\\
\end{itemize}

Consensus algorithms
\begin{itemize}
	\item Proof of Work (PoW)
	\item Proof of Capacity (PoC)
	\item Proof of Authority (PoA)
	\item Proof of Stake (PoS)
	\item Proof of Bizantine Fault Tolerant (PBFT)
\end{itemize}


\subsubsection{Summary and discussion}











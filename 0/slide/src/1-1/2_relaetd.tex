\subsection{Heterogeneity}

\begin{frame}{Literature review}{Related work}
\begin{itemize}
	\item[\cite{qin_software_2014}] Many studies have identified \green{SDN} as a potential solution to the WSN challenges,
	as well as a model for \red{heterogeneous} integration.
	
	\item[\cite{qin_software_2014}] This \red{shortfall} can be resolved by using the \green{SDN approach.}
	
	\item[\cite{kobo_survey_2017}] \green{SDN} also enhances better control of \red{heterogeneous} network infrastructures.
	
	\item[\cite{kobo_survey_2017}] Anadiotis et al. define a \green{SDN operating system for IoT} that integrates SDN based WSN \textbf{(SDN-WISE)}.
		This experiment shows how \red{heterogeneity} between different kinds of SDN networks can be achieved.
		
	\item[\cite{kobo_survey_2017}] In cellular networks,
		OpenRoads presents an approach of introducing \green{SDN} based \red{heterogeneity} in wireless networks for operators.

	\item[\cite{ndiaye_software_2017}] There has been a plethora of (industrial) studies \green{synergising SDN in IoT}.
		The major characteristics of IoT are low latency,wireless access, mobility and \red{heterogeneity}.
	
	\item[\cite{ndiaye_software_2017}] Thus a bottom-up approach application of \green{SDN} to the realisation of \red{heterogeneous IoT} is suggested.

	\item[\cite{ndiaye_software_2017}] Perhaps a more complete IoT architecture is proposed,
		where the authors apply \green{SDN} principles in IoT \red{heterogeneous} networks.
		
	\item[\cite{bera_softwaredefined_2017}] it provides the \green{SDWSN} with a proper model of network management,
		especially considering the potential of \red{heterogeneity} in SDWSN.
		
	\item[\cite{bera_softwaredefined_2017}] We conjecture that the \green{SDN paradigm} is a good candidate to solve the \red{heterogeneity} in IoT.
	
\end{itemize}

\end{frame}

\begin{frame}{SDN}{Related work}
%	\begin{columns}
%		\begin{column}{0.5\textwidth}
			\begin{center}
			
				\begin{figure}
					\includegraphics[width=.7\columnwidth]{res/sdn2.png}
					\caption{\label{fig:sdn2} Architecture and technology abstraction.}
				\end{figure}
				
			\end{center}
%		\end{column}
%		
%		\begin{column}{0.5\textwidth}
%		
%			\begin{itemize}
%				\item 
%				
%				\item 
%					\begin{itemize}
%						\item 
%						\item 
%					\end{itemize}
%				
%				\item 
%			\end{itemize}
%			
%		\end{column}
%	\end{columns}
	

\end{frame}

\subsection{Security}

\begin{frame}{Security in SDN}{related work}

	\begin{figure}
		\includegraphics[width=\columnwidth]{sdn-osi.png}
		\caption*{Table: \label{fig:sdn-osi}SDN vs OSI layer}
	\end{figure}

\end{frame}

\begin{frame}{SDN based sensor network}{}
\begin{table}[h!]
\changefontsizes{5pt}
\begin{center}
	\begin{tabulary}{\columnwidth}{L|L|C|C|C|C|C}
	\textbf{Management architecture}                 & \textbf{Management feature}            & \textbf{Controller configuration} & \textbf{Traffic Control} & \textbf{Configuration and monitoring} & \textbf{Scapability and localization} & \textbf{Communication management}\\\hline
	\textbf{\cite{luo_sensor_2012} Sensor Open Flow} & SDN support protocol                   & Distributed                       & in/out-band              & \ok                                   & \ok                                   & \ok                              \\\hline
	\textbf{\cite{costanzo_software_} SDWN}          & Duty sycling, aggregation, routing     & Centralized                       & in-band                  & \ok                                   &                                       & \\\hline
	\textbf{\cite{galluccio_sdnwise_2015} SDN-WISE}  & Programming simplicity and aggregation & Distributed                       & in-band                  &                                       & \ok                                   & \\\hline
	\textbf{\cite{degante_smart_2014a} Smart}        & Efficiency in resource allocation      & Distributed                       & in-band                  &                                       & \ok                                   & \\\hline
	\textbf{SDCSN}                                   & Network reliability and QoS            & Distributed                       & in-band                  &                                       & \ok                                   & \\\hline
	\textbf{TinySDN}                                 & In-band-traffic control                & Distributed                       & in-band                  &                                       & \ok                                   & \\\hline
	\textbf{Virtual Overlay}                         & Network flexibility                    & Distributed                       & in-band                  &                                       & \ok                                   & \\\hline
	\textbf{Context based}                           & Network scalability and performance    & Distributed                       & in-band                  &                                       & \ok                                   & \\\hline
	\textbf{CRLB}                                    & Node localization                      & Centralized                       & in-band                  &                                       &                                       & \\\hline
	\textbf{Multi-hope}                              & Traffic and energy control             & Centralized                       & in-band                  &                                       &                                       & \ok                              \\\hline
	\textbf{Tiny-SDN}                                & Network task measurement               & -                                 & in-band                  &                                       &                                       & \\
	\end{tabulary}
	\caption{\label{tab:Table} SDN-based network and topology management architectures. \cite{ndiaye_software_2017}}
\end{center}
\end{table}

\end{frame}

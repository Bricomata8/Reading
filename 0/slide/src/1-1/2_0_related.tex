\begin{frame}{State of the art}{Standardization}
	\begin{table}
		\begin{tabulary}{\textwidth}{L|C|C|C|C|C|C|C}
			\bf{Application protocol}                                                 & DDS                                     & CoAP                              & AMQP                              & MQTT                                & MQTT-SN & XMPP & HTTP\\\hline
			\bf{Service discovery}     & \multicolumn{3}{c}{mDNS}                & \multicolumn{4}{c}{DNS-SD}                                                                                                         \\
			\bf{Network layer}         & \multicolumn{7}{c}{RPL}                                                                                                                                                      \\
%			\bf{Network layer}         & \multicolumn{3}{c}{6LoWPAN}             & \multicolumn{4}{c}{IPv4/IPv6}                                                                                                      \\
			\bf{Link layer}            & \multicolumn{7}{c}{IEEE 802.15.4}                                                                                                                                            \\
			\bf{Physical layer}        & \multicolumn{3}{c}{EPCglobal}     & \multicolumn{2}{c}{IEEE 802.15.4} & \multicolumn{2}{c}{Z-Wave}                                 \\\hline
		\end{tabulary}
		\caption{\label{tab:Tablej} Standardization efforts that support the IoT}
	\end{table}
	
	\note{
		\begin{itemize}
			\item Contenu:
			\begin{itemize}
				\item Tableau comparatif (articles connexes/avantages et désavantages)
				\item Les limites de l’existant
				\item Notre travaille traite le meme x que les travaux précidants mais utilise y au lieu de z (xy/xz)
			\end{itemize}
			\item Procedure:
			\begin{itemize}
				\item Lecture en largeur
				\begin{itemize}
					\item Lecture de beaucoup de papiers connexes
					\item Comprendre le domaine
					\item Comprendre les travaux existants
					\item Sélection des travaux intéressants
				\end{itemize}
				\item Lecture en profondeur
				\begin{itemize}
					\item Lecture et analyse des travaux sélectionnés
					\item Descendre jusqu’au détail du détail
					\begin{itemize}
						\item Poser toujours la question pourquoi?
						\item Être capable d’implémenter de suite l’approche.
					\end{itemize}
				\end{itemize}
				\item Situer le travail par rapport à l’existant sur la base de La problématique traitée
				\begin{itemize}
					\item Les critiques faites sur l’existant
					\item Les hypothèses du travail courant
					\item Les objectifs initiales du travail
					\item Les résultats théoriques et expérimentales obtenus
				\end{itemize}
			\end{itemize}
			\item Article:
			\begin{itemize}
				\item Est-ce que le problème est toujours intéressant ?
				\item Est-ce qu'on peux traiter le problème d'une autre manière ?
				\item Est-ce que les hypothèses sont réalistes ?
				\item Est-ce que le travail est applicable dans le contexte actuel ?
				\item Est-ce que tous les aspects du problème ont été traités ?
				\item Existe-t-il d’autres manières pour le résoudre ?
			\end{itemize}
		\end{itemize}
	}
\end{frame}

\begin{frame}{State of the art}{Standardization}
	\begin{figure}
		\includegraphics[width=\columnwidth]{rime.png}
		\caption{\label{fig:rime} Rime Stack}
	\end{figure}
\end{frame}

\begin{frame}{State of the art}{Standardization}
	\begin{figure}
		\includegraphics[width=0.5\columnwidth]{uip.png}
		\caption{\label{fig:uip} Uip Stack}
	\end{figure}
\end{frame}

\begin{frame}{State of the art}{Standardization}
	\begin{figure}
		\includegraphics[width=0.5\columnwidth]{rimeVSuip.png}
		\caption{\label{fig:rimeVSuip} rime VS uip}
	\end{figure}
\end{frame}

%\begin{frame}{State of the art}{Comparison}
%	\begin{table}
%		\begin{tabular}{c|c|c|c|c}
%			Paper & A1 & A2 & A3 & A4 \\\hline
%				  &    &    &    & \\\hline
%				  &    &    &    & \\\hline
%				  &    &    &    & \\\hline
%				  &    &    &    & 
%		\end{tabular}
%		\caption{\label{tab:Tableju} An example table.}
%	\end{table}
%	
%	\note{
%		\begin{itemize}
%			\item Conseils:
%			\begin{itemize}
%				\item Qu'est ce qui réuni et divise tous c'est travaux
%				\item Ce chapitre ne doit pas être une simple revue de la bibliographie
%				\item Présentation des travaux antérieurs et connexes
%				\begin{itemize}
%					\item Choisir les travaux reliés
%					\item Critique des travaux antérieurs
%					\item Description du lien entre le sujet traité dans le mémoire et les travaux antérieurs
%					\item le lien: qqch en commun (méthode, approche, outil ...)
%					\item Cibler les critiques où le candidat apporte des contributions
%					\item Résumer l’analyse de ces travaux dans un tableau récapitulatif
%					\item Il faut analyser les travaux pour proposer une contribution
%					\begin{itemize}
%						\item Classification des travaux
%						\item Avantages \& inconvénients
%						\item Contextes d’utilisation
%					\end{itemize}
%					\item Formulation du problème théorique
%					\item Présentation des hypothèses explicatives
%					\item Suite à la lecture de ce chapitre, le lecteur doit avoir compris la motivation pour le choix du sujet et son importance
%				\end{itemize}
%			\end{itemize}
%		\end{itemize}
%	}
%\end{frame}



\begin{frame}{CoAP application}{State of the art}
	\begin{columns}
	
		\begin{column}{0.5\textwidth}
			\begin{enumerate}
				\item service or application requirements, network topology, and device properties are registered to the controller and stored in the database;
				\item the controller translates service requirements into network QoS requirements. Preprocessing and analysis is performed if necessary;
				\item the controller exploits the algorithm described in Section V-B to schedule flows, in order to fulfill QoS requirements;
				\item the controller sends flow entries to controlled devices in charge of routing flows. A flow entry contains information such as source/destination IP address/port, IP address of next hop, and the new destination IP address;
				\item controlled devices receive flow entries from the controller;
				\item controlled devices identify each flow going through (by source/destination IP address/port), and check whether there is an entry for this flow, then do actions determined by IP address of next hop and the new destination IP address.
				
			\end{enumerate}
			
		\end{column}
		
		\begin{column}{0.5\textwidth}
			\begin{center}
			
				\begin{figure}
					\includegraphics[width=\columnwidth]{flowDiagram.png}
					\caption{\label{fig:flowDiagram.png} Operational Flow Diagram}
				\end{figure}
				
			\end{center}
		\end{column}
		
	\end{columns}
	
\end{frame}

\begin{frame}{CoAP application}{State of the art}

	\begin{table}[h!]
	\begin{center}
		\begin{tabulary}{\textwidth}{|C|C|C|C|C|}
		0 1 & 2 3 & 4-7 & 8-15 & 16-31 \\\hline
		Ver & T & OC & CODE & Message ID \\\hline
		\multicolumn{5}{|c|}{Token} \\\hline
		\multicolumn{5}{|c|}{Options} \\\hline
		\multicolumn{5}{|c|}{Payload} \\\hline
		\end{tabulary}
		\caption{\label{tab:CoapPacket}CoAP message format.}
	\end{center}
	\end{table}

	\hspace*{1.2cm}
	\begin{minipage}{\textwidth}
	\begin{itemize}
		\item[Ver:] is the version of CoAP
		\item[T:] is the type of Transaction
		\item[OC:] is Option count
		\item[Code:] represents the request method (1-10) or response code (40-255).
			\begin{itemize}
				\item Ex: the code for GET, POST, PUT, and DELETE is 1, 2, 3, and 4, respectively.
			\end{itemize}
		\item[Message ID:] is a unique identifier for matching the response.
	\end{itemize}
	\end{minipage}
	
\end{frame}

\begin{frame}{MQTT application}{State of the art}
	\begin{columns}
		
		\begin{column}{0.6\textwidth}
			\begin{itemize}
				\item Message Queue Telemetry Transport
				\item Andy Stanford-Clark of IBM and Arlen Nipper of Arcom
					\begin{itemize}
						\item Standardized in 2013 at OASIS
					\end{itemize}
				\item MQTT uses the publish/subscribe pattern to provide transition flexibility and simplicity of implementation
				\item MQTT is built on top of the TCP protocol
				\item MQTT delivers messages through three levels of QoS
				\item Specifications
					\begin{itemize}
						\item MQTT v3.1 and MQTT-SN (MQTT-S or V1.2)
						\item MQTT v3.1 adds broker support for indexing topic names
					\end{itemize}
				\item The publisher acts as a generator of interesting data.
			\end{itemize}
			
		\end{column}
		
		\begin{column}{0.4\textwidth}
			\begin{center}
			
				\begin{figure}
					\includegraphics[width=\columnwidth]{mqtt.png}
					\caption{\label{fig:mqtt} MQTT application}
				\end{figure}
				
			\end{center}
		\end{column}
		
	\end{columns}
	
\end{frame}

\begin{frame}{MQTT application}{State of the art}

				\begin{table}[h!]
				\begin{center}
					\begin{tabulary}{\textwidth}{|C|C|C|C|}
					0-3          & 4   & 5 6       & 7                   \\\hline
					Message type & DUP & QoS level & Retain              \\\hline
					\multicolumn{4}{|c|}{Remaining length}               \\\hline
					\multicolumn{4}{|c|}{Variable length header}         \\\hline
					\multicolumn{4}{|c|}{Variable length message payload}\\\hline
					\end{tabulary}
					\caption{\label{tab:MqttPacket}MQTT message format.}
				\end{center}
				\end{table}
			
			\hspace*{1.2cm}
%			\addtolenght{\textwidth}{-3cm}
			\begin{minipage}{\textwidth}
				\begin{itemize}
				\item[Message type:] CONNECT (1), CONNACK (2), PUBLISH (3), SUBSCRIBE (8) and so on
				\item[DUP flag:] indicates that the massage is duplicated
				\item[QoS Level:] identify the three levels of QoS for delivery assurance of Publish messages
				\item[Retain field:] retain the last received Publish message and submit it to new subscribers as a first message
				\end{itemize}
			\end{minipage}
%			\addtolenght{\textwidth}{+3cm}
			
\end{frame}

\begin{frame}{XMPP application}{State of the art}
	\begin{columns}
		
		\begin{column}{0.6\textwidth}
		
			\begin{itemize}
				\item Extensible Messaging and Presence Protocol
				\item Developed by the Jabber open source community
				\item An IETF instant messaging standard used for:
					\begin{itemize}
						\item multi-party chatting, voice and telepresence
					\end{itemize}
				\item Connects a client to a server using a XML stanzas
				\item An XML stanza is divided into 3 components:
					\begin{itemize}
						\item message: fills the subject and body fields
						\item presence: notifies customers of status updates
						\item iq (info/query): pairs message senders and receivers
					\end{itemize}
				\item Message stanzas identify:
					\begin{itemize}
						\item the source (from) and destination (to) addresses
						\item types, and IDs of XMPP entities
					\end{itemize}
			\end{itemize}
			
		\end{column}
	
			\begin{column}{0.4\textwidth}
			\begin{center}
			
				\begin{figure}
					\includegraphics[width=\columnwidth]{xmpp.png}
					\caption{\label{fig:xmpp} XMPP application}
				\end{figure}
				
				\begin{figure}
					\includegraphics[width=.7\columnwidth]{xmpp2.png}
					\caption{\label{fig:xmpp2} XML stanza}
				\end{figure}
				
			\end{center}
		\end{column}
		
	\end{columns}
	
\end{frame}

\begin{frame}{AMQP application}{State of the art}
	\begin{columns}
		
		\begin{column}{0.6\textwidth}
		
			\begin{itemize}
				\item Advanced Message Queuing Protocol
				\item Communications are handled by two main components
					\begin{itemize}
						\item exchanges: route the messages to appropriate queues.
						\item message queues: Messages can be stored in message queues and then be sent to receivers
					\end{itemize}
				\item It also supports the publish/subscribe communications.
				\item It defines a layer of messaging on top of its transport layer.
				\item AMQP defines two types of messages
					\begin{itemize}
						\item bare massages: supplied by the sender
						\item annotated messages: seen at the receiver
					\end{itemize}
				\item The header in this format conveys the delivery parameters:
					\begin{itemize}
						\item durability, priority, time to live, first acquirer \& delivery count.
					\end{itemize}
				\item AMQP frame format
					\begin{itemize}
						\item[Size] the frame size.
						\item[DOFF] the position of the body inside the frame.
						\item[Type] the format and purpose of the frame.
							\begin{itemize}
								\item Ex: 0x00 show that the frame is an AMQP frame
								\item Ex: 0x01 represents a SASL frame.
							\end{itemize}
					\end{itemize}
			\end{itemize}
			
		\end{column}
		
		\begin{column}{0.4\textwidth}
			\begin{center}
			
				\begin{figure}
					\includegraphics[width=\columnwidth]{amqp.png}
					\caption{\label{fig:amqp} AMQP application}
				\end{figure}
				
				\begin{figure}
					\includegraphics[width=.8\columnwidth]{amqp2.png}
					\caption{\label{fig:amqp2} AMQP frame format}
				\end{figure}
				
			\end{center}
		\end{column}
	\end{columns}
	
\end{frame}


\begin{frame}{DDS application}{State of the art}
	\begin{columns}
		
		\begin{column}{0.65\textwidth}
		
			\begin{itemize}
				\item Data Distribution Service
				\item Developed by Object Management Group (OMG)
				\item Supports 23 QoS policies:
					\begin{itemize}
						\item like security, urgency, priority, durability, reliability, etc
					\end{itemize}
				\item Relies on a broker-less architecture
					\begin{itemize}
						\item uses multicasting to bring excellent Quality of Service
						\item real-time constraints
					\end{itemize}
				\item DDS architecture defines two layers:
					\begin{itemize}
						\item[DLRL] Data-Local Reconstruction Layer
							\begin{itemize}
								\item serves as the interface to the DCPS functionalities
							\end{itemize}
						\item[DCPS] Data-Centric Publish/Subscribe
							\begin{itemize}
								\item delivering the information to the subscribers
							\end{itemize}
					\end{itemize}
				\item 5 entities are involved with the data flow in the DCPS layer:
					\begin{itemize}
						\item Publisher:disseminates data
						\item DataWriter: used by app to interact with the publisher
						\item Subscriber: receives published data and delivers them to app
						\item DataReader: employed by Subscriber to access received data
						\item Topic: relate DataWriters to DataReaders
					\end{itemize}
			\end{itemize}
			
		\end{column}
		
		\begin{column}{0.35\textwidth}
			\begin{center}
			
				\begin{figure}
					\includegraphics[width=\columnwidth]{dds.png}
					\caption{\label{fig:dds} DDS application}
				\end{figure}
				
			\end{center}
		\end{column}
		
	\end{columns}
	
\end{frame}



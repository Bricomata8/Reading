\subsection*{Context}
\begin{frame}{Context}{Introduction} % Current state

	\begin{figure}
		\includegraphics[width=.7\columnwidth]{context.png}
		\caption{\label{fig:context} The IoT Platform}
	\end{figure}
	
\begin{itemize}
	\item \cite{ndiaye_software_2017} Connect sensors to the gateway\supercite{ndiaye_software_2017}.
	\item Connect the gateway to the infrastructure.
	\item Store \& Analyze sensors data\supercite{thubert_6tisch_2015}. 
\end{itemize}

\note{
	\begin{itemize}
		\item Avant de commencer ma présentation, j'aimerai donner un peit apperçu du context de travaille pour ceux qui ne connaissent pas ...
		\item Délimitation du champ de l'étude
		\begin{itemize}
			\item Ce que vous traiterez.
			\item Ce que vous ne traiterez pas.
		\end{itemize}
		\item Voila c'est juste pour vous donner un petit apperçu sur le context
	\end{itemize}
}

\end{frame}

%%%%%%%%%%%

\subsection*{Problematic}
\begin{frame}{Problematic}{Introduction} % Bad state -> problem

%\Figure{!htb}{1}{stat_interoperability.png}{Key barriers in adopting the Industrial Internet \footfullcite{industrialinternetofthings_executive_}}

\begin{figure}
	\includegraphics[width=.7\columnwidth]{iotChallenges.png}
	\caption{\label{fig:iotChallenges}The IoT problematics}
\end{figure}
\begin{itemize}
	\item How to communicate sensors efficiently
		\begin{itemize}
			\item IEEE 802.15.4, 6LowPAN
			\item Throughput, Delay, Jitter, Loss rate and Availability.
		\end{itemize}
	\item How to communicate sensors with the infrastructure efficiently
		\begin{itemize}
			\item LPWAN, LoraWan
			\item Heterogeneity ?
		\end{itemize}
	\item How to extract knowledge from sensors data.
		\begin{itemize}
			\item Data mining: Classification, Clustering
			\item Deep learning: Machine learning
		\end{itemize}
	\end{itemize}
				\note{
			\begin{itemize}
				\item La manière de poser le problème implique la manière de le résoudre
				\item Savoir ce qui est essentiel de ce qui ne l’est pas.
				\item délimiter un problème
				\item découvrir et rassembler une documentation à son propos
				\item ordonner un contenu
				\item conduire une réflexion personnelle sur le problème choisi
				\item établir des contacts directs avec des personnes, des institutions, des champs d’activités
				\item analyser l’information
				\item exercer son esprit critique
				\item communiquer les résultats de cette procédure d’étude
			\end{itemize}
		}
	\end{frame}
		\begin{frame}[noframenumbering]{Problematic}{Introduction}
		\begin{figure}
		\includegraphics[width=.7\columnwidth]{iotChallenges.png}
		\caption*{\blue{Figure} \ref{fig:iotChallenges}: The IoT problematics}
	\end{figure}
		\begin{itemize}
		\item How to communicate sensors efficiently
			\begin{itemize}
				\item IEEE 802.15.4, 6LowPAN
				\item Throughput, Delay, Jitter, Loss rate and Availability.
			\end{itemize}
		\item How to communicate sensors with the infrastructure efficiently
			\begin{itemize}
				\item LPWAN, LoraWan
				\item \textbf{Heterogeneity ?}
			\end{itemize}
		\item How to extract knowledge from sensors data.
			\begin{itemize}
				\item Data mining: Classification, Clustering
				\item Deep learning: Machine learning
			\end{itemize}
		\end{itemize}
						\note{
				\begin{itemize}
					\item La manière de poser le problème implique la manière de le résoudre
					\item Savoir ce qui est essentiel de ce qui ne l’est pas.
					\item délimiter un problème
					\item découvrir et rassembler une documentation à son propos
					\item ordonner un contenu
					\item conduire une réflexion personnelle sur le problème choisi
					\item établir des contacts directs avec des personnes, des institutions, des champs d’activités
					\item analyser l’information
					\item exercer son esprit critique
					\item communiquer les résultats de cette procédure d’étude
				\end{itemize}
			}
		\end{frame}


%%%%%%%%


\subsection*{Motivation}
\begin{frame}{Motivations}{Introduction} % Who and what could we win if we resolve this probleme, why we didn't let it like it is
	\begin{columns}
		\begin{column}{0.5\textwidth}
					\begin{itemize}
				\item First Motivation
					\begin{itemize}
						\item First Motivation
						\begin{itemize}
							\item First Motivation
							\item Second Motivation
						\end{itemize}
						\item Second Motivation
					\end{itemize}
				\item Second Motivation
					\begin{itemize}
						\item First Motivation
						\item Second Motivation
					\end{itemize}
				\item Third Motivation
					\begin{itemize}
						\item First Motivation
						\item Second Motivation
					\end{itemize}
				\item Fourth Motivation
					\begin{itemize}
						\item First Motivation
						\item Second Motivation
					\end{itemize}
				\end{itemize}
							\end{column}
					\begin{column}{0.5\textwidth}
				\begin{center}
									\begin{figure}
						\includegraphics[width=\columnwidth]{mail.png}
						\caption{\label{fig:}}
					\end{figure}
									\end{center}
			\end{column}
				\end{columns}
		\note{
		\begin{itemize}
			\item Le sujet est il intéressant à traiter ?
			\item qui va bénificie de la solution et comment
			\item quelle est l'impacte de cette solution sur la varie vie
			\item Motivation pour le choix du sujet
			\item Intérêt du sujet
		\end{itemize}
	}
		\end{frame}


%%%%%%%



\subsection*{Goals}
\begin{frame}{Goals}{Introduction}% specifique, mesurables, atteignable, réalistic, time ? 
	\begin{columns}
		\begin{column}{0.5\textwidth}
					\begin{itemize}
				\item First goal
					\begin{itemize}
						\item First goal
						\begin{itemize}
							\item First goal
							\item Second goal
						\end{itemize}
						\item Second goal
					\end{itemize}
				\item Second goal
					\begin{itemize}
						\item First goal
						\item Second goal
					\end{itemize}
				\item Third goal
					\begin{itemize}
						\item First goal
						\item Second goal
					\end{itemize}
				\item Fourth goal
					\begin{itemize}
						\item First goal
						\item Second goal
					\end{itemize}
				\end{itemize}
							\end{column}
					\begin{column}{0.5\textwidth}
				\begin{center}
									\begin{figure}
						\includegraphics[width=\columnwidth]{mail.png}
						\caption{\label{fig:}}
					\end{figure}
									\end{center}
			\end{column}
				\end{columns}
		\note{
		\begin{itemize}
			\item Le sujet est il intéressant à traiter ?
			\item qui va bénificie de la solution et comment
			\item quelle est l'impacte de cette solution sur la varie vie
			\item goal pour le choix du sujet
			\item Intérêt du sujet
		\end{itemize}
	}
		\end{frame}


%%%%%%


\subsection*{Challenges}
\begin{frame}[bg]{Challenges}{Introduction} % Why others didn't resolve the problem
	\begin{itemize}
		\item First Challenge
		\begin{itemize}
			\item L'objectif est de réduire le taux de mortalité
			\item L'objectif est de rendre nos route plus sure
		\end{itemize}
				\item Second Challenge
		\begin{itemize}
			\item Connecter les pietons et le vehicule
			\item augmenter la présision GPS
			\item réduire la latence
		\end{itemize}
				\item Third Challenge
		\begin{itemize}
			\item Connecter les pietons et le vehicule
			\item augmenter la présision GPS
			\item réduire la latence
		\end{itemize}
	\end{itemize}
	\note{
		\begin{itemize}
			\item L’objectif est il ambitieux ?
			\item Définition du but du travail
			\item Méthode pour la vérification et validation des objectifs
		\end{itemize}
	}
\end{frame}

\subsection*{Contributions}
\begin{frame}{Contributions}{Introduction} % just a list
	\begin{itemize}
			\item First contribution
			\begin{itemize}
				\item Privacy settings
				\item Information propagation
				\item 
			\end{itemize}
				\item Second contribution
			\begin{itemize}
				\item Privacy settings
				\item I
			\end{itemize}
				\item Third contribution
			\begin{itemize}
				\item Privacy settings
				\item I
			\end{itemize}
		\end{itemize}
				\note{Nous allons essayer de traiter ce problème qui n'a pas été traité}
		\note{
			\begin{itemize}
				\item Bien expliquer la problématique
				\item Montrer que le problème est intéressant
				\item Montrer que sa résolution est importante
				\item Montrer que les solutions existantes sont limitées
				\item Ce qui manque dans les travaux éxistant
				\item Description claire de la contribution de votre travail
			\end{itemize}
		}
	\end{frame}


%%%%%

\subsection{Conclusion}
\begin{frame}{Conclusion}

	\begin{columns}
		\begin{column}{0.5\textwidth}
%			\newcolumntype{a}{>{\columncolor[gray]{0.8}}c}
			\begin{figure}
				\includegraphics[scale=0.09]{res/mail.png}
				\caption{\label{fig:g0}Cag.}
			\end{figure}
			\begin{figure}
				\includegraphics[scale=0.09]{res/mail.png}
				\caption{\label{fig:g1}Cag.}
			\end{figure}
		\end{column}
		\begin{column}{0.5\textwidth}
			\begin{figure}
				\includegraphics[scale=0.09]{res/mail.png}
				\caption{\label{fig:g2}Cag.}
			\end{figure}
			
			\begin{figure}
				\includegraphics[scale=0.09]{res/mail.png}
				\caption{\label{fig:g3}Cag.}
			\end{figure}
		\end{column}
	\end{columns}

	% Commands to include a figure:
	%\begin{figure}
	%\includegraphics[width=\textwidth]{your-figure's-file-name}
	%\caption{\label{fig:your-figure}Caption goes here.}
	%\end{figure}
	
	\note{
		\begin{itemize}
			\item Rappel des objectifs assignés au début
				\item Synthèse de ce qui a été réalisé
				\item Synthèse de ce qui n’a pas été réalisé
				\item Perspectives du travail
				\begin{itemize}
					\item Améliorations, Extensions, Ouvertures
				\end{itemize}
			\item Il ne faut pas s’arrêter
				\begin{itemize}
					\item Effectuer une autocritique
					\item Identifier les aspects qui nécessitent une amélioration
					\item Identifier les éventuelles améliorations du travail
					\begin{itemize}
						\item Nouvelles hypothèses, Nouveau environnement
					\end{itemize}
				\end{itemize}
		\end{itemize}
	}

\end{frame}


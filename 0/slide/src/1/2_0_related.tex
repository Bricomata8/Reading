\begin{frame}{State of the art}{Standardization}
	
	\note{
		\begin{itemize}
			\item Contenu:
			\begin{itemize}
				\item Tableau comparatif (articles connexes/avantages et désavantages)
				\item Les limites de l’existant
				\item Notre travaille traite le meme x que les travaux précidants mais utilise y au lieu de z (xy/xz)
			\end{itemize}
			\item Procedure:
			\begin{itemize}
				\item Lecture en largeur
				\begin{itemize}
					\item Lecture de beaucoup de papiers connexes
					\item Comprendre le domaine
					\item Comprendre les travaux existants
					\item Sélection des travaux intéressants
				\end{itemize}
				\item Lecture en profondeur
				\begin{itemize}
					\item Lecture et analyse des travaux sélectionnés
					\item Descendre jusqu’au détail du détail
					\begin{itemize}
						\item Poser toujours la question pourquoi?
						\item Être capable d’implémenter de suite l’approche.
					\end{itemize}
				\end{itemize}
				\item Situer le travail par rapport à l’existant sur la base de La problématique traitée
				\begin{itemize}
					\item Les critiques faites sur l’existant
					\item Les hypothèses du travail courant
					\item Les objectifs initiales du travail
					\item Les résultats théoriques et expérimentales obtenus
				\end{itemize}
			\end{itemize}
			\item Article:
			\begin{itemize}
				\item Est-ce que le problème est toujours intéressant ?
				\item Est-ce qu'on peux traiter le problème d'une autre manière ?
				\item Est-ce que les hypothèses sont réalistes ?
				\item Est-ce que le travail est applicable dans le contexte actuel ?
				\item Est-ce que tous les aspects du problème ont été traités ?
				\item Existe-t-il d’autres manières pour le résoudre ?
			\end{itemize}
		\end{itemize}
	}
\end{frame}


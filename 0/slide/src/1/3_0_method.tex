\subsection{Contagion process}

\begin{frame}{Marcov chain}{Methods}


\begin{equation}
V(s, \pi)=\mathbb{E}_{s}^{\pi}\left(\sum_{k=0}^{\mathrm{inf}} \gamma^{k} \cdot r\left(s_{k}, a_{k}\right)\right), s \in \mathbb{S}
\end{equation}

\begin{equation}
r\left(s_{k}, a_{k}\right)=G_{k} \cdot P R R\left(a_{k}\right)
\end{equation}

\begin{equation}
\pi^{*}=\arg \max _{\pi} V(s, \pi)
\end{equation}


\stamp{HGHG}
% \begin{tikzpicture}[remember picture, overlay]
% 	\node[draw, rotate=30] at (25em, 7ex) {\color{red!90}\huge\bfseries APPROVED};
% \end{tikzpicture}

\begin{equation}
PRR=(1-BER)^{L}
\end{equation}

\begin{equation}
BER=10^{\alpha e^{\beta SNR}}
\end{equation}

\end{frame}


\begin{frame}{Genetic Algorithm}{Methods}

\Itemize{
	\item 
	\item S = {SF12, BW125, 4/8, 17 dBm}
	\item Input: 
	\Itemize{
		\item Problem: f(x) = {max($x^{2}$), x \in [0,32]}
		\Itemize{
			\item $x_{1}: 01101_{b}$ 
			\item $x_{2}: 11000_{b}$
			\item $x_{3}: 01000_{b}$
			\item $x_{4}: 10011_{b}$
		}
	}

	\item Method: Genetic algorithm
	\Itemize{
		\item Generate a set of random possible solution
		\item Test each solution and see how good it is (ranking)
		\Enumerate{
			\item Remove some bad solutions
			\item Duplicate some good solutions
			\item Make small changes to some of them (Crossover, Mutation)
		}
	}

	\item Output:
	\Itemize{
			\item $x_{1}$: 01101  (169)  (14.4)
			\item $x_{2}$: 11000  (576)  (49.2)
			\item $x_{3}$: 01000  (64 )  (5.5)
			\item $x_{4}$: 10011  (361)  (30.9)
	}
}
\end{frame}


\begin{frame}{Game theory}{Methods}
\Itemize{
	\item Players: $K = \{1, ... , K\}$
	\item Strategies: $S =S_{1} \times \ldots \times S_{K}$
	\Itemize{
		\item $S_{k}$ is the strategy set of the $k^{th}$ player.
	}
	\item Rewards: $u_{k} : S \longrightarrow R_{+}$ and is denoted by $r_{k} (s_{k} , s_{-k})$
	\Itemize{
		\item $s_{-k}=\left(s_{1}, \dots, s_{k-1}, s_{k+1}, \ldots, s_{K}\right) \in S_{1} \times \ldots \times S_{k-1} \times S_{k+1} \times \ldots \times S_{K}$
	}
}
 \end{frame}


%\note{
%	\begin{itemize}
%		\item Contenu:
%		\begin{itemize}
%			\item L’approche doit être soigneusement détaillée
%			\item Motiver les étapes, les hypothèses, le contexte
%			\item Dérouler un exemple si nécessaire
%			\item Illustrer par des schémas et figures
%			\item Se concentrer sur les aspects où l’approche apporte une contribution
%			\item Montrer comment on est différent de l’existant
%		\end{itemize}
%		\item Conseils:
%		\begin{itemize}
%			\item Présentation de l'approche utilisée pour résoudre le problème posé: approche(contrainte, parametre du probleme) = solution
%			\begin{itemize}
%				\item Justification du choix de l'approche
%				\item Description générale de l’approche comme une boite noir
%				\begin{itemize}
%					\item Entrées, Sorties, Contraintes, Hypothèses
%				\end{itemize}
%				\item Description détaillée de la solution du problème
%				\begin{itemize}
%					\item Description détaillée des étapes: paramètre -> étape1 -> étape2, ... -> solution
%					\item Modélisation des objets manipulés
%				\end{itemize}
%			\end{itemize}
%			\item Mise en œuvre des hypothèses
%			\item Description de la solution du problème
%		\end{itemize}
%		\item Conseils 2:
%		\begin{itemize}
%			\item Utilisez un exemple pour le dérouler tout au long des étapes de l’approche
%			\item Ne parlez dans ce chapitre que de votre travail, ce qu’ont fait les autres est dans l’état de l’art
%			\item Insistez sur les parties où vous apportez des contributions
%			\item Montrez comment votre travail est différent des autres
%			\item Montrez les modules/algorithmes pris de l’existant, ne réinventez pas la roue.
%			\item le lecteur doit pouvoir reproduire les résultats en appliquant la même approche.
%		\end{itemize}
%	\end{itemize}
%}

\begin{frame}{... (step 2)}{Methods}
	\Itemize{
		\item 
		\item 
	}
\end{frame}

\begin{frame}{... (step 3)}{Methods}
	\Itemize{
		\item 
		\item 
	}
\end{frame}

\begin{frame}{... (step 4)}{Methods}
	\Itemize{
		\item 
		\item 
	}
\end{frame}

\begin{frame}{Results}{Comparison}
	\begin{table}[h!]
	\scriptsize
		\begin{tabulary}{\textwidth}{L|L|L|L|L}
		\  &  &  &  &  \\\hline
		\  &  &  &  &  \\\hline
		\  &  &  &  &  \\\hline
		\  &  &  &  &  \\\hline
		\  &  &  &  &  \\\hline
		\end{tabulary}
	\caption{\label{tab:} }
	\end{table}
\end{frame}


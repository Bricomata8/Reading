\section{Introduction}


%%
Collecting data from sensors is an important task as it allows users to understand their environment better.
We propose Web Sensor Data Processing Engine,
	that allows to collect and process sensor data.

The sensors attached with a micro controller in the LoRa module will communicate to the cloud environment through the LoRa gateway.
A web page provides the interface to the residents and to the authorities to gauge the air quality after analyzing the data using the prediction algorithm.

Furthermore,
	it provides a solution for handling the locks that have been employed in and around the village to control the flow of pollution in a timely manner.

The system should also provides an alert mechanism which notifies the different level of authorities through email and SMS in case of any issues.

%%%
Air pollution monitoring has become an essential requirement for cities worldwide.
Currently,
	the most extended way to monitor air pollution is via fixed monitoring stations,
	which are expensive and hard to install.
To solve this problem,
	we have developed EcoSensor,
	a solution to monitor air pollution through mobile sensors.
It is deployed with off-the-shelf hardware such as Waspmote (based on the Arduino platform),
	low-end sensors,
	and Raspberry Pi devices.
EcoSensor collects air pollution using embedded sensors and transfers the captured data to an Android-based device,
	which displays to the user the air pollution levels in real time.
EcoSensor also stores the different pollution traces to a Cloud-based server to analyze the pollution distribution.
The cloud server uses the uploaded data,
	together with highly-accurate data made available by the existing air monitoring infrastructure,
	to create detailed pollution distribution maps using kriging-based spatial prediction techniques.
To optimize the usage of our system,
	we analyze the impact of sensor orientation in the presence of mobility.
Also,
	we analyze the best time and space sampling strategies to determine the most effective data capturing strategy.
Experimental results show that the sensor orientation and the sampling period have a lot less impact on created maps than the actual path taken.



%
Mobile sensing is becoming the best option to monitor our environment due to its ease of use,
	high flexibility,
	and low price.
In this paper,
	we present a mobile sensing architecture able to monitor different pollutants using low-end sensors.
Although the proposed solution can be deployed everywhere,
	it becomes especially meaningful in crowded cities where pollution values are often high,
	being of great concern to both population and authorities.
Our architecture is composed of three different modules:
	a mobile sensor for monitoring environment pollutants,
	an Android-based device for transferring the gathered data to a central server,
	and a central processing server for analyzing the pollution distribution.
Moreover,
	we analyze different issues related to the monitoring process:
	(i) filtering captured data to reduce the variability of consecutive measurements;
	(ii) converting the sensor output to actual pollution levels;
	(iii) reducing the temporal variations produced by mobile sensing process;
	and (iv) applying interpolation techniques for creating detailed pollution maps.
In addition,
	we study the best strategy to use mobile sensors by first determining the influence of sensor orientation on the captured values and then analyzing the influence of time and space sampling in the interpolation process.






%

The rapid development of network and internet technology has supported the growth of the Internet of Things in the World.
With the Internet of Things,
	every sensor and actuator will connect to an object,
	so that information from each object can be communicated to third parties through internet network technology.

The air monitoring process utilizes an array of gas sensors comprising eight sensors,
Other research about air monitoring is the process of air monitoring using Arduino and gadgets as data transmission medium [10].
In this project,
	the measurable data can be directly observed through the gadget.
In this research,
	the system can measure the intensity of CO and the delivery of test data with a good performance because have low of delay and low packet loss.
So the use of Low Power Wide Area Network (LPWAN) technology using LoRa as a transceiver medium is expected to overcome the distance problem in transmit data testing,
	another advantage of LoRa is the low use of power when transmitting information [13-17].
The life span of LoRa batteries is around 10 years [18].
With some of the advantages possessed by LoRa,
	this system can support IoT communication infrastructure [19, 20].





In order to develope pollution sensors to predict pollution based on data sent by sensors network implanted in à construction site.
We design this web application to operate the sensor network and apply Data Mining & Deep Learning to all sensors data.
This application analyze,
	represent and display the correlations between traffic,
	speed of pollution travel and exposure.
Users will be asked to enable GPS on their device to get their current position in order to receive data of the closest sensors.


Design a low cost pollution application:
\Itemize{
	\item Personal Exposure Monitoring
	\item Environmental Alarm
	\item Mobile Sensing
}

The sensor captures information on:
\Itemize{
	\item 
	\item PM
	\item Humidity
	\item Temperature
	\item Pressure
	\item GPS
}

Sends them to a cloud server via Wireless


\documentclass[10pt,a4paper]{.altacv}

\usepackage{graphicx}

\geometry{left=0.5cm,right=8.2cm,marginparwidth=7.6cm,marginparsep=0.3cm,top=0.7cm,bottom=0.5cm}

\setmainfont{Lato}

\definecolor{VividPurple}{HTML}{3E0097}
\definecolor{SlateGrey}{HTML}{2E2E2E}
\definecolor{LightGrey}{HTML}{666666}
\colorlet{heading}{VividPurple}
\colorlet{accent}{VividPurple}
\colorlet{emphasis}{SlateGrey}
\colorlet{body}{LightGrey}

\renewcommand{\itemmarker}{{\small\textbullet}}
\renewcommand{\ratingmarker}{\faCircle}

\addbibresource{sample.bib}

\begin{document}
\name{Aghiles DJOUDI}
\tagline{Ingénieur en sécurité des réseaux informatiques}
\photo{2cm}{res/aghiles}
\personalinfo{
	%\printinfo{symbol}{detail}
	\email{aghilesdjoudi@gmail.com}
	\phone{0780.73.35.11}
	\mailaddress{11 rue des sorrières, 92160,Antony} %xx, rue des sorrières, 92160, Antony}
%	\mailaddress{14 Grande Allée du Champy, 93160, Noisy-le-Grand} %xx, rue des sorrières, 92160, Antony}
%	\location{Paris, FR}
%	\homepage{marissamayr.tumblr.com/}
%	\twitter{@marissamayer}
	\linkedin{linkedin.com/in/aghiles-djoudi-b4b9a3113}
	\github{github.com/Aghiles8}
%	\orcid{orcid.org/0000-0000-0000-0000}
}

\begin{adjustwidth}{}{-8cm}
\makecvheader
\end{adjustwidth}


\cvsection[sidebar]{Expériences}

\cvevent{ \textbf{\large Ingénieur d'études: Évaluation de la vulnérabilité des messageries}}{ Université: Sorbonne Université}{Jan 2018 -- Déc 2018}{Paris, FR}
\begin{itemize}
	\item Étudier l’état de l’art de l'évaluation de la vulnérabilité des utilisateurs.
	\item Appliquer un processus de diffusion de vulnérabilité dans la messagerie.
	\item Afficher aux utilisateurs le score final de la diffusion pour les sensibiliser.
\end{itemize}

\divider

\cvevent{ \textbf{\large Stage: Communication Véhicule \& Piéton (V2P)}}{ École d'ingénieur: Institut Supérieur de l'Automobile et des Transports (ISAT) }{Mar 2017 -- Juil 2017}{Nevers, FR}
\begin{itemize}
	\item Simuler le \textbf {trafic routier} (\textbf {SUMO}) et les \textbf {communications véhiculaires} (\textbf {NS3}).
	\item Développer une application mobile pour détecter les \textbf {collisions V2P}.
%	\item Évaluer et améliorer la \textbf {précision des données GPS}
	\item Analyser les PDR \& RTD du réseau \textbf{LTE} avec \textbf{Fog computing} de l'application.
%	\item Évaluer les performances de l'\textbf {algorithme de détection} de collision
\end{itemize}

\divider

\cvevent{ \textbf{\large Stage: Data Mining }}{ Université: Sorbonne Université}{Mai 2016 -- Juil 2016}{Paris, FR}
L'UPMC et l'Institut de Macao ont conçu des capteurs de pollution.
\begin{itemize}
	\item Concevoir une application mobile d'exploitation des données météorologiques
%	\item Récupérer la position actuelle des capteurs et signaler leurs états.
	\item Classer les données obtenues pour faire une estimation de la pollution.
	\item Analyser, représenter et afficher les corrélations entre les résultats trouvés.
\end{itemize}


%\cvevent{\textbf{\large Développement mobile: Sécuriser le transfert sans fil des données.}}{Start-up: Night4Us}{Mai 2016 -- Juil 2016}{Paris, FR}
%\begin{itemize}
%	\item Étude conceptuelle avec UML et Réalisation de la plateforme.
%	\item Étudier la conception de l'application et modéliser son \textbf {système de sécurité}.
%	\item Sécuriser le transfert d'informations entre les clients et le serveur.
%	\item Réalisation de la plateforme et de l'interface d'utilisation.
%	\item Réfléchir à de nouvelles perspectives pour améliorer l'application.
%\end{itemize}

%\divider

%\cvevent{\textbf{\large Stage: Ingénierie du trafic}}{Entreprise: SONATRACH}{Juin 2015 -- août 2015}{Alger, DZ}
%\begin{itemize}
%	\item Configurer le routage \textbf {IPv4}, \textbf {IPv6} et le routage inter \textbf {VLAN} (Routeurs Cisco).
%%	\item Configurer les points d'accès sans fils et gérer les ACL.
%	\item Gérer les \textbf {ACL} et améliorer le trafic en utilisant les techniques de \textbf {QoS}.
%	\item Analyse du trafic réseau avec \textbf {Syslog et NetFlow}.
%%	\item Contrôler des ACL dans les réseaux IPv4 et IPv6.
%%	\item Configurer et dépanner des VLAN et le routage inter VLAN.
%\end{itemize}

\divider

\cvevent{Stage: Initiateur de natation}{Club: ASCOB}{Oct 2014 -- Juillet 2015}{Alger, DZ}
\begin{itemize}
	\item Entraîner la trame dynamique (flottaison, propulsion, respiration).
	\item Maîtriser les qualités morales d'un éducateur.
	\item Établir une fiche de séance.
	\item Maîtriser les techniques de nage et de sauvetage.
\end{itemize}

%\divider

\cvsection{Projets \& Travaux}

\textbf{\large Projet de recherche: Sécurité du SDN.}
\medskip
\begin{itemize}
	\item Comprendre la problématique des réseaux actuels. Étudier l’état de l’art.
%	\item Analyser et comparer les différents types d'\textbf {architecture du SDN}.
	\item Se familiariser avec \textbf {OpenFlow} et comprendre les difficultés du \textbf {NFV}.
\end{itemize}

\divider

\textbf{\large Projet de recherche: Système de détection d'intrusion intelligent.}
\medskip
\begin{itemize}
	\item Comparer le comportement du réseau en présence et en absence d'attaque.
	\item Configurer les poids du \textbf {réseau de neurone} (Tensorflow) et minimiser l'erreur
\end{itemize}

\divider

\textbf{\large Projet technique: Transmission vidéo sans fil.}
\medskip
\begin{itemize}
	\item Configurer le réseau \textbf {ad-hoc} entre les Raspberry pi et le stabiliser.
	\item Contrôler le \textbf {débit}, le \textbf  {délai de transmission}, la \textbf {gigue} et le \textbf {taux de pertes}.
\end{itemize}

\divider

%\textbf{\large Projet technique: Contrôler un PC avec un smartphone via WiFi.}
%\medskip
%\begin{itemize}
%	\item Étudier le \textbf {système de sécurité} Windows/Linux et les \textbf {méthodes d’accès}.
%	\item Explorer l'architecture des systèmes mobiles et la technologie WiFi.
%	\item Réalisation de l'application et de l'interface d'utilisation.
%	\item Réfléchir à de nouvelles perspectives pour améliorer l'application.
%\end{itemize}

%\divider

\textbf{\large Projet technique: Logiciel client-serveur de calcul distribué.}

\divider

\textbf{\large Projet technique: Logiciel client-serveur sécurisé avec PKI.}

%\clearpage

%\cvsection[sidebar2]{Publications}

%\nocite{*}

%\printbibliography[heading=pubtype,title={\printinfo{\faBook}{Books}},type=book]

%\divider

%\printbibliography[heading=pubtype,title={\printinfo{\faFileTextO}{Journal Articles}}, type=article]

%\divider

%\printbibliography[heading=pubtype,title={\printinfo{\faGroup}{Conference Proceedings}},type=inproceedings]


\end{document}

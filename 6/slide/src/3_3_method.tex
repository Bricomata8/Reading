

\begin{frame}{Etape 3: Calcule de la vulnérabilité sociale}{Théorie de l'influence sociale de Freidkin}

	\Columns{.65}{.35}{
		\Itemize{
			\item Entrée:
			\Itemize{
				\item $Y^{(1)}$ = Vecteur des vulnérabilités individuelles de N utilisateurs (eq \ref{eq:vulnerability})
				\item \alpha = Le niveau de réputation (d'influence) de chaque utilisateur (eq \ref{eq:1})
				\item M  = Matrice d'adjacence N x N
			}
			\item Modèle:
			\Itemize{
				\item[] \Equation{2}{Y^{(t)}=\alpha MY^{(t-1)} + (1 - \alpha) Y^{(t-1)}}
			}
%			\item[] \Figure{h}{.3}{io.png}
			\item Sortie:
			\Itemize{
				\item  $Y^{(t)}$ = Vecteur des vulnérabilités sociales des N utilisateurs
			}
		}
	}{
		\Figure{h}{.73}{oo.png}{Vulnérabilité Sociale}
	}

\end{frame}

\begin{frame}[noframenumbering]
	\movie[autostart,label=cells,width=\textwidth,height=\textheight,showcontrols]{}{movie.mp4}
\end{frame}

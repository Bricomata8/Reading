\subsection{Social (OSN)}

%Stalking Online: on User Privacy in Social Networks
In online social networks,
	users are sometimes either oblivious about their privacy,
	or concerned but underestimate the privacy risks.
% A Study of Online Social Network Privacy Via the TAPE Framework
OSN service providers allow users to manage who can access which information and communication (e.g. Facebook and Google+).
Researcher studied privacy protection from two directions:

%first direction
Along the first direction,
	fundamental changes to the current design of OSN were suggested to enhance users' privacy.
Within this direction, \textbf{Privacy by Design (PbD)} is an important approach.
For example,
	in \cite{baden_persona_2009},
	Baden et al. proposed a new type of OSNs by using \textbf{attribute-based encryption} to hide user data,
	in which symmetric keys are used to encrypt messages and only the designated friend groups can decrypt the messages.
In \cite{erkin_generating_2011},
	Erkin et al. proposed to use \textbf{homomorphic encryption and multi-party computation techniques} to hide privacy-sensitive data from the service provider in a recommender system.

%second direction
The second direction is developing privacy protection tools based on existing OSNs.
In our work,
	we focus on the second direction to deal with current OSNs.

%%%%%%%%%%%%%%%%%%%%%%%%%%%%%%%%%%%%%%%%%%%%%%%%%%%%%%%%%%%%%%%%%%%%%%%%%%%%%%%%%%%%%%%%%%%%%%%%%%%%%%%%%%%%%%%%%%%%%%%%%%%%%%%%%%%%%%%%%%%%%%%%%%%%%%%%%%%%%%%%%%%%%%%%%%%%%%%%
%\subsubsection{Concept of privacy}

%Privometer
The authors in \cite{talukder_privometer_2010} develop a tool,
	\textbf{Privometer},
	to measure information leakage based on user profiles and their social graph.
The leakage is indicated by probability.
Privometer is based on an augmented inference model where a potentially malicious application installed in the user’s friend profiles can access substantially more information.
Privometer is implemented as a Facebook application.
It operates in two modes.
In online mode,
	inference is performed based on the friend’s profile where most frequently value is selected.
In offline mode,
	it uses only immediate friends and "network-only Bayes classifier" to measure the probability of inference.

%1
%\cite{akcora_profiling_2014} focuses on on the risk of new interactions from a privacy point of view.

%2
\cite{b.s._privacy_2015} proposed a privacy control framework for information dispersal on social network,
	they use the \textbf{quadratic form of bezier curve} to arrive at privacy scores for friends,
	they use the communication information for pre-sorting of friends which is lacking in \cite{vidyalakshmi_privacy_2015}.

%3
\textbf{Privacy Index (PIDX)} proposed in \cite{nepali_sonet_2013} is a measure of a user’s privacy exposure in a social network.
PIDX is a numerical value between 0 and 100 with high value indicating high privacy risk in social networks.
Each attributes privacy impact factor is the ratio of its privacy impact to full privacy disclosure,
	is the summation of privacy impact factors of each attribute visible.

%4
%\cite{bilogrevic_multi-dimensional_2014} study the privacy of Social Relationships in Pervasive Networks.

%5
\cite{akcora_risks_2012} develop a \textbf{graph-based approach and a risk model} to learn risk labels of strangers,
	the intuition of such an approach is that risky strangers are more likely to violate privacy constraints.

%Privacy Wizard
Fang and Le Fevre \cite{fang_privacy_2010} proposed a \textbf{Privacy Wizard}.
	the goal of this tool is to automatically configure a user’s privacy settings with minimal effort and interaction from the user.

%PViz
Similarly,
	Mazzia and Adar \cite{mazzia_pviz_2012} introduce \textbf{PViz},
	a system that allows users to understand the visibility of their profiles according to natural sub-groupings of friends,
	and at different levels of granularity.
PViz relies on a graphical output model that illustrates the user’s social network.

%PRISM
Patil and Kobsa \cite{patil_enhancing_2010} propose the \textbf{PRISM (PRIvacy-Sensitive Messaging)} system,
	providing Internet Messaging users with various informative visualizations as well as mechanisms for presenting oneself differently to various groups of contacts.

%Risk score
In a similar vein,
	some studies \cite{maximilien_privacyasaservice_2009} propose a methodology for quantifying the risk posed by a user’s privacy settings.
A \textbf{risk score} reveals to the user how far his/her privacy settings are from those of other users.
It provides feedback regarding the state of his/her existing settings.
However,
	it does not help the user refine his/her settings in order to achieve a more acceptable configuration.

%NARS
On the other hand,
	Carmagnola et al. \cite{carmagnola_sonars_2009} recently presented \textbf{NARS},
	an algorithm for social recommender systems.
NARS targets users as members of social networks,
	suggesting items that reflect the trend of the network itself,
	based on its structure and on the \textbf{influence relationships} between users.

%Trust metrics can be classified to two main categories: \textbf{global and local trust metrics}.
%%Global trust metrics
%Global trust metrics [7] predict a global reputation value for each node. %PageRank
%%Local trust metrics
%Local trust metrics [16],
%	on the other hand,
%	compute trust values that are dependent on the target user.
%Local trust metrics take into account the very personal and subjective views of the users and predict different values of trust in other users for every single user.
%%TaRS
%TaRS \cite{massa_trust-aware_2007} builds a recommendation system capable of operating both global and local trust metrics.

Trust metrics can be classified to two main categories: global and local trust metrics.
%Local trust metrics
\textbf{Local trust metrics},
	compute trust values that are dependent on the target user,
	Local trust metrics take into account the very personal and subjective views of the users,
	they predict different values of trust for every single user based on their own experience.
%Global trust metrics
\textbf{Global trust metrics},
	on the other hand,
	predict a global reputation value for each node.

%%%%%%%%%%%%%%%%%%%%%%%%%%%%%%%%%%%%%%%%%%%%%%%%%%%%%%%%%%%%%%%%%%%%%%%%%%%%%%%%%%%%%%%%%%%%%%%%%%%%%%%%%%%%%%%%%%%%%%%%%%%%%%%%%%%%%%%%%%%%%%%%%%%%%%%%%%%%%%%%%%%%%%%%%%%%%%%%
\subsubsection{Concept of trust}

%Privacy and Social Capital in Online Social Networks
The concept of trust is used to indicate the relationship between two entities.
%Trust-involved access control in collaborative open social networks
Trust in a person is a commitment to an action based on a belief that the future actions of that person will lead to a good outcome.
There are three main properties of trust that are relevant to the development of algorithms for computing it \cite{wang_trustinvolved_2010},
	namely,
	transitivity,
	asymmetry,
	and personalization.
%transitivity
The primary property of trust that is used in our work is transitivity.
	if Alice highly trusts Bob,
	and Bob highly trusts Chuck,
	it does not always and exactly follow that Alice will highly trust Chuck.
%asymmetry
It is also important to note the asymmetry of trust,
	for two people involved in a relationship,
	trust is not necessarily identical in both directions.
%personalization
The third property of trust that is important in social networks is the personalization of trust,
	trust is inherently a personal opinion,
	two people often have very different opinions about the trustworthiness of the same person.
%%%%%%%%%%%%%%%

%10
%Protect_ U & Privacy Wizard
While much work has focused on tools for understanding and adjusting existing privacy settings,
	\textbf{Protect\_ U} \cite{gandouz_protect_2012}uses machine learning techniques to recommend privacy settings based on a user’s personal data and trustworthy friends.
Protect\_ U analyzes user profile contents and ranks them according to four risk levels: Low Risk,
	Medium Risk,
	Risky and Critical.
The system then suggests personalized recommendations to allow users to make their accounts safer.
In order to achieve this,
	it draws upon two protection models: local and community-based.
The first model uses the Facebook user’s personal data in order to suggest recommendations,
The second model seeks the user’s trustworthy friends to encourage them to help improve the safety of their counter part’s account.

%Social Market
Despite the mole of work on social trust,
	\textbf{Social Market} is the first system to propose the use of trust relationships to build a \textbf{decentralized interest-based marketplace}.

%9 TAPE
Similarly,
	\textbf{TAPE} \cite{yongbozeng_study_2015} is the first attempt to combine explicit and implicit social networks into a single gossip protocol.
Zeng et al. \cite{yongbozeng_study_2015} approaches the privacy quantification problem from a different angle.
First,
	they consider how likely a friend reveals others’ personal information,
	by computing the \textbf{privacy trust score},
	which is a widely studied research problem \cite{gundecha_exploiting_2011}.
Furthermore,
	the proposed work is related to information diffusion in OSNs such as \cite{fang_privacy_2010}.
Finally,
TAPE framework differs from other work,
	in considering information diffusion in the context of privacy protection,
	which requires different sets of features and considerations.

%9
%Zeng and Xing \cite{yongbo_zeng_study_2015}
%	studied how individual users can expand their social networks by making trustful friends who will not leak their private IFs to unknown parties.

%Ostra
\textbf{Ostra} \cite{mislove_ostra_2008} utilizes trust relationship to thwart unwanted communication,
	where the number of a user’s trust relationships is used to limit the amount of unwanted communications he can produce.
Ostra relies on existing trust networks to connect senders and receivers via \textbf{chains of pair-wise trust relationship},
	they use a pair-wise link-based credit scheme to impose a cost on originator of unwanted communication.
Unfortunatly,
	the scalability of this system stays uncertain as it employs a per-link credit scheme.

%%8
%In \cite{gundecha_exploiting_2011},
%	Sun et al proposed a \textbf{probability trust model} that uses Beta function to address concatenation propagation and multi-path propagation of trust.

%
In \cite{zeng_trustaware_2014},
	Sun et al proposed a \textbf{probability trust model} that uses Beta function to address concatenation propagation and multi-path propagation of trust.

%LENS: Leveraging Social Networking and Trust to Prevent Spam Transmission

%SOAP
\textbf{SOAP} \cite{li_soap_2011} presents a social network based \textbf{personalized spam filter} that integrates social closeness,
	user (dis)interest and adaptive trust management into a Bayesian filter.
However,
	several issues with SOAP,
	including the intrinsic cost of initialization and continuous adaptation of social closeness (between sender and recipient) 
		and social interests (of an individual) in the Bayesian filter,
	limit its usage.

%6
Relationship between \textbf{user’s trustworthiness} and privacy risk is presented in \cite{pandey_computing_2015}.

%%%Social Market: Combining Explicit and Implicit Social Networks

%%SybilGuard
%SybilGuard [23] and SybilLimit [22] propose protocols that exploit trust relationships between friends to protect peer-to-peer systems from sybil attacks.

%%NABT
%NABT [15] proposes the use of trust between friends to prevent free-riding behaviors using an indirect trust relationships,
%	NABT’s credit-based approach can be viewed as a basic form of trust inference between friends of friends.
%A more advanced approach to trust-inference is adopted by SUNNY [14],
%	a centralized protocol that takes into account both trust and confidence to build a Bayesian network.

%%TrustWalker
%TrustWalker [10] combines trust and item-based collaborative filtering.

\paragraph{Trust networks and trust metrics}

In \textbf{trust networks} users can ask to rate other users,
	this means that,
	a user can express her level of trust in another user she has interacted with,
	i.e. express a trust statement such as "Alice, trust Bob as 0.8 in [0,1]".
The system can then aggregate all the trust statements in a single trust networks representing the relationships between users.

\textbf{Trust metrics} are algorithms whose goal is to predict,
	based on the trust network,
	the trustworthiness of "unknown" users,
	i.e. users in which a certain user didn’t express a trust statement.
Their aim is to reduce social complexity by suggesting how much an unknown user is trustworthy.

%%%%%%%%%%%%%%%%%%%%%%%%%%%%%%%%%%%%%%%%%%%%%%%%%%%%%%%%%%%%%%%%%%%%%%%%%%%%%%%%%%%%%%%%%%%%%%%%%%%%%%%%%%%%%%%%%%%%%%%%%%%%%%%%%%%%%%%%%%%%%%%%%%%%%%%%%%%%%%%%%%%%%%%%%%%%%%%%%%
\subsubsection{Concept of reputation}

Trust and reputation concepts are used in order to preserve user’s privacy while increasing their \textbf{social capital} in OSNs.
%Here it has a balance point of social capital and privacy thresholds that maximizes correct information diffusion while minimizing illegal private information leak out,
%	given user’s risk motivation for preserving privacy.
%Privacy and Social Capital in Online Social Networks
Reputation concept is used to refer to a more general sense of trust towards a particular entity based on opinions of multiple entities.

%Social network analysis for cluster-based IP spam reputation
%Reputation-based systems.
A reputation system collects,
	distributes and aggregates feedback about participants’ past behavior.
Such systems help people decide whom to trust,
	encourage trustworthy behavior and deter participation by those who are unskilled or dishonest.
%Real-time reputation-based systems.
Various applications use real-time reputation-based systems,
	including online markets and anti-spam solutions.
Anti-spam reputation systems generate a score,
	or rating,
	for each incoming message or IP,
	based on analysis of various parameters: message volume,
	type of traffic (e.g. sporadic vs continuous),
	rate of user complaint reports,
	feedback from spam traps,
	compliance with regulations,
	etc.
This aggregated information,
	collected over time,
	forms the reputation of the sender.

%SNARE
SNARE \cite{hao_detecting_2009} infers the reputation of a message sender based on \textbf{network-level features},
	without looking at the contents of a message.
Using an automated reputation engine,
	SNARE classifies message senders as spammers or legitimate with about a 70\% detection rate for less than a 0.3\% false positive rate.
However,
	lacking authentication and non-repudiation in standard trust and reputation solution make these solutions be subject to identity spoofing,
	false accusation and collusion attacks.
Further,
	these solutions consume extra valuable resources of messaging servers on message reception and filtering.
A recently developed sender reputation engine named SNARE (Spatio-Temporal Network Level Automatic Reputation Engine),
	can automatically classify messages based on a combination of various lightweight network-level features 
	(e.g. geodesic distance between sender and recipient, number of recipients).
The most influential feature in the system was the AS number of the sender.

%TrustMail
In \textbf{TrustMail},
	which is a prototype E-mail client,
	an approach is proposed that makes use of OSN reputation ratings to attribute different scores to E-mails \cite{golbeck_reputation_2004}.
The actual benefit of this system is that,
	by using social network data,
	it identifies potentially important and relevant messages even if the recipient does not know the sender \cite{golbeck_reputation_2004}.

%cluster-based reputation

%1 reputation based approach
Kuter and Golbeck \cite{kuter_semantic_2009} targeted Web Ontology Language for Services (OWL-S) and followed a \textbf{reputation based approach} for selecting highly trusted composite web service.
Authors developed a service-composition algorithm, called Trusty, to compute social trust in OWL-S style semantic Web services.

%2 multi-agent based reputation model
Paradesi et al. \cite{paradesi_integrating_2009} adopted a \textbf{multi-agent based reputation model} to define \textbf{trustworthiness of services}.
Moreover,
	they developed a trust framework to derive trust for a composite service from trust model of component services.

%%%%%%%%%%%%%%%%%%%%%%%%%%%%%%%%%%%%%%%%%%%%%%%%%%%%%%%%%%%%%%%%%%%%%%%%%%%%%%%%%%%%%%%%%%%%%%%%%%%%%%%%%%%%%%%%%%%%%%%%%%%%%%%%%%%%%%%%%%%%%%%%%%%%%%%%%%%%%%%%%%%%%%%%%%%%%%%%
\subsubsection{Collaborative management}

%Trust-involved access control in collaborative open social networks
%collaborative community

The trust value assigned to a person in previous work is estimated on the basis of his/her reputation,
	which can be assessed taking into account the person behaviour.
Indeed,
	it is a matter of fact that people assign to a person with unfair behaviour a bad reputation and,
	as a consequence,
	a low level of trust.
A possible solution is to estimate the trust level to be assigned to a user in a collaborative community on the basis of his/her reputation,
	given by his/her behavior with regards to all the other users in the community.

%In our scenario,
%	this can be done by making a user able to monitor the behaviour of the other users with respect to the release of private information or resources.
%However,
%	this generate serious privacy concerns,
%	because a participant might not agree in releasing information about the decisions he/she has made,
%	even if these are signals of good behaviour.

%Detecting and Resolving Privacy Conflicts for collaborative Data Sharing in Online Social Networks
Hu et al. \cite{hu_detecting_2011} propose an approach to enable \textbf{collaborative privacy management} of shared data in OSNs.
In particular,
	they provide a systematic mechanism to identify and resolve privacy conflicts for collaborative data sharing.
their conflict resolution indicates a trade-off between privacy protection and data sharing by quantifying privacy risk and sharing loss

%Interdependent Privacy: Let Me Share Your Data
%collaborative
Dealing with \textbf{collaborative information sharing},
	Hu et al. \cite{biczok_interdependent_2013} proposed a method to detect and resolve privacy conflicts.

%COAT : Collaborative Outgoing Anti-Spam Technique
%community collaboration
The collaborative systems, called \textbf{COAT} \cite{ahmad_coat_2012}, do not rely upon semantic analysis but on the community to identify spam messages.
Once a message is tagged as spam by one SMTP server,
	the signature of that message is transmitted to all other SMTP servers.
This class requires the collaboration of multiple SMTP servers to implement the system.

%SocialFilter
\textbf{SocialFilter} \cite{yang_socialfilter_2009} proposes a collaborative spam mitigation system that uses social trust embedded in OSN to asses the trustworthiness of Spam reporter.
The spammer reports from the SocialFilter nodes are stored at a centralized repository that computes the trust values of the reports and identifies spammers based on IP addresses.
However,
	the SocialFilter’s effectiveness is doubtful as spammers may use dynamic IPs.




% Frame step2
\begin{frame}{Etape 2: Calcule de la réputation des utilisateurs}{Méthode}
	\Columns{0.65}{0.35}{
		
		\Itemize{
			\item Entrée:
			\Itemize{
				\item Fréquence d'utilisation de la messagerie.
				\item Horaire, durée des échanges (1:5)
				\item \% des échanges chiffrés, signés, claires (1:3)
				\item Importance des interlocuteurs: Liste favoris (2), noir(1)
				\item Type de données: Texte, images, vidéos, script (1:4)
			}
			\item Méthode:
			\Itemize{
				\item Loi binomiale
			}
			\item Output:
			\Itemize{
				\item[] \Equation{1}{P(reputation) = P(X \ge 1)~=~1~-~(1~-~P(trust))^{n}}
				\item Where,
				\Itemize{
					\item X: Niveau de confiance, X \sim B(n,p)
					\item n:~deg(noeud)
					\item P(X=1): La probabilité de se faire attribué une confiance par un interlocuteur
				}
			}
		}
	}{
		\Figure{h}{.7}{p_trust.png}{Niveau de réputation}
%		\movie[autostart,label=cells,width=\textwidth,height=0.5\textheight,showcontrols]{}{movie.mp4}
	}
\end{frame}


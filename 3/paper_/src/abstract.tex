\begin{abstract}

% Problem
Major traffic light's control systems of actual cities are wired and have a static behavior.
They are only time-based or with a pre-configured pattern.
% Existing solutions
Even if the connectivity between vehicles and lights is now possible,
	it remains insufficient to ensure adaptability,
	scalability and interoperability among wireless and wired networks.
One of the main criterion of such systems is Quality of Service (QoS) or delay aware.
%System with new applications needs the interoperability between all wireless networks with delay aware of data exchange.
% Our method and results
In this paper,
	we propose an Urban Traffic Light Control based on an IoT network architecture (IoT-UTLC).
The objective is to interconnect both vehicles and roads' infrastructure to traffic lights through an IoT Cloud platform.
We designed our IoT-UTLC by selecting sensors,
	actuators,
	wireless motes and protocols.
Message Queuing Telemetry Transport (MQTT) protocol has been integrated to manage QoS.
It enables lights to adapt remotely and smoothly interrupt traffic light's classic cycles.
%\blue{Our experimental results show that the MQTT protocol is efficient when the packets rate exceed 35\% of traffic flow,
%it reduces traffic delay up to 0.05s at 90\% of congestion}.
After verification and validation using a UPPAAL model checker,
	our system has been prototyped.
Motes functions have been implemented on Contiki OS and connected through a 6LoWPAN wireless network.
Time-stamping messages have been performed throughout the system to evaluate the MQTT protocol with different reliability levels and data rates.
Our experimental results show the MQTT protocol decreases the packets delays when the packets number exceeds 35\% of all the traffic.

\end{abstract}


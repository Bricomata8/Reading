\section{Related work}

%%%%%%%%%%%%%%%%%%%%%%%%%%%%%%%%%%%%%%%%%%%%%%%%%%%%%%%%%%%%%%%%%% Research categories %%%%%%%%%%%%%%%%%%%%%%%%%%%%%%%%%%%%%%%%%%%%%%%%%%%%%%%%%%%%%%%%%%%%%%%%%%%
To evaluate the privacy risk of social network users,
	trust metrics are used to measure the extent to which users can be trusted.
Trust metrics can be classified into two main categories:
	global and local trust metrics.

%Local trust metrics 
Local trust metrics,
	compute trust values that are dependent on the target user,
	they take into account the very personal and subjective views of the users,
	they predict different values of trust for every single user based on their own experience.
%Global trust metrics 
Global trust metrics,
	on the other hand,
	predict a global reputation value for each node,
	based rather on the experience of all other users or on the topology of the social network.

%\subsection{Local Trust-based methods}

%%%%%%%%%%%%%%%%%%%%%%%%%%%%%%%%%%%%%%%%%%% Discuss important findings and find some relevance for them %%%%%%%%%%%%%%%%%%%%%%%%%%%%%%%%%%%%%%%%%%%%%%%
%Protect_U & Privacy Wizard
While much work has focused on tools for understanding and adjusting existing privacy settings,
	Protect\_U \cite{gandouz_protect_2012} uses machine learning techniques to recommend privacy settings based on a user’s personal data and trustworthy friends.
Protect\_U analyzes user profile contents and ranks them according to four risk levels: Low Risk, Medium Risk, Risky and Critical.
The system then suggests personalized recommendations to allow users making their accounts safer.
In order to achieve this,
	it draws upon two protection models: local and community-based.
The first model uses the user’s personal data in order to suggest recommendations,
The second model seeks the user’s trustworthy friends to encourage them to help improve the safety of their counter part’s account.

%Social Market
Despite the mole of work on social trust,
	Social Market \cite{frey_social_2011} is the first system to propose the use of trust relationships to build a decentralized interest-based marketplace.
%9 TAPE
Similarly,
	TAPE \cite{yongbozeng_study_2015} is the first attempt to combine explicit and implicit social networks into a single gossip protocol.
Zeng et al. \cite{yongbozeng_study_2015} approaches the privacy quantification problem from a different angle.
First,
	they consider how likely a friend reveals others’ personal information,
	by computing the privacy trust score,
	which is a widely studied research problem \cite{gundecha_exploiting_2011}.
Furthermore,
	the proposed work is related to information diffusion in OSNs such as \cite{fang_privacy_2010}.
TAPE framework differs from other work,
	in considering information diffusion in the context of privacy protection,
	which requires different sets of features and considerations.

%9
Zeng and Xing \cite{zeng_trustaware_2014}
	studied how individual users can expand their social networks by making trustful friends who will not leak their private information to unknown parties.
This work proposes a security risk estimation framework of social networking privacy to calculate the probability of individual privacy leakage through the social graph.
The framework is composed of two parts,
	the calculation of Individual Privacy Leakage Probability (IPLP) and the Relationship Privacy Leakage Probability (RPLP).
Relationship Privacy Leakage Probability considers the factors of relationship strength and interactive behaviors.
Two vectors namely privacy protection awareness (PPA) and privacy protection trust (PPT) are proposed in this paper to estimate IPLP.

%IPLP is regarded as the probability user would gossip others’ privacy in his/her moments,
%	 while the user has heard and been attracted by others’ personal information,

%Ostra
Ostra \cite{mislove_ostra_2008} utilizes trust relationship to thwart unwanted communication,
	where the number of a user’s trust relationships is used to limit the number of unwanted communications he can produce.
Ostra utilizes the existing trust relationship among users to charge the senders of unwanted messages and thus block spam.
It relies on existing trust networks to connect senders and receivers via chains of a pair-wise trust relationship,
	they use a pair-wise link-based credit scheme to impose a cost on the originator of the unwanted communication.
Unfortunately,
	the scalability of this system stays uncertain as it employs a per-link credit scheme.

%8
Gundecha et al. \cite{gundecha_exploiting_2011} propose a feasible approach to the problem of identifying a user’s vulnerable friends on a social networking site.
Vulnerability is somewhat contagious in this context.
Their work differs from existing work addressing social networking privacy by introducing a vulnerability-centered approach to a user security on a social networking site.
On most social networking sites,
	privacy-related efforts have been concentrated on protecting individual attributes only.
However,
	users are often vulnerable through community attributes.
Unfriending vulnerable friends can help protect users against the security risks.

%LENS: Leveraging Social Networking and Trust to Prevent Spam Transmission
Hameed \cite{hameed_lens_2011} proposed LENS,
	which extends the friend of friend network by adding trusted users from outside of the FoF networks to mitigate spam beyond social circles.
Only emails to a recipient that have been vouched by the trusted nodes can be sent into the network.
The authors proposed using social networks and trust and reputation systems to combat spam.
In contrast,
	LENS can reject unwanted email traffic during the SMTP time.

%SocialEmail
SocialEmail \cite{tran_social_2010} considers trust as an integral part of networking rather than working alongside an existing communication system.
SocialEmail leverages social network trust paths to rate the messages.
The key feature of SocialEmail is that instead of directly connecting the sender and the recipient,
	messages are routed through existing friendship links.
This gives each email recipient control over who can message him/her,
In contrast,
	such social interaction-based methods are not sufficiently effective in dealing with legitimate emails from senders outside of the social network of the receiver.

%Social interactions
Social interactions (e.g.,
	the exchange of messages between users) have been suggested as an indicator of interpersonal tie strength \cite{xiang_modeling_2010}.
As a consequence,
	an unsupervised model has been developed to estimate the 
		relationship strength from the interaction activity and the user similarity in the OSN \cite{xiang_modeling_2010}.
Although interaction-based methods leverage social relationships for extracting trust,
	the applications are not designed to be automated in the sense that the user must explicitly score other users,
	score messages,
	create whitelists or adjust the credits.

Vidyalakshmi et al. \cite{b.s._privacy_2015} proposed a privacy control framework for information dispersal on social network,
	they use the quadratic form of bezier curve to arrive at privacy scores for friends,
	they use the communication information for pre-sorting of friends which is lacking in \cite{vidyalakshmi_privacy_2015}.
Similarly,
Akcora et al. \cite{akcora_risks_2012} develop a graph-based approach and a risk model to learn risk labels of strangers,
	the intuition of such an approach is that risky strangers are more likely to violate privacy constraints.

Privacy Index (PIDX) proposed in \cite{nepali_sonet_2013} is a measure of a user’s privacy exposure in a social network.
PIDX is a numerical value between 0 and 100 with a high value indicating high privacy risk in social networks.
An attribute’s privacy impact factor is a ratio of its privacy impact to full privacy disclosure.
Thus,
	an attribute’s privacy impact has a value between 0 and 1.
They consider the privacy impact factor for full privacy disclosure is 1.

%Privacy Wizard
Fang and Le Fevre \cite{fang_privacy_2010} proposed a Privacy Wizard to help users grant privileges to their friends.
	the goal of this tool is to automatically configure a user’s privacy settings with minimal effort and interaction from the user.
The wizard asks users to first assign privacy labels to selected friends,
	and then uses this as input to construct a classifier which classifies friends based on their profiles and automatically assign privacy labels to the unlabeled friends.
%Risk score
In a similar way,
	some studies \cite{maximilien_privacyasaservice_2009} propose a methodology for quantifying the risk posed by a user’s privacy settings.
A risk score reveals to the user how far her privacy settings are from those of other users.
It provides feedback regarding the state of her existing settings.
However,
	it does not help the user refine her settings in order to achieve a more acceptable configuration.

Abdul-Rahman and Hailes The trust model presented by Abdul-Rahman and Hailes \cite{abdul-rahman_supporting_2000} 
	is focused on virtual communities related to e-commerce and artificial autonomous agents.
The model defines direct trust and recommender trust.
Direct trust is the trust of an entity in another one based on direct experience.
Whereas recommender trust is the trust of an entity in the ability to provide good recommendations.
Trust can only have discrete labeled values,
	namely Very Trustworthy,
	Trustworthy,
	Untrustworthy,
	and,
	Very Untrustworthy for direct trust,
	and Very good,
	good,
	bad and,
	very bad for recommender trust.
The difference between the two ratings from different entities can be computed as semantic distance.
This semantic distance can be used to adjust further recommendations.
The combination of ratings is done as a weighted sum,
	where the weights depend on the recommender trust.

All previous work didn't take into consideration the topological aspect of interactions to measure social vulnerabilities of users.
The closest study to our approach is that presented in \cite{zeng_trustaware_2014}.
However,
	this solution doesn't study the social interaction environment of users and the potential information leakage through a vulnerable social environment.
In this paper,
	we study the impact of trusting vulnerable users in preserving the privacy of all users in the communication network.

%For our best knowledge,
%	it does not exist a study that
%In this regard,
%	this is the first paper that 


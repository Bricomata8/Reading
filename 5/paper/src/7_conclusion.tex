\section{Conclusions} \label{sec:Conclusions}

% Restate the main challenges
In this article,
	we have investigated homogeneous and heterogeneous networks for large scale deployments (up to 10000 nodes per gateway),
	SFs allocations,
	and performance in terms of PDR,
	throughput,
	and power consumption.
% Restate the main contribution
We have developed a LoRa module in the WSNet simulator considering the co-channel rejection due to the quasi-orthogonality of SFs,
	the gateway capture effect based on the SX1301 transceiver,
	and a random access MAC to obtain accurate results for large scale deployments.
% Restate the main findings
Our simulations compare the performance for homogeneous and heterogeneous deployments as a function of the number of nodes and traffic intensity.
First,
	we have analyzed homogeneous deployments for different SFs from SF6 to SF12.
Simulations show better performance for the SF6 deployment but reduced cell coverage.
Second,
	we have compared heterogeneous and homogeneous deployments:
	the Heterogeneous deployment that selects its LoRa configuration according to its link budget results in the best PDR and throughput,
	as well as the lowest average power consumption compared to other deployments,
	for a different number of nodes and different traffic intensity.
The results clearly show the benefits of heterogeneity for large scale network deployments and the need for adaptive SF allocation strategies.
% Future challenges current bad state
In future work,
	we plan to define and simulate several decision modules to enable adaptive LoRa operation.
The module will optimally select the configuration according to the scenario criteria (e.g.,
	high data rate,
	energy efficiency,
	or network congestion) and the radio environment (e.g.,
	link budget,
	level of interference,
	device mobility).
It should minimize the overhead and the impact of bidirectional communications on performance.





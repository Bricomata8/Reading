\begin{abstract}

% Statistics
LoRa technology has emerged as an interesting solution for Low Power Wide Area (LPWA) applications.
% Challenges
The main LPWA research directions are about large scale networks to support massive number of devices,
	interference issues,
	link optimization and adaptability.
% Problem
Thus heterogeneous network deployments and Spreading Factor (SF) allocation strategies need to be studied.
In this paper,
	we investigate the performance of homogeneous networks (i.e.
when all the nodes select the same LoRa configuration) and heterogeneous networks (i.e.
when each node selects its LoRa configuration according to its link budget or their needs) for large scale deployments (up to 10000 nodes per gateway).
% Contribution
For that purpose we have developed a LoRa Module,
	based on improved WSNet simulator,
	including a spectrum usage abstraction,
	the co-channel rejection due to the quasi-orthogonality of SFs and the gateway capture effect.
% Experimentation & results
Simulation results show the performance comparison in terms of reliability,
	network capacity and power consumption for homogeneous and heterogeneous deployments as a function of the number of nodes and the traffic intensity.
The comparison shows the benefits of the heterogeneous deployment where each node selects its configuration according to its link budget.

\end{abstract}


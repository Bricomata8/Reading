\section{Results exploitation} \label{sec:Results exploitation}

% Initiation
Below we report the results of applying the contagion process model to the Enron Email dataset.

\subsection{Deployment Scenarios and Assumptions}
In this study,
	we have simulated the deployment scenarios described in Section III.F with a SX1301 gateway,
	at the center,
	configured with one single channel of 125 kHz and up to 10000 SX1276 nodes.
Hence,
	the gateway can simultaneously receive up to 7 packets configured with different SF each (SF6 to SF12).
Packets are dropped according to the modulation and interference models.
Table III presents the simulation parameters.
For sake of simplicity,
	we have considered the 868 MHz band and the Okumura Hata pathloss model.
	
\subsection{Homogeneous Deployments Scenarios}

In this subsection,
	we present the results of simulations for homogeneous deployments with different SFs as a function of the number of nodes (up to 10000) with AP set to 60 s (i.e.,
	each node transmits a packet every minute).
We have to emphasize that nodes are deployed in a disk of radius equal to the maximum transmission range of SF i for each scenario,
	i.e.
the Homogeneous SF6 in a disk of radius D max (SF 6 ) and so on.
Figure 2 shows PDR for homogeneous deployments with different SFs from 6 to 12.
We can see that PDR decreases when the number of nodes increases.
The SF6 Homogeneous deployment has better PDR compared with the others due to its short air time but its range is reduced


Figure 3 shows the throughput in packets per second for homogeneous deployments with different SFs.
We can see that when the number of nodes increases,
	the throughput becomes saturated.
The SF12 Homogeneous deployment converges faster and presents the lowest throughput due to its low data rate.
Nevertheless,
	it presents the longest range.
We have extended the previous analysis by varying AP up to 40 min and we set the number of nodes to 10000.
As shown in Figure 4,
	increasing the Application Period reduces the probability of collision,
	then PDR increases.
Figure 5 shows the throughput as a function of the Application Period.
As we can observe,
	the throughput is higher when nodes use lower SF.
Thus,
	in dense networks with high intensity traffic,
	we have to prioritize the use of the SF6 configuration.
Nevertheless,
	it reduces the range and therefore,
	multi-hop communications or denser deployment with several gateways may be necessary to cover the same area.
Figure 6 compares the average power consumption per node for the homogeneous deployments as a function of the Application Period for a simulation time offers 100*AP.
Increasing traffic intensity (e.g. AP from 40 min to 1 min) results in increased power consumption.
SF6 configuration presents low consumption due to its shorter time on air.
Higher consumption for the SF12 configuration due to its longer time on air.

\subsection{Heterogeneous vs. Homogeneous Deployment Scenarios}
In this subsection,
	we compare the simulation results of the heterogeneous strategies with Homogeneous SF12 and Multi-Homogeneous deployments.
Each deployment covers the same area (disk of radius D max (SF 12 )).
For homogeneous deployments,
	only Homogeneous SF12 is used for comparisons in order to keep the same cell coverage.
We simulate the heterogeneous deployments with up to 10000 nodes and AP is fixed to 60 s.
Figure 7 compares PDR of homogeneous and heterogeneous deployments.
The Heterogeneous f(Dmax) deployment presents the best performance.
This is because in heterogeneous deployments we reduce packetloss taking advantage of the quasi-orthogonality of SFs and the deployment strategy.
For 100 nodes,
	the gain in terms of PDR for the Heterogeneous f(Dmax) deployment is 300\% compared to the Homogeneous SF12
deployment.
For the Multi-Homogeneous and Heterogeneous Random deployments,
	the gains are respectively 214\% and 200\%.
A reliability of 20\% is reached at
170, 360, 400 and 2600 nodes for the Homogeneous SF12,
	Heterogeneous Random,
	Multi-Homogeneous and Heterogeneous f(Dmax),
	respectively.
Figure 8 compares the network throughput for the homogeneous and heterogeneous deployments in received packets per second.
When the number of nodes increases,
	the throughput saturates.
The Heterogeneous f(Dmax) deployment presents better performance compared with others deployments up to 10000 nodes.
Figure 9 compares PDR as a function of traffic intensity (i.e.
with AP from 1 min to 40 min).
Decreasing traffic intensity reduces the packet loss,
	then PDR increases.
The Heterogeneous f(Dmax) deployment presents better PDR because it takes advantage of the orthogonality of SFs reducing the interference,
	e.g.,
	all nodes set up to SF12 are far from the gateway,
	then the interference to nodes close to the gateway set up to SF6 are reduced.
The Homogeneous SF12 deployment presents lower PDR due to its long packet duration and the low spectral efficiency.
Figure 10 compares the throughput for homogeneous and heterogeneous deployments.
When traffic intensity decreases,
	the throughput decreases.
For high traffic intensity (e.g., 1 min Application Period),
	the Heterogeneous f(Dmax) deployment presents a throughput of 10 packets per second whereas other strategies have a throughput less than 2 packets per second.
Figure 11 compares average power consumption for the homogeneous and heterogeneous deployments for a simulation time of 100*AP.
When increasing traffic intensity,
	power consumption increases.
Taking the Homogeneous SF12 deployment as reference,
	the comparison shows that power consumption for the Heterogeneous f(Dmax) deployment increases smoothly compared with the exponential increasing of the Homogeneous SF12 deployment.

\subsection{Application Period and Packet Duration vs. ERC Regulations}

In Europe,
	ERC regulates access to radio frequency bands.
For most of the sub-bands in the 868 MHz band,
	the duty cycle must be lower than 1\%.
We analyze
this constraint for a 50 Bytes packet configured with different SFs.
Table IV shows the packet duration and the Application Period required to respect the regulation for different values of SF.
For an AP of 1 min,
	the packet duration must be lower than 600 ms to respect the duty cycle of 1\%.
Table IV
shows that the constraint can be satisfied with all SFs except SF11 and SF12.
APs for SF12 and SF11 must be longer than 2.6 min and 1.4 min,
	respectively,
	in order to respect the regulations.

% Final findings
%In summary,
%	results presented in this section show that if the trust coefficient between users is up to 0.8,
%	the vulnerability diffusion process through trust relationship is at its high level of speed.
%This what happens when a new information appears in a communication network and users forward it largely in the network.
%In addition,
%	this work gives a new insight to understand the relationship between trust,
%	reputation,
%	individual vulnerability and social vulnerability in the context of messaging services such as emails.



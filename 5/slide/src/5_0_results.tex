\subsection{Exploitation des résultats}
% Affichage des résultats Graphiques, tableaux
% Interprétation des résultats: Les tendances, les cas extrêmes, les cas favorables, les cas défavorables
% Comparaison avec les résultats des contributions existantes par rapport aux hypothèses considérées
% Justifier le choix de l'approche avec des résultats


\begin{frame}{Résultats}{Comparaison}

	\Columns{0.5}{0.5}{
		Valeurs initiales:
		\Itemize{
			\item générées aléatoirement (distribution normale)
			\item représentent les vulnérabilités individuelles.
			\item couleur foncée = vulnérabilité élevé
		}
		Valeurs finales:
		\Itemize{
			\item obtenu après convergence.
			\item représentent les vulnérabilités sociales.
%					\item ...
		}
	}{
		\TowFigureH{h}{.45}{local_.png}{Vulnérabilité individuelle}{social_.png}{Vulnérabilité Sociale}{graph}{Vulnérabilité individuelle \& sociale}
		% \Table{|c|c|c|}{table1}{Différence entre les vulnérabilités individuelles et sociales en matière de protection de la vie privée}{
		% 	\ User ID & Vul individuel           & Vul sociale \\\hline
		% 	\ 34      & 0.84                     & 0.67                 \\
		% 	\ 67      & 0.12                     & 0.87                 \\
		% 	\ 206     & 0.76                     & 0.33                 \\
		% 	\ 588     & 0.23                     & 0.78                 \\\hline
		% }
	}
	
\end{frame}


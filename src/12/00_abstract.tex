\begin{abstract}

%Definition of the probleme: State why the research is important to a broader non-scientific audience
%Introduce the procedure simply
%Describe the experiment in detail
%Offer a brief overview of the results

TODO

%A Tool for Automatic Assessment and Awareness of Privacy Disclosure
%With increasing frequency,
%	the communication between citizens and institutions occurs via some type of e-mechanism,
%	such as web- sites,
%	emails,
%	and social media.
%In particular,
%	social media platforms are widely being adopted because of their simplicity of use,
%	the large user base,
%	and their high pervasiveness.
%One concern is that users may disclose sensitive information beyond the scope of the interaction with the institutions,
%	not realizing that such data re- mains on these platforms.
%While awareness about basic data (e.g. address,
%	date of birth) protection has risen in the past few years,
%	many users still neglect or fail to realize the amount and significance of the personal information deliberately or involuntarily disclosed on these communication platforms.
%Determining private from non-private data is difficult.
%The goal of this work is to devise a method to detect messages carrying sensitive information from those that not.
%Specifically,
%	we employ machine learning methods to build a privacy decision making tool.
%This work will contribute to develop a privacy protection framework where a client-side privacy awareness mechanism can alert users of the potential private information leakages in their communications.

%In this work we address ...
%in this research work, we introduce …
%since the …, …
%by examining …, our tool identifies … 
%our work is motivated by the potential of improved privacy metrics to identify …
%we evaluated …

%%Estimation of privacy risk through centrality metrics
%Users are not often aware of privacy risks and disclose information in online social networks.
%They do not consider the audience that will have access to it or the risk that the information continues to spread and may reach an unexpected audience.
%Moreover,
%	not all users have the same perception of risk.
%To overcome these issues,
%	we propose a Privacy Risk Score (PRS)


\end{abstract}

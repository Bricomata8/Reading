\subsection{Processus de diffusion}
% Steps of contributions

\begin{frame}{Etape 1: Calcule de la vulnérabilité individuelle}{Méthode}

	\Columns{0.65}{0.35}{
		\Itemize{
			\item Entrée:
			\Itemize{
				\item Vulnérabilité de la machine utilisée:
				\Itemize{
					\item Connexion réseaux (privé (1) ou publique (2))
					\item Type d'architecture: Ethernet, 5G, 4G, Wifi (1:4)
					\item Système d'exploitation (Windows, Unix) (1:2)
					\item Navigateur web (1:10)
				}
				\item Vulnérabilité du compte utilisé
				\Itemize{
					\item Mdp utilisé, mode de récupération des mdp (1:5)
%					\item Certificat valid
					\item Nombre de sessions ouvertes en même temps.(1:nbr)
					\item Mode de chiffrement, signature, version TLS
				}
			}
			\item Sortie:
			\Itemize{
				\item[] \Equation{vulnerability}{Pi = \sum_{i}^{n}\dfrac{{w*V}}{n}}
				\Itemize{
					\item Pi: Vulnérabilité individuelle
					\item w: Poids de chaque vulnérabilité
					\item V: Les vulnérabilités cités au dessus
				}
			}
		}
	}{
		% \Figure{h}{.7}{vulnerability_.png}{Vulnérabilité individuelle}
%		\movie[autostart,label=cells,width=\textwidth,height=0.5\textheight,showcontrols]{}{movie.mp4}
	}

\end{frame}

% Frame step2
\begin{frame}{Etape 2: Calcule de la réputation des utilisateurs}{Méthode}
	\Columns{0.65}{0.35}{
		
		\Itemize{
			\item Entrée:
			\Itemize{
				\item Fréquence d'utilisation de la messagerie.
				\item Horaire, durée des échanges (1:5)
				\item \% des échanges chiffrés, signés, claires (1:3)
				\item Importance des interlocuteurs: Liste favoris (2), noir(1)
				\item Type de données: Texte, images, vidéos, script (1:4)
			}
			\item Méthode:
			\Itemize{
				\item Loi binomiale
			}
			\item Output:
			\Itemize{
				\item[] \Equation{1}{P(reputation) = P(X \ge 1)~=~1~-~(1~-~P(trust))^{n}}
				\item Where,
				\Itemize{
					\item X: Niveau de confiance, X \sim B(n,p)
					\item n:~deg(noeud)
					\item P(X=1): La probabilité de se faire attribué une confiance par un interlocuteur
				}
			}
		}
	}{
		\Figure{h}{.7}{p_trust.png}{Niveau de réputation}
%		\movie[autostart,label=cells,width=\textwidth,height=0.5\textheight,showcontrols]{}{movie.mp4}
	}
\end{frame}




\begin{frame}{Etape 3: Calcule de la vulnérabilité sociale}{Théorie de l'influence sociale de Freidkin}

	\Columns{.65}{.35}{
		\Itemize{
			\item Entrée:
			\Itemize{
				\item $Y^{(1)}$ = Vecteur des vulnérabilités individuelles de N utilisateurs (eq \ref{eq:vulnerability})
				\item \alpha = Le niveau de réputation (d'influence) de chaque utilisateur (eq \ref{eq:1})
				\item M  = Matrice d'adjacence N x N
			}
			\item Modèle:
			\Itemize{
				\item[] \Equation{2}{Y^{(t)}=\alpha MY^{(t-1)} + (1 - \alpha) Y^{(t-1)}}
			}
%			\item[] \Figure{h}{.3}{io.png}
			\item Sortie:
			\Itemize{
				\item  $Y^{(t)}$ = Vecteur des vulnérabilités sociales des N utilisateurs
			}
		}
	}{
		\Figure{h}{.73}{o.png}{Vulnérabilité Sociale}
	}

\end{frame}

\begin{frame}[noframenumbering]
	\movie[autostart,label=cells,width=\textwidth,height=\textheight,showcontrols]{}{movie.mp4}
\end{frame}


\begin{frame}{Etape 3: Calcule de la vulnérabilité sociale}{Théorie de l'influence sociale de Freidkin}

	\Columns{.65}{.35}{
		Propriétés formelles du modèle:
		\Itemize{
			\item Lorsque l'influence d'un utilisateur est élevé, le modèle se réduit aux:
			\Itemize{
				\item vulnérabilités moyennes de ses amis pondérées par leur niveaux de confiances.
				\item[] \EquationT{2}{Y^{(t)}=\red{1 * MY^{(t-1)}} + (1 - 1) Y^{(t-1)}}
				\item[] \centering $Y^{(t)}=  \red{MY^{(t-1)}}$
			}
			\item En absence d'influence, le modèle se réduit à:
			\Itemize{
				\item sa propre vulnérabilité pondérée par le niveau de méfiance de ses amis
				\item[] \EquationT{2}{Y^{(t)}=0 * MY^{(t-1)} + \blue{(1 - 0) Y^{(t-1)}}}
				\item[] \centering $Y^{(t)}= \blue{Y^{(t-1)}}$
			}
		}
	}{
		\Figure{h}{.73}{o.png}{Vulnérabilité sociale}
	}

\end{frame}





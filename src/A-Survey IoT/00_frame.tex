\subsection{Lora}

%\textheight 10.5in
%\setlength{\voffset}{-0.4in}
% \setlength{\hoffset}{-1in}

% \pagestyle{empty}
\begin{landscape}

\begin{figure}
\tiny
\begin{bytefield}[bitwidth=6.3em]{22}
	\y{2}{Preamble}																		&	\y{1}{Sync msg} 			&	\y{1}{PHY Header}  			&	\y{1}{PHDR-CRC} 			&	\y{15}{PHY Payload}																																																																																																																													&	\y{2}{CRC }	                          \\
	\y{1}{Modulation}					&	\y{1}{length}					&	\y{1}{Sync msg} 			&	\y{1}{PHY Header}  			&	\y{1}{PHDR-CRC} 			&	\y{3}{MAC Header}																	&	\y{11}{MAC Payload}																																																																																															& \y{1}{MIC}	&	\y{1}{CRC Type}		&	\y{1}{Polynomial}	\\		
	\y{1}{Modulation}					&	\y{1}{length}					&	\y{1}{Sync msg} 			&	\y{1}{PHY Header}  			&	\y{1}{PHDR-CRC} 			&	\y{1}{MType}	&	\y{1}{RFU}	&	\y{1}{Major}				&	\y{9}{Frame Header }																																																																															&	\y{1}{FPort} 	&	\y{1}{\textbf{Frame Payload}}		& \y{1}{MIC}	&	\y{1}{CRC Type}		&	\y{1}{Polynomial}	\\
	\y{1}{Modulation}					&	\y{1}{length}					&	\y{1}{Sync msg} 			&	\y{1}{PHY Header}  			&	\y{1}{PHDR-CRC} 			&	\y{1}{MType}	&	\y{1}{RFU}	&	\y{1}{Major}				&	\y{2}{Dev Address}										&	\y{5}{FCtrl }																																									&	\y{1}{FCnt} 		&	\y{1}{FOpts} 		&	\y{1}{FPort} 	&	\y{1}{\textbf{Frame Payload}}		& \y{1}{MIC}	&	\y{1}{CRC Type}		&	\y{1}{Polynomial}	\\	
	\y{1}{Modulation}					&	\y{1}{length}					&	\y{1}{Sync msg} 			&	\y{1}{PHY Header}  			&	\y{1}{PHDR-CRC} 			&	\y{1}{MType}	&	\y{1}{RFU}	&	\y{1}{Major}				&	\y{1}{NwkID} 				&	\y{1}{NwkAddr} 	&	\y{1}{ADR} 		&	\y{1}{ADRACKReq} 	&	\y{1}{ACK} 	&	\y{1}{FPending /RFU}	&	\y{1}{FOptsLen}			&	\y{1}{FCnt} 		&	\y{1}{FOpts} 		&	\y{1}{FPort} 	&	\y{1}{\textbf{Frame Payload}}		& \y{1}{MIC}	&	\y{1}{CRC Type}		&	\y{1}{Polynomial}	\\\\
	\bitheader{0-21}                                                  \\
\end{bytefield}
% \caption{LoRaWAN frame format.\cite{augustin-study-2016}}\label{fig:jhjh}
\end{figure}

\scriptsize
\begin{multicols}{2}
\begin{itemize}
	\item[0)] \textbf{Modulation :}
	\begin{itemize}
		\item Lora:  8 Symbols, 0x34 (Sync Word)
		\item FSK:  5 Bytes, 0xC194C1 (Sync Word)
	\end{itemize}
\end{itemize}
\begin{itemize}
	\item \textbf{Length :}
	\item \textbf{Sync msg :}
	\item \textbf{PHY Header :}  It contains:
	\begin{itemize}
		\item The Payload length (Bytes)
		\item \textbf{The Code rate}
		\item Optional 16bit CRC for payload 
	\end{itemize}
	\item \textbf{Phy Header :} CRC  It contains CRC of Physical Layer Header
	\item \textbf{MType :}  is the message type (uplink or a downlink)
	\begin{itemize}
		\item whether or not it is a confirmed message (reqst ack)
		\item 000 	Join Request
		\item 001 	Join Accept
		\item 010 	Unconfirmed Data Up
		\item 011 	Unconfirmed Data Down
		\item 100 	Confirmed Data Up
		\item 101 	Confirmed Data Down
		\item 110 	RFU
		\item 111 	Proprietary
	\end{itemize}
	\item \textbf{RFU :} Reserved for Future Use
	\item \textbf{Major :}  is the LoRaWAN version; currently, only a value of zero is valid
	\begin{itemize}
		\item 00 	LoRaWAN R1
		\item 01-11 	RFU
	\end{itemize}
	\item \textbf{NwkID :} the short address of the device (Network ID): 31th to 25th
	\item \textbf{NwkAddr :} the short address of the device (Network Address): 24th to 0th
	\item \textbf{ADR :}  Network server will change the data rate through appropriate MAC commands
	\begin{itemize}
		\item 1  To change the data rate
		\item 0  No change
	\end{itemize}
	\item \textbf{ADRACKReq :} (Adaptive Data Rate ACK Request): if network doesn't respont in 'ADR-ACK-DELAY' time, end-device switch to next lower data rate.
	\begin{itemize}
		\item 1  if (ADR-ACK-CNT) >= (ADR-ACK-Limit)
		\item 0  otherwise
	\end{itemize}
	\item \textbf{ACK :} (Message Acknowledgement): If end-device is the sender then gateway will send the ACK in next receive window  else if gateway is the sender then end-device will send the ACK in next transmission.
	\begin{itemize}
		\item 1  if confirmed data message
		\item 0  otherwise
	\end{itemize}
	\item \textbf{FPending$\downarrow$ /RFU $\uparrow$ :} (Only in downlink), if gateway has more data pending to be send then it asks end-device to open another receive window ASAP
	\begin{itemize}
		\item 1  to ask for more receive windows
		\item 0  otherwise
	\end{itemize}
	\item \textbf{FOptsLen :} is the length of the FOpts field in bytes   0000 to 1111 
	\item \textbf{FCnt :}  2 type of frame counters 
	\begin{itemize}
		\item FCntUp:  counter for uplink data frame, MAX-FCNT-GAP
		\item FCntDown:  counter for downlink data frame, MAX-FCNY-GAP
	\end{itemize}  
	\item \textbf{FOpts :} is used to piggyback MAC commands on a data message	
	\item \textbf{FPort :}  a multiplexing port field
	\begin{itemize}
		\item 0  the payload contains only MAC commands
		 \item 1 to 223  Application Specific
		 \item 224 \& 225  RFU
	\end{itemize}
    \item \textbf{FRMPayload :} (Frame Payload)  Encrypted (AES, 128 key length) Data                                 
	\item \textbf{MIC :}  is a cryptographic message integrity code
	\begin{itemize}
		\item computed over the fields MHDR, FHDR, FPort and the encrypted FRMPayload.
	\end{itemize}
	\item \textbf{CRC :} (only in uplink), 
	\begin{itemize}
		\item CCITT  $x^{16} + x^{12} + x^{5} + 1$
		\item IBM  $x^{16} + x^{15} + x^{5} + 1$
	\end{itemize}
\end{itemize}

\end{multicols}

	% \item[B] indicates that the network server has additional data to send
	% \item[CID] is the MAC command identifier
	% \item[Args] are the optional arguments of the commands

\end{landscape}

% \appendix


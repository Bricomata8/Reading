\begin{abstract}

% Problem
Most traffic light's control systems in smart cities are wired and have a semi-static behavior.
They are time-based, with pre-configured pattern and expensive cameras.
% Existing solutions
Although traffic lights can communicate wirelessly with incoming vehicles,
	they are less adapted to an urban environment.
If we consider light signs as an Internet of Things (IoT) network,
	one issue is to model thoroughly the change of signs' states and the Quality of Service (QoS) of this network.
In this paper,
	we propose a new architecture of Urban Traffic Light Control based on an IoT network (IoT-UTLC).
The objective is to interconnect both roads' infrastructures and traffic lights through an IoT platform.
We designed our IoT-UTLC by selecting motes and protocols of wireless sensor network (WSN).
Message Queuing Telemetry Transport (MQTT) protocol has been integrated to manage QoS.
It enables lights to adapt remotely to any situation and smoothly interrupt traffic light's classic cycles.
Our experimental results show that the MQTT protocol is efficient when the packets rate exceeds 35\% of traffic flow,
	it reduces traffic delay up to 0.05s at 90\% of congestion.
After verification and validation of our solution using a UPPAAL model checker,
	our system has been prototyped.
Motes' functions have been implemented on Contiki OS and connected through a 6LoWPAN/IEEE 802.15.4 network.
Time-stamping messages have been performed throughout the system to evaluate the MQTT protocol with different QoS levels.
In our experiments,
	we measured the Round-trip delay time (RTT) of messages exchanged between the WSN and IoT Cloud.
The results show that MQTT decreases the RTT when the Cumulative Distributed Function (CDF) of generated messages exceeds 35\%.

\end{abstract}
\section{Experimentation} \label{sec:Experimentation}

For our experimentation we use both real environment (Figure \ref{fig:lora_experimentaion}) and ns-3 simulator with SX1276 Lora module.
However,
	to test the scalability of genetic algorithm with numerous IoT devices in a real environment,
	we use FIT IoT-LAB platform among other platforms presented in \cite{tonneau_how_2015}.
This choice is motivated by the number of devices supported by this platform (up to 2000 nodes).

Figure \ref{fig:gateway} presents the LoRa gateway that we build using a low cost LoRa gateway \cite{lowcostloragateway} on a Raspberry-pi.
Figure \ref{fig:node}  presents one of the two Arduino boards equipped with an antenna that cover the 868 MHz band and with a SX1276 LoRa Transceiver.


\FigureH{h!}{.48}{gateway}{Gateway (Raspberry-pi)}{node}{Sensor node (Arduino)}{lora_experimentaion}{Gateway \& Sensor node}


\section{Discussion} \label{sec:Conclusion}

% Restate the main challenges
The main challenge of this work was to explore the application of genetic algorithm in LoRa transmission parameter selection.
The efficiency of such algorithms is measured by the ability to satisfy each application requirement.
% Restate the main contribution
Our main contribution was to build 3 applications that requires 3 different levels of QoS,
	such as text, sound and image transmission. 
We used a low cost LoRa gateway on a Raspberry-pi with 2 Arduino boards equipped with 2 LoRa Transceivers based on the Semtech SX1276 specification.
% Restate the main findings
To measure the accuracy of applying genetic algorithm in an edge computing we expect to compare our approach with other adaptive data rate solutions.
% Our simulations compare the performance of each configuration selection. 


% homogeneous and heterogeneous deployments as a function of the number of nodes and traffic intensity.
% First,
% 	we have analyzed homogeneous deployments for different SFs from SF6 to SF12.
% Simulations show better performance for the SF6 deployment but reduced cell coverage.
% Second,
% 	we have compared heterogeneous and homogeneous deployments:
% 	the Heterogeneous deployment that selects its LoRa configuration according to its link budget results in the best PDR and throughput,
% 	as well as the lowest average power consumption compared to other deployments,
% 	for a different number of nodes and different traffic intensity.
% The results clearly show the benefits of heterogeneity for large scale network deployments and the need for adaptive SF allocation strategies.
% % Future challenges current bad state
% As a future work,
% 	we plan to validate our appproach by using both simulation and real enviroment testbeds. 
% We plane to study the performance of applying our approach in terms of PDR,
% 	throughput,
% 	and power consumption.
% The module will optimally select the configuration according to the scenario criteria (e.g.,
% 	high data rate,
% 	energy efficiency,
% 	or network congestion) and the radio environment (e.g.,
% 	link budget,
% 	level of interference,
% 	device mobility).

\subsection{Related work}
\begin{frame}{Related work}{Comparison}
	\begin{table}
		\begin{tabular}{c|c|c|c|c}
			Paper & A1 & A2 & A3 & A4 \\\hline
				  &    &    &    & \\\hline
				  &    &    &    & \\\hline
				  &    &    &    & \\\hline
				  &    &    &    & 
		\end{tabular}
		\caption{\label{tab:Tablej} An example table.}
	\end{table}
\end{frame}


\subsubsection{Bandit Algorithm}
\begin{frame}{Multi-Armed-Bandit Algorithm}{Related work}


\begin{itemize}
	\item Arms: K = {1, ... , K}
	\item Decision: T = {1, ... , T}
	\item Reward: $X^{k}_{t}$ with $\mu^{k}_{t}$ = E $[X^{k}_{t}]$
	\begin{itemize}
		\item Best reward: $X^{*}_{t}$ with $\mu^{*}_{t}$ = max $\mu^{k}_{t}$,  k\in  K
	\end{itemize}
\end{itemize}

\end{frame}


\begin{frame}{Binary code analysis: Why?}

       \placetextbox{0.25}{0.8}{
       		\tiny 
       		\mywhiteblackbox{
       			jh
    		}
       }

	\begin{tikzpicture}
		\scriptsize
		% \node[draw,align=left,dashed, text width = 0.2\linewidth] at (3,6) {Criteria c1 fuzzy based control};
		\node[draw,align=left,dashed, text width = 2cm, text height = 2cm] at (3,6) {Multiple criteria decision making};
		\node[draw,align=left,dashed, text width = 2cm, text height = 2cm] at (8,6) {Multiple criteria decision making};
		\node[draw,align=left,dashed, text width = 2cm, text height = 2cm] at (8,3) {Genetic algorithm to determine weights of criteria (w1, ..., wn)};
		\draw[->,black,thick,dashed] (0,0) -- (1,1);
	\end{tikzpicture}

\end{frame}


\subsubsection{Genetic Algorithm}

\begin{frame}{Genetic Algorithm}{Related work \cite{alkhawlani_access_2008a}}

\begin{itemize}
	\item Heterogeneous wireless network: (RAT 1 ,RAT 2 ,...,RAT n)
	\item Criteria up to i (c 1 ,c 2 ,...,c i ) the operators, the applications, and the network conditions.
	\item 
	\item The different sets of scores (d 1 , d 2 ,...,d i ) are sent to the MCDM in the second component.
	\item GA component assigns a suitable weight (w 1 ,w 2 ,...,w i )
\end{itemize}

\end{frame}

\begin{frame}{Genetic Algorithm}{Related work}

\Itemize{
	\item 
	\item S = {SF12, BW125, 4/8, 17 dBm}
	\item Input: 
	\Itemize{
		\item Problem: f(x) = {max($x^{2}$), x \in [0,32]}
		\Itemize{
			\item $x_{1}: 01101_{b}$ 
			\item $x_{2}: 11000_{b}$
			\item $x_{3}: 01000_{b}$
			\item $x_{4}: 10011_{b}$
		}
	}

	\item Method: Genetic algorithm
	\Itemize{
		\item Generate a set of random possible solution
		\item Test each solution and see how good it is (ranking)
		\Itemize{
			\item Remove some bad solutions
			\item Duplicate some good solutions
			\item Make small changes to some of them (Crossover, Mutation)
		}
	}

	\item Output:
	\Itemize{
			\item $x_{1}$: 01101  (169)  (14.4)
			\item $x_{2}$: 11000  (576)  (49.2)
			\item $x_{3}$: 01000  (64 )  (5.5)
			\item $x_{4}$: 10011  (361)  (30.9)
	}
}
\end{frame}



\subsubsection{Marcov chain}

\begin{frame}{Marcov chain}{Related work}

\begin{equation}
V(s, \pi)=\mathbb{E}_{s}^{\pi}\left(\sum_{k=0}^{\mathrm{inf}} \gamma^{k} \cdot r\left(s_{k}, a_{k}\right)\right), s \in \mathbb{S}
\end{equation}

\begin{equation}
r\left(s_{k}, a_{k}\right)=G_{k} \cdot P R R\left(a_{k}\right)
\end{equation}

\begin{equation}
\pi^{*}=\arg \max _{\pi} V(s, \pi)
\end{equation}


\stamp{HGHGJ}
% \begin{tikzpicture}[remember picture, overlay]
% 	\node[draw, rotate=30] at (25em, 7ex) {\color{red!90}\huge\bfseries APPROVED};
% \end{tikzpicture}

\begin{equation}
PRR=(1-BER)^{L}
\end{equation}

\begin{equation}
BER=10^{\alpha e^{\beta SNR}}
\end{equation}

\end{frame}

\begin{frame}{Marcov chain}{Related work}
\Figure{h}{1}{markov}{}
\end{frame}


\subsubsection{Game theory}
\begin{frame}{Game theory}{Related work}
\Itemize{
	\item Players: $K = \{1, ... , K\}$
	\item Strategies: $S =S_{1} \times \ldots \times S_{K}$
	\Itemize{
		\item $S_{k}$ is the strategy set of the $k^{th}$ player.
	}
	\item Rewards: $u_{k} : S \longrightarrow R_{+}$ and is denoted by $r_{k} (s_{k} , s_{-k})$
	\Itemize{
		\item $s_{-k}=\left(s_{1}, \dots, s_{k-1}, s_{k+1}, \ldots, s_{K}\right) \in S_{1} \times \ldots \times S_{k-1} \times S_{k+1} \times \ldots \times S_{K}$
	}
}
 \end{frame}


%\note{
%	\begin{itemize}
%		\item Contenu:
%		\begin{itemize}
%			\item Tableau comparatif (articles connexes/avantages et désavantages)
%			\item Les limites de l’existant
%			\item Notre travaille traite le meme x que les travaux précidants mais utilise y au lieu de z (xy/xz)
%		\end{itemize}
%		\item Procedure:
%		\begin{itemize}
%			\item Lecture en largeur
%			\begin{itemize}
%				\item Lecture de beaucoup de papiers connexes
%				\item Comprendre le domaine
%				\item Comprendre les travaux existants
%				\item Sélection des travaux intéressants
%			\end{itemize}
%			\item Lecture en profondeur
%			\begin{itemize}
%				\item Lecture et analyse des travaux sélectionnés
%				\item Descendre jusqu’au détail du détail
%			%					\begin{itemize}
%			%						\item Poser toujours la question pourquoi?
%			%						\item Être capable d’implémenter de suite l’approche.
%			%					\end{itemize}
%				\end{itemize}
%				\item Situer le travail par rapport à l’existant sur la base de La problématique traitée
%				\begin{itemize}
%					\item Les critiques faites sur l’existant
%					\item Les hypothèses du travail courant
%					\item Les objectifs initiales du travail
%					\item Les résultats théoriques et expérimentales obtenus
%				\end{itemize}
%			\end{itemize}
%			\item Article:
%			\begin{itemize}
%				\item Est-ce que le problème est toujours intéressant ?
%				\item Est-ce qu'on peux traiter le problème d'une autre manière ?
%				\item Est-ce que les hypothèses sont réalistes ?
%				\item Est-ce que le travail est applicable dans le contexte actuel ?
%				\item Est-ce que tous les aspects du problème ont été traités ?
%				\item Existe-t-il d’autres manières pour le résoudre ?
%			\end{itemize}
%		\end{itemize}
%	}

\begin{frame}{Related work}{Comparison}
	\begin{table}
		\begin{tabular}{c|c|c|c|c}
			Paper & A1 & A2 & A3 & A4 \\\hline
				  &    &    &    & \\\hline
				  &    &    &    & \\\hline
				  &    &    &    & \\\hline
				  &    &    &    & 
		\end{tabular}
		\caption{\label{tab:Tableju} An example table.}
	\end{table}
\end{frame}

%\note{
%	\begin{itemize}
%		\item Conseils:
%		\begin{itemize}
%			\item Qu'est ce qui réuni et divise tous c'est travaux
%			\item Ce chapitre ne doit pas être une simple revue de la bibliographie
%			\item Présentation des travaux antérieurs et connexes
%			\begin{itemize}
%				\item Choisir les travaux reliés
%				\item Critique des travaux antérieurs
%				\item Description du lien entre le sujet traité dans le mémoire et les travaux antérieurs
%				\item le lien: qqch en commun (méthode, approche, outil ...)
%				\item Cibler les critiques où le candidat apporte des contributions
%				\item Résumer l’analyse de ces travaux dans un tableau récapitulatif
%				\item Il faut analyser les travaux pour proposer une contribution
%			%					\begin{itemize}
%			%						\item Classification des travaux
%			%						\item Avantages \& inconvénients
%			%						\item Contextes d’utilisation
%			%					\end{itemize}
%				\item Formulation du problème théorique
%				\item Présentation des hypothèses explicatives
%				\item Suite à la lecture de ce chapitre, le lecteur doit avoir compris la motivation pour le choix du sujet et son importance
%			\end{itemize}
%		\end{itemize}
%	\end{itemize}
%}


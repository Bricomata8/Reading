\subsection{Problem statement}


\begin{frame}{Problem statement}{Introduction \cite{dimartino_internet_2018} ?}

\ac{BW}  \ac{SF} \ac{CR} \ac{Tx} \ac{RS} \ac{SNR} \ac{DR} ,\ac{AT}, \ac{PktL}

\begin{table}[h!]
% \scriptsize
	\begin{tabular}{l|m{1mm}l|l|l}
	\textbf{Setting}& \multicolumn{2}{l|}{\textbf{Values}} 				    & \textbf{Rewards}		   & \textbf{Costs} 					    \\\hline
	\ac{BW}         & $7.8 $ 	& \ding{224} $500 kHz$  								& \ac{DR}          		   & \ac{RS}, \blue{Range} 			  \\\hline
	\ac{SF}         & $2^{6}$ 	& \ding{224} $2^{12}$ 									& \ac{RS}, \blue{Range}    & \ac{DR}, \ac{SNR}, \ac{PktL}, \ac{Tx}    \\\hline
	\ac{CR}         & $4/5$ 	& \ding{224} $4/8$    								  	& Resilience 			   &  \ac{PktL}, \ac{Tx}, \ac{AT} 				\\\hline
	\ac{Tx}         & $-4$ 		& \ding{224} $20 dBm$    								& \ac{SNR} 				   & \ac{Tx}  								\\\hline
	\end{tabular}
\caption{\label{tab:} \cite{cattani_experimental_2017}}
\end{table}

\end{frame}



%	\note{
%		\begin{itemize}
%			\item Objectifs
%			\begin{itemize}
%				\item Prouver la faisabilité de l'approche
%				\item Valider certains choix dans l’approche
%				\item Configurer l’approche
%				\item Tester l’approche dans des conditions extrêmes
%				\item Comparer l’approche par rapport à l’existant
%			\end{itemize}
%			\item Etapes
%			\begin{itemize}
%				\item Établir un plan d’expérimentation
%				\item Préparer un jeux de données
%				\item Préparer des scénarii
%				\item Dérouler les scénarii
%			\end{itemize}
%			\item Conseils:
%			\begin{itemize}
%				\item Utiliser des benchmarks de préférence
%				\item Montrer par rapport à la problématique les situations où l’approche est intéressante et là où elle ne l’est pas
%				\item Présenter les résultats sous forme de graphes
%				\item Bien expliquer et analyser les résultats
%			\end{itemize}
%		\end{itemize}
%	}



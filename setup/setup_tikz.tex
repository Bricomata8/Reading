% \mode<presentation>{}
% \usepackage[tel={60~00},fax={60~00},lab={LSL}]{styles/beamerthemeCeaList2012Red}
\usepackage{etex}  %% pour regler conflits tikz et pgfplot
\usepackage{bigstrut}
\usepackage{mathtools}
\usepackage{tcolorbox}
\usepackage{eso-pic}
\usepackage{listings}
\usepackage{pgfplots}
\usepackage{array}
\usepackage{stmaryrd}
\usepackage{amsmath}
% %\usepackage{styles/gantt}
\usepackage{lscape}
% \usepackage{bookmark}
\usepackage{etoolbox}
\usepackage{xcolor,colortbl}
\usepackage{textcomp}
% \usepackage{fdsymbol}
\usepackage{bbding}
\usepackage[makeroom]{cancel}
\usepackage{tikz}




\usepackage{makecell}
\usepackage{amssymb}
\usepackage{tikz,nicefrac,amsmath,pifont}
\usepackage{tikz-timing}
\usetikzlibrary{mindmap,trees}
\usepackage{pgfplots}
\usepackage{pgfplotstable}
\usetikzlibrary{calc,3d,arrows,arrows.meta,snakes,backgrounds,datavisualization,patterns,matrix,shapes,fit,calc,shadows,plotmarks,positioning,decorations,fadings}
\usepackage{wasysym}
\usepackage{stmaryrd}

\newlength{\imagewidth}
\newlength{\imagescale}
\newcommand{\itab}[1]{\hspace{0em}\rlap{#1}}
\newcommand{\tab}[1]{\hspace{.09\textwidth}\rlap{#1}}

\DeclareMathAlphabet{\mathcal}{OMS}{cmsy}{m}{n} 
\DeclareFontFamily{U}{MnSymbolA}{}
\DeclareFontShape{U}{MnSymbolA}{m}{n}{
  <-6> MnSymbolA5
  <6-7> MnSymbolA6
  <7-8> MnSymbolA7
  <8-9> MnSymbolA8
  <9-10> MnSymbolA9
  <10-12> MnSymbolA10
  <12-> MnSymbolA12}{}
\DeclareSymbolFont{MnSyA}{U}{MnSymbolA}{m}{n}
\DeclareMathSymbol{\downlsquigarrow}{\mathrel}{MnSyA}{163}
% \def\res{../..}
 \pgfplotsset{
 	table/search path={\res/data/},
 }

\newcommand{\Node}[4]{
node[#1] (#2) {#3} [#4]
}
\newcommand{\Child}[2]{
child[#1] {#2}
}

\newcommand*{\info}[4][16.3]{%
  \node [ annotation, #3, scale=0.65, text width = #1em,
          inner sep = 2mm ] at (#2) {%
  \list{$\bullet$}{\topsep=0pt\itemsep=0pt\parsep=0pt
    \parskip=0pt\labelwidth=8pt\leftmargin=8pt
    \itemindent=0pt\labelsep=2pt}%
    #4
  \endlist
  };
}














\usetikzlibrary{arrows.meta}
\tikzset{>={Latex[width = 2mm,length=2mm]},
            base/.style = {rectangle, rounded corners, draw=black, minimum width=4cm, minimum height=.6cm, text centered, font=\sffamily},
  activityStarts/.style = {base, fill=violet, double=white, text=white},
       startstop/.style = {base, fill=red!30},
    selection/.style = {base, fill=white!30},
         process/.style = {base, minimum width=2.5cm, fill=orange!15, font=\ttfamily},
}

\tikzset{
  boxStyle/.style={draw=blue!80, fill=blue!9, very thick,
    rectangle, rounded corners=3mm, inner sep=10pt, inner ysep=15pt},
  boxTitleStyle/.style={fill=blue!80, rectangle, rounded corners=2mm,
    text=white, inner sep=10pt, inner ysep=7pt, left=10pt},
}
\newenvironment{textBox}[1]{%
  \def\title{#1}%
  \begin{tikzpicture}
    \node [boxStyle] (box)
    \bgroup\minipage{0.5\textwidth}%
}{
    \endminipage%
    \egroup;
%    \node[boxTitleStyle] at (box.north east) {\title};
  \end{tikzpicture}%
}
\newenvironment{whiteblock} {
  \setbeamertemplate{blocks}[rounded]
  \setbeamercolor{block title}{fg=white,bg=black!85!black}
  \setbeamercolor{block body}{fg=black,bg=white}
  \begin{block}
  }{\end{block}}
\newenvironment{greenblock} {
  \setbeamertemplate{blocks}[rounded][shadow=true]
  \setbeamercolor{block title}{fg=white,bg=green!85!black}
  \setbeamercolor{block body}{fg=black,bg=green!30!white}
  \begin{block}
  }{\end{block}}
\newenvironment{orangeblock} {
  \setbeamertemplate{blocks}[rounded][shadow=true]
  \setbeamercolor{block title}{fg=white,bg=orange!85!black}
  \setbeamercolor{block body}{fg=black,bg=orange!30!white}
  \begin{block}
  }{\end{block}}
\newenvironment{blueblock} {
  \setbeamertemplate{blocks}[rounded][shadow=true]
  \setbeamercolor{block title}{fg=white,bg=blue!85!black}
  \setbeamercolor{block body}{fg=black,bg=blue!30!white}
  \begin{block}
  }{\end{block}}
\newenvironment{redblock} {
  \setbeamertemplate{blocks}[rounded][shadow=true]
  \setbeamercolor{block title}{fg=white,bg=red!85!black}
  \setbeamercolor{block body}{fg=black,bg=red!30!white}
  \begin{block}
  }{\end{block}}
\newenvironment{grayblock} {
  \setbeamertemplate{blocks}[rounded][shadow=true]
  \setbeamercolor{block title}{fg=white,bg=black!85!white}
  \setbeamercolor{block body}{fg=black,bg=black!30!white}
  \begin{block}
  }{\end{block}}
\newenvironment{changemargin}[2]{%
  \begin{list}{}{%
      \setlength{\topsep}{0pt}%
      \setlength{\leftmargin}{#1}%
      \setlength{\rightmargin}{#2}%
      \setlength{\listparindent}{\parindent}%
      \setlength{\itemindent}{\parindent}%
      \setlength{\parsep}{\parskip}%
    }%
  \item[]}{\end{list}}


\xdefinecolor{mongris}{rgb}{0.5,0.5,0.5}
\definecolor{LightCyan}{rgb}{0.9,1,1}
\definecolor{DarkCyan}{rgb}{0.7,1,1}
\definecolor{LightGray}{gray}{0.5}
\definecolor{DarkGray}{gray}{0.85}
\definecolor{LightRed}{rgb}{1, 0.9, 0.9}
\definecolor{DarkRed}{rgb}{1, 0.8, 0.8}
\definecolor{mygreen}{rgb}{0,0.6,0}
\definecolor{mygray}{rgb}{0.5,0.5,0.5}
\definecolor{mymauve}{rgb}{0.58,0,0.82}


\newcommand{\monbleu}[1]{{\color{blue}{#1}}}
\newcommand{\rouge}[1]{{\color{red}{#1}}}
\newcommand{\mygray}[1]{\textcolor{mongris}{#1}}
\newcommand{\citat}[1]{{\tiny{\color{green!40!black}{[#1]}}}}
\newcommand{\mynote}[1]{{\scriptsize{\color{red!40!black}{[#1]}}}}

\newcommand{\vundef}{{\color{red}{\bot_V}}}
\newcommand{\vblackundef}{{\bot_V}}
\newcommand{\condundef}{{\color{red}{(U)}}}
\newcommand{\yundefined}{{\color{red}{(U)}}}
\newcommand{\outofmem}{{\color{red}{out of mem}}}
\newcommand{\outoftime}{{\color{red}{$>$ 10min}}}
\newcommand{\env}{\mathcal{E}nv}
\newcommand{\fullsymb}{logicSymb}
\newcommand{\varfullsymb}{logicSymb^\#}
\newcommand{\memmodel}{\varphi}
\newcommand{\condenv}{\varrho}
\newcommand{\varcondenv}{\varrho^\#}
\newcommand{\simplify}{simplify}
\newcommand{\valsimplify}{simplify^+}
\newcommand{\resconstraint}{\tau}
\newcommand{\nosimplbot}{NoSimp_\bot}
\newcommand{\nosimpltop}{NoSimp_\top}
\newcommand{\symb}{s}
\newcommand{\varsymb}{s^\#}
\newcommand{\mybadmark}{{\color{red}{\bf $\times$}}}
\newcommand{\mycheckmark}{{\color{green}{\bf $\checkmark$}}}
\newcommand{\mywarnmark}{{\color{orange}{\bf $\bullet$}}}
\newcommand{\mymaybemark}{{\color{orange}{\bf $\checkmark$}}}
\newcommand{\rewritingmode}{{\tt rewriting}}
\newcommand{\logicmode}{{\tt logic}}
\newcommand{\hybridmode}{{\tt hybrid}}
\newcommand{\vo}{{\mathcal{O}}}
\newcommand{\vr}{{\mathcal{R}}}
\newcommand{\vm}{{\mathcal{M}}}
\newcommand{\vw}{{\mathcal{W}}}
\newcommand{\vh}{{\mathcal{H}}}
\newcommand{\vx}{{\mathcal{X}}}
\newcommand{\mybadite}{{\color{red}{${{\bf \times}}$}}}
\newcommand{\mybadstore}{{\color{red}{${{\bf \times}}$}}}
\newcommand{\mybadload}{{\color{red}{${{\bf \times}}$}}}
\newcommand{\mybadjmp}{{\color{red}{${{\bf \times}}$}}}
\newcommand{\mybadtout}{{\color{red}{${{\bf \geq 10m}}$}}}
\newcommand{\smalltexttt}[1]{\texttt{\scriptsize #1}}
\newcommand{\tinytexttt}[1]{\texttt{\tiny #1}}


\makeatletter

\newcommand{\mywhiteblackbox}[1]{
  \setbox0=\hbox{#1}
  \setlength{\@tempdima}{\dimexpr\wd0+13pt}
  \begin{tcolorbox}[
    colback=white,
    colframe=black,
    boxrule=0.5pt,
    arc=4pt,
    left=6pt,
    right=6pt,
    top=6pt,
    bottom=6pt,
    boxsep=0pt,
    width=\@tempdima]
    #1
  \end{tcolorbox}
}
\newcommand{\mybluebox}[1]{
  \setbox0=\hbox{#1}
  \setlength{\@tempdima}{\dimexpr\wd0+13pt}
  \begin{tcolorbox}[
    colback=blue!30!white,
    colframe=black!25!white,
    boxrule=0.5pt,
    arc=4pt,
    left=6pt,
    right=6pt,
    top=6pt,
    bottom=6pt,
    boxsep=0pt,
    width=\@tempdima]
    #1
  \end{tcolorbox}
}
\newcommand{\myorangebox}[1]{
  \setbox0=\hbox{#1}
  \setlength{\@tempdima}{\dimexpr\wd0+13pt}
  \begin{tcolorbox}[
    colback=orange!30!white,
    colframe=black!25!white,
    boxrule=0.5pt,
    arc=4pt,
    left=6pt,
    right=6pt,
    top=6pt,
    bottom=6pt,
    boxsep=0pt,
    width=\@tempdima]
    #1
  \end{tcolorbox}
}
\newcommand{\mygreenbox}[1]{
  \setbox0=\hbox{#1}
  \setlength{\@tempdima}{\dimexpr\wd0+13pt}
  \begin{tcolorbox}[
    colback=green!40!white,
    colframe=black!25!white,
    boxrule=0.5pt,
    arc=4pt,
    left=6pt,
    right=1pt,
    top=6pt,
    bottom=6pt,
    boxsep=0pt,
    width=\@tempdima]
    #1
  \end{tcolorbox}
}
\newcommand{\myredbox}[1]{
  \setbox0=\hbox{#1}
  \setlength{\@tempdima}{\dimexpr\wd0+13pt}
  \begin{tcolorbox}[
    colback=red!40!white,
    colframe=black!25!white,
    boxrule=0.5pt,
    arc=4pt,
    left=6pt,
    right=1pt,
    top=6pt,
    bottom=6pt,
    boxsep=0pt,
    width=\@tempdima]
    #1
  \end{tcolorbox}
}
\newcommand{\placetextbox}[3]{
  % \placetextbox{<horizontal pos>}{<vertical pos>}{<stuff>}
  \AddToShipoutPictureFG*{
    \put(\LenToUnit{#1\paperwidth},\LenToUnit{#2\paperheight})
    {\vtop{{\null}\makebox[0pt][c]{#3}}}%
  }
}

\usetikzlibrary{fit}
\usetikzlibrary{calc}
\usetikzlibrary{mindmap}
\usetikzlibrary{decorations.text}
\pgfplotsset{compat=1.7}
\usetikzlibrary{positioning, shapes}
\usetikzlibrary{arrows}
\usetikzlibrary{shapes.geometric}
\usetikzlibrary{shapes.arrows}
\usetikzlibrary{backgrounds}
\usetikzlibrary{decorations.pathreplacing}
\usetikzlibrary{hobby}

\tikzstyle{every picture}+=[remember picture]
\tikzset{state/.style={
    rectangle,
    rounded corners,
    draw=black,
    minimum height=2em,
    inner sep=2pt,
    text centered}}
\tikzset{mynode/.style={fill,circle,inner sep=2pt,outer sep=0pt}}
\tikzset{mynodesol/.style={fill,rectangle,inner sep=2pt,outer sep=0pt}}

% \title{\bigskip \bigskip \textbf{Binary-level static analysis}}
% \author{{\bf Adel Djoudi}}
% \date{02/12/2016 \newline \hspace{-1cm}{PhD defense}}
% \titlegraphic{\includegraphics[width=4cm,height=2cm]{CEA_LIST.jpg}}
% \institute{}



\newcommand{\bubblethis}[2]{
  \tikz[remember picture,baseline]{\node[anchor=base,inner sep=0,outer sep=0]%
    (#1) {\underline{#1}};\node[overlay,cloud callout,callout relative
    pointer={(0.2cm,-0.7cm)},%
    aspect=2.5,fill=yellow!90] at ($(#1.north)+(-0.5cm,1.6cm)$) {#2};}%
}%

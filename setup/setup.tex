\usepackage{multirow}
\usepackage{tabulary}
	%\renewcommand{\arraystretch}{1.5}

\usepackage{eurosym} % euro s

\newcommand\Mark[1]{\textsuperscript#1}
\def\res{../res}

%\begin{figure}
%\begin{center}
%\begin{tabular}{cc}
%\resizebox{60mm}{!}{\includegraphics{test1.eps}} &
%\resizebox{60mm}{!}{\includegraphics{test2.eps}} \\
%\resizebox{60mm}{!}{\includegraphics{test3.eps}} &
%\resizebox{60mm}{!}{\includegraphics{test4.eps}} \\
%\end{tabular}
%\caption{This is sample figures.}
%\label{test4}
%\end{center}
%\end{figure}

\usepackage{pgfplots}
\usepackage{filecontents}
\pgfplotsset{compat=newest}
\usepgfplotslibrary{patchplots}
\usepackage{tikz}
\usepackage{pgfplots}
\pgfplotsset{
	table/search path={\res/data/},
}
\usetikzlibrary{backgrounds}
\usetikzlibrary{graphs, graphs.standard, graphdrawing}



\newcommand{\shaded}[1]{\textcolor{gray!30}{#1}}
\usepackage{pifont}
	\newcommand{\ok}{\textcolor{green}{\ding{51}}}
	\newcommand{\ko}{\textcolor{red}{\ding{55}}}

\usepackage{xcolor}
\definecolor{fondtitre}{rgb}{0.20,0.43,0.09}  % vert fonce
\definecolor{coultitre}{rgb}{0.41,0.05,0.05}  % marron
\definecolor{green}{rgb}{0.2,1,0.2}
\definecolor{beamer@blendedblue}{HTML}{404bbf}
\definecolor{violet}{HTML}{5c3566}

\def\blue#1{\textcolor{beamer@blendedblue}{#1}}
\def\violet#1{\textcolor{violet}{#1}}
\def\yellow#1{\textcolor{yellow}{#1}}
\def\red#1{\textcolor{red}{#1}}
\def\green#1{\textcolor{green}{#1}}
\def\black#1{\textcolor{black}{#1}}


\usepackage{color, colortbl}
	\definecolor{mygray}{rgb}{0.5,0.5,0.5}
	\definecolor{mygreen}{rgb}{0,0.6,0}
	\definecolor{gray75}{gray}{0.75}
	\definecolor{marshalcolor}{rgb}{0.93,0.33,0.93} % MARSHAL blue
	\definecolor{draftcolor}{rgb}{0.86,0.86,0.86} % MARSHAL draft color
	\definecolor{darkraspberry}{rgb}{0.53, 0.15, 0.34}


\usepackage{hyperref}
\hypersetup{
%	pdftex,
%	pdfpagemode=FullScreen,
    pdftoolbar=true,        % show Acrobat’s toolbar?
%    pdfborder={0 0 0},
    pdfmenubar=true,        % show Acrobat’s menu?
    pdffitwindow=true,     % window fit to page when opened
    pdfstartview={FitH},    % fits the width of the page to the window
%    pdftitle={title},    % title
%    pdfauthor={Salvatore Mazzarino},     % author
%    pdfsubject={Subject},   % subject of the document
%    pdfcreator={Salvatore Mazzarino},   % creator of the document
%    pdfproducer={Salvatore Mazzarino}, % producer of the document
%    pdfkeywords={Green Networking} {Mobile Cloud} {Network Coding} {Energy}, % list of keywords
    pdfnewwindow=true,      % links in new window
    colorlinks=true,       % false: boxed links; true: colored links
    linkcolor=beamer@blendedblue,          % color of internal links (change box color with linkbordercolor)
    citecolor=beamer@blendedblue,        % color of links to bibliography
    filecolor=beamer@blendedblue,      % color of file links
    urlcolor=beamer@blendedblue           % color of external links
}
    
%    hyperfootnotes=true,
%	breaklinks=true,
%	bookmarksopen=true,
%	backref=page

% sudo apt-get install biber
\usepackage[sorting=none,backref=true,backend=biber, doi=false,url=false,isbn=false, style=numeric]{biblatex}%sorting=nyt, sorting=rasha
%\usepackage[sorting=none,backref=true,backend=biber,sorting=rasha, doi=false,url=false,isbn=false, citetracker,pagetracker=page]{biblatex}%sorting=nyt
	%\setbeamertemplate{bibliography item}{\insertbiblabel}
	%Comme back from citation backref=true
	\addbibresource{/home/aghiles/Aghiles/Redaction/setup/lib/main.bib}
	\addbibresource{/home/aghiles/Aghiles/Redaction/setup/lib/main1.bib}
	\addbibresource{/home/aghiles/Aghiles/Redaction/setup/lib/privacy.bib}
	\def\bibfont{\small}
	\DefineBibliographyStrings{english}{%
		backrefpage = {p.},% originally "cited on page"
		backrefpages = {p.},% originally "cited on pages"
	}
	% Define new format that applies a hypertext reference
	\DeclareFieldFormat{linked}{%
		\ifboolexpr{ test {\ifhyperref} and not test {\ifentrytype{online}}}{
			\iffieldundef{file}{
				\iffieldundef{url}{#1}{\href{run:\thefield{url}}{#1}}
			}{	\href{run:\thefield{file}}{#1}
%				\StrCount{\thefield{file}}{:}[\nbmatch]%
%				\StrCut[\nbmatch]{\thefield{file}}{:}\strfirst\strsecond
%				\StrCount{\strfirst}{:}[\nbmatch]%
%				\StrCut[\nbmatch]{\strfirst}{:}\strfirst\strsecond
%				\href[pdfnewwindow]{run:\strsecond}{#1}
			}
		}{#1}
	}
	% Based on generic definition from biblatex.def
	\renewbibmacro{title}{
		\ifboolexpr{ test {\iffieldundef{title}} }{}{
			\printtext[title]{
				\printtext[linked]{\printfield[titlecase]{title}}
			}
			\newunit
		}
		\printfield{titleaddon}
	}
	%
	\DeclareSourcemap{
	\maps[datatype=bibtex]{
			\map[overwrite=true]{
				\step[fieldsource=groups, fieldtarget=keywords]
			}
		}
	}
	%
%	\DeclareSortingTemplate{rasha}{
%		\sort[direction=ascending]{
%			\field{year}}
%		\sort{\field{presort}}
%	}
	
	\newcommand{\aghiles}[1]{\printbibliography[heading=subbibliography, keyword={#1}, title={#1}]}

%Todo
\newcounter{todo}
\usepackage{tcolorbox}
	\newtcbox{\mytodobox}{colback=white,colframe=white!75!white}
\newcommand\todo[1]{
	\refstepcounter{todo}
	\mytodobox{\hypertarget{todo\thetodo}{#1}}
	\addcontentsline{tod}{subsection}{\protect\hyperlink{todo\thetodo}{\thetodo~#1}\par} }
\makeatletter
\newcommand\listoftodos{
	\@starttoc{tod}}
\makeatother

\usepackage{graphicx}
	\graphicspath{ {\res/} {\res/diagram/} {\res/icon/} {\res/plot/} {\res/stat/} {\res/tikz/} {\res/tmp/}}

\newcommand{\towFigure}[5]{
	\medskip
	\begin{figure}
		\includegraphics[width=#1\columnwidth]{#2} \\
		\includegraphics[width=#1\columnwidth]{#3}
		\caption{#5.}\label{fig:#4}
	\end{figure}
	\medskip
}

\newcommand{\towFigureT}[5]{
	\medskip
	\begin{figure}
		\includegraphics[width=#1\columnwidth]{#2} \\
		\includegraphics[width=#1\columnwidth]{#3}
		\caption*{\blue{Figure} \ref{fig:#4}: #5.}
%		\caption{#5.}\label{fig:#4}
	\end{figure}
	\medskip
}

\newcommand{\Figure}[4]{
	\begin{figure}[#1]
	\centering
	\includegraphics[width=#2\columnwidth]{#3}
	\caption{#4.}\label{fig:#3}
	\end{figure}
}

\newcommand{\Tickz}[4]{
	\medskip
	\begin{figure}[#1]
			\centering
			\begin{tikzpicture}[scale=#2,line width=1pt]
				\input{\res/tikz/#3}
			\end{tikzpicture}
	\caption{#4.}\label{fig:#3}
	\end{figure}
	\medskip
}

\newcommand{\FigureS}[4]{
	\medskip
	\begin{figure}[#1]
	\centering
	\includegraphics[width=#2\columnwidth]{#3}
	\caption*{\blue{Figure \ref{fig:#3}:} #4.}
	\end{figure}
	\medskip
}

\newcommand{\FigureT}[3]{
	\medskip
	\begin{figure}[#1]
	\centering
	\includegraphics[width=#2\columnwidth]{#3}
	\end{figure}
	\medskip
}

\usepackage{subcaption}
	\captionsetup{justification=centering}
%	\captionsetup{labelfont=it,textfont={bf,it},justification=centering}
	
\renewcommand{\thesubfigure}{\alph{subfigure}}
\renewcommand{\thefigure}{\arabic{figure}}

\newcommand{\FigureH}[8]{
%	\medskip
	\begin{figure}
		\centering
		\begin{subfigure}[#1]{#2\columnwidth}
			\centering
			\includegraphics[width=\columnwidth]{#3}
			\caption{#4.}\label{fig:#3}
		\end{subfigure}
		~ % \quad, \qquad, \hfill
		\begin{subfigure}[#1]{#2\columnwidth}
			\centering
			\includegraphics[width=\columnwidth]{#5}
			\caption{#6.}\label{fig:#5}
		\end{subfigure}
		
		\caption{#8.}\label{fig:#7}
	\end{figure}
%	\medskip
}

\newcommand{\FigureV}[8]{
%	\medskip
	\begin{figure}
		\centering
		\begin{subfigure}[#1]{\columnwidth}
			\centering
			\includegraphics[width=#2\columnwidth]{#3}
			\caption{#4.}\label{fig:#3}
		\end{subfigure}
		~ % \quad, \qquad, \hfill
		\begin{subfigure}[#1]{\columnwidth}
			\centering
			\includegraphics[width=#2\columnwidth]{#5}
			\caption{#6.}\label{fig:#5}
		\end{subfigure}
		
		\caption{#8.}\label{fig:#7}
	\end{figure}
%	\medskip
}

\newcommand{\TickzH}[9]{
%	\medskip
	\begin{figure}
	\begin{center}
		\begin{subfigure}[#1]{#2\columnwidth}
			\centering
			\begin{tikzpicture}[scale=#9,line width=1pt]
				\input{\res/tikz/#3}
			\end{tikzpicture}
			\caption{#4.}\label{fig:#3}
		\end{subfigure}
		~ % \quad, \qquad, \hfill
		\begin{subfigure}[#1]{#2\columnwidth}
			\centering
			\begin{tikzpicture}[scale=#9,line width=1pt]
				\input{\res/tikz/#5}
			\end{tikzpicture}
			\caption{#6.}\label{fig:#5}
		\end{subfigure}
		\caption{#8.}\label{fig:#7}
	\end{center}
	\end{figure}
%	\medskip
}

\newcommand{\TickzV}[8]{
%	\medskip
	\begin{figure}
		\centering
		\begin{subfigure}[#1]{\columnwidth}
			\centering
			\begin{tikzpicture}[scale=#2,line width=1pt]
				\input{\res/tikz/#3}
			\end{tikzpicture}
			\caption{#4.}\label{fig:#3}
		\end{subfigure}
		~ % \quad, \qquad, \hfill
		\begin{subfigure}[#1]{\columnwidth}
			\centering
			\begin{tikzpicture}[scale=#2,line width=1pt]
				\input{\res/tikz/#5}
			\end{tikzpicture}
			\caption{#6.}\label{fig:#5}
		\end{subfigure}
		
		\caption{#8.}\label{fig:#7}
	\end{figure}
%	\medskip
}


\newcommand{\Equation}[2]{
	\begin{equation}\label{eq:#1}
		#2
	\end{equation}
}

\newcommand{\EquationT}[2]{
	\begin{equation*}\tag{\ref{eq:#1}}
		#2
	\end{equation*}
}

\newcommand{\EquationS}[1]{
	\begin{equation*}
		#1
	\end{equation*}
}

\newcommand{\Center}[1]{
	\begin{center}
		#1
	\end{center}
}

\newcommand{\Itemize}[1]{
	\begin{itemize}
		#1
	\end{itemize}
}

\newcommand{\Enumerate}[1]{
	\begin{enumerate}
		#1
	\end{enumerate}
}

\newcommand{\Columns}[4]{
	\begin{columns}
		\begin{column}{#1\textwidth}
		#3
		\end{column}
		\begin{column}{#2\textwidth}
		#4
		\end{column}
	\end{columns}
}

\newcommand{\Table}[4]{
	\begin{table}[h!]
		\centering
		\begin{tabulary}{\textwidth}{#1}
			#4
		\end{tabulary}
		\caption{\label{table:#2} #3.}
	\end{table}
}

\newcommand{\TableT}[4]{
	\begin{table}[h!]
	\centering
		\begin{tabular}{#1}
			#4
		\end{tabular}
		\caption*{\blue{Table \ref{table:#2}:~} #3.}
	\end{table}
}

\usepackage{bytefield}
\newcommand{\y}[2]{\bitbox{#1}{#2}}


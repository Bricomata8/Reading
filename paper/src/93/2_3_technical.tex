\subsection{Technical}

Nowadays, the Web converged over two main protocols,
	namely HTTP/HTTPS and SMTP/SMTPS
Beside these,
	DNS still has a central role for reaching almost any Web server.
In this section we focus on SMTP privacy,
	the next section relies on HTTP privacy.

%%An Effective Defense Against Email Spam Laundering
Many anti-spam techniques have been proposed and deployed to counter email Spam from different perspectives.
Based on the placement of anti-Spam mechanisms,
	these techniques can be divided into two main categories: recipient-oriented and sender-oriented.

\subsubsection{Recipient-oriented Techniques}

This class of techniques either
	(1) block/delay email Spam from reaching the recipient’s mailbox or
	(2) remove/mark Spam in the recipient’s mailbox.
Due to the flourish of techniques in this category,
	we further divide them into content-based and non-content-based sub-categories.

\paragraph{Content-based Techniques}
The techniques in this sub-category detect and filter spam by analyzing the content of received messages,
	including both message header and message body.

\textbf{Email address filters:} Email address filters are simply whitelists or blacklists.
Whitelists consist of all acceptable email addresses and blacklists are the opposite.
Blacklists can be easily broken when spammers forge new email addresses,
	but using whitelists alone makes the world enclosed.

\textbf{Heuristic filters:} The features that are rare in normal messages but appear frequently in spam,
	such as nonexisting domain names and spam-related keywords,
	can be used to distinguish spam from normal email.

\textbf{Machine learning based filters:} Since spam detection can be converted into the problem of text classification,
	many content-based filters utilize machine-learning algorithms for filtering spam.
As these filters can adapt their classification engines with the change of message content,
	they outperform heuristic filters.

\paragraph{Non-content-based Techniques}
The techniques in this sub-category use non-content spam characteristics,
	such as source IP address,
	message sending rate,
	and violation of SMTP standards,
	to detect email spam.

\textbf{DNSBLs:} DNSBLs are distributed blacklists,
	which record IP addresses of spam sources and are accessed via DNS queries.
When an SMTP connection is being established,
	the receiving MTA (Mail Transfer Agent) can verify the sending machine’s IP address by querying the subscribed DNSBL.
Mail server records the number and frequency of the same email sent to multiple destinations from specific IP addresses.
If the number and frequency exceed thresholds,
	the node with the specific IP address is blocked.
Even DNSBLs have been widely used,
	their effectiveness and responsiveness \cite{jung_empirical_2004, ramachandran_can_2006} are still under study.

\textbf{MARID:} MARID (MTA Authorization Records In DNS) is a class of techniques to counter forged email addresses by enforcing sender authentication.
MARID is also based on DNS and can be seen as a distributed whitelist of authorized MTAs.
Multiple MARID drafts have been proposed,
	some of them (SPF and DKIM) are deployed in real world \cite{spf:_2018, BibEntry2014Dec}.
	PGP and S/MIME are also

\textbf{Challenge-Response (CR):} CR is used to keep the merit of whitelist without losing important messages.
To add a sender email address in the whitelist,
	senders are requested a challenge that needs to be solved by a human being.
After a proper response is received,
	the sender’s address can be added into the whitelist.

\textbf{Cryptographic:}
Pretty Good Privacy (PGP) \cite{pgpaghiles2007} and S/MIME are both cryptographic approaches that sign the message body using public-key cryptography and append the signature in the body.
In PGP,
	Keys are stored in end-user key rings or in public key-servers.
Key management uses a peer-to-peer web of trust architecture.
Whereas in S/MIME,
	management follows a hierarchical model similar to SSL and keys are signed by a certificate authority.

\textbf{Delaying:} As a variation of rate limiting,
	delaying is triggered by an unusually high sending rate.
Most delaying mechanisms are applied at receiving MTAs.

\subsubsection{Sender-oriented techniques}

To effectively deny spam at the source,
	ISPs and ESPs (Email Service Providers) have taken various measures to manage the usage of email services.
For example,
	message submission protocol \cite{BibEntry1998Dec} has been proposed to replace SMTP,
	when a message is submitted from an MUA (Mail User Agent) to its MTA.

%COAT
The proposed work in \cite{ahmad_coat_2012} differs from the other techniques in a way that all of them categorize mail messages at receiver side,
	whereas COAT works at the sender side and reduces outgoing spam rather than inbox spam.
%We have hardly found any work in literature about saving the Internet bandwidth and resource wastage by spam.

\textbf{Cost-based approaches:} Borrowing the idea of postage from regular mail systems,
	many cost-based techniques attempt to shift the cost of thwarting spam from receiver side to sender side.

All these techniques assume that the average email cost for a normal user is trivial and negligible,
	but the accumulative charge for a spammer will be high enough to drive them out of business.

Cost concept may have different forms in different proposals.
Bonded Sender \cite{BibEntry2018May} advocates associating email with real money.
%Both centralized \cite{krishnamurthy_shred_2004} and distributed \cite{walfish_distributed_2006} cost enforcement mechanisms have also been proposed.

\subsection{Technical HTTP}

%Privacy leakage vs. Protection measures: the growing disconnect
In this section we enumerate existing privacy protection measures available to users and one new protection proposal \cite{mayer-do-not-track-00}.

%Blocking requests to targeted third parties
\textbf{Blocking requests to targeted third parties:}
This block measure includes using an advertisement blocking tool (AdBlock Plus \cite{adblockplus}) to syntactically block selected third parties via server/domain name.
Another measure blockhidden \cite{krishnamurthy_privacy_2009} determines the true source of hidden third-parties by examining their authoritative DNS servers.

%Refusing cookies to prevent tracking
\textbf{Refusing cookies to prevent tracking:}
Browsers can be set to refuse all cookies (nocook) or just third-party cookies (no3rdcook).

%Disabling script execution
\textbf{Disabling script execution:}
JavaScript execution can be disabled (nojs) either permanently via the browser or selectively via a tool such as NoScript \cite{NoScript}.

%Filtering protocol headers
\textbf{Filtering protocol headers:}
This is done via extensions or at an intermediary and includes the referrer measure available in some browsers to modify or remove the Referer header in an HTTP request.

%Anonymizing the user and user action
\textbf{Anonymizing the user and user actions:} One such anon measure is anonymizing user’s IP address via an anonymizing proxy or by using Tor.

%Do-Not-Track HTTP header proposal
\textbf{Do-Not-Track HTTP header proposal:}
Researchers proposed,
	in early 2010,
	that browsers add a HTTP DoNot-Track-Header (DNT-Header) \cite{mayer-do-not-track-00} to allow users to express their interest in not being tracked by any aggregator or ad network.
However,
	the extent to which third parties would honor such a header is unknown.



\section{Introduction} \label{sec:Introduction}

% Needs, Requirements, Constraints by Statistics (2-3 lines)
The need of Low power wide area networks (LPWAN) increased significantly these five last years.
The main factor is that IoT devices require low power consumption to transmit data in a wide area.
Lora,
	Sigfox and NB-IoT are the most known technologies that satisfy these requirements.
Applications like smart building and smart environment are one of hundreds use cases that need to be deployed with such technologies.
Unlike Sigfox and NB-IoT,
	Lora is more open for academic research because the specification that governs how the network is managed is relatively open.
LoRa is a wireless modulation technique that uses Chirp Spread Spectrum (CSS) in combination with Pulse-Position Modulation (PPM).
The transmission could be configured with 4 parameters:
	\ac{SF},
	\ac{Tx},
	\ac{CR} and \ac{BW},
	to achieve better performance.
% Difficulties ans Challenges (1-2 lines)
The main LPWAN research directions are about link optimization, adaptability and large scale networks to support massive number of devices.
The selection of an appropriate transmission parameter for IoT networks typically depends on the nature of the application.
% Clear Problem (4-5 lines)
Thus heterogeneous transmission configuration and \ac{SF} allocation strategies need to be studied.
In this paper,
	we investigate the performance of heterogeneous networks (i.e.
when each IoT device selects its LoRa transmission parameters according to its link budget and the application requirements).
% Clear Contribution (3-4 lines)
For that purpose we have developed a LoRa transmission adaptation mechanism,
	based on both ns-3 simulator and the Low cost Lora Gateway \cite{lowcostloragateway}.
The computation tasks of the selection process will run on the Gateway device (Raspberry-pi) and the required settings will be sent to nodes for the next transmission.

% Experimentation & results (4-5 lines)
% \red{Simulation results show the performance comparison in terms of reliability,
% 	network capacity and power consumption for homogeneous and heterogeneous deployments as a function of the number of nodes and the traffic intensity.
% The comparison shows the benefits of the heterogeneous deployment where each node selects its configuration according to its link budget.
% }
% The structure 
This paper is organized as follows.
% Introduction
Section \ref{sec:Related work} elucidates summary of related works.
% Background
% Section \ref{sec:Background} provide the required background.
% Approach
In section \ref{sec:Approach}, we propose our approach to solve LoRa parameter selection problem.
% Experimentation
Our experimentation in a real environment is presented in section \ref{sec:Experimentation}.
% Results
% Our findings are presented in section \ref{sec:Results}.
% Conclusion
Section \ref{sec:Conclusion} concludes this paper.


% \begin{tikzpicture}
% \placetextbox{0}{0}{
% 	\mywhiteblackbox{
% 	\begin{tabular}{ll}
% 	ii fg fgf & fgfg fgf \\
% 	fgf fgf & fgfg fgfg 
% 	\end{tabular}
% 	}
% }
% \end{tikzpicture}


% \subsection{Context}% Current needs

% \subsection{Problem statement}

% \subsection{Purpose (Goal)}

% \subsection{Challenges}

% \subsection{Method}

% \subsection{Structure}


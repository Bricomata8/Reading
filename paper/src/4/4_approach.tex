\section{Methodology} \label{sec:Approach}


A generic scheme to solve the configuration selection problem and any other similar selection problem is given in Figure 1.
The genetic selection scheme consists of three main steps,
	the first step contains a set of small parallel fuzzy logic (FL)-based subsystems,
	the second step is a multiplecriteria decision making (MCDM) system,
	and the third step is a genetic algorithm (GA)-based component to assign a suitable weight for the criteria in the second component.
The scheme decision phase can be described in more detail as follows.





(i) The heterogeneous wireless environment contains up to n networks (RAT 1 ,RAT 2 ,...,RAT n ) and the framework has to select the most promising one or to rank the RATs according to their suitability.

(ii) The selection depends on multiple criteria up to i (c 1 ,c2,...,c i ).
Different type of criteria can be measured from different sources to cover the different view points of the users,
	the operators,
	the applications,
	and the network conditions.
Each criterion is measured then passed to its FL-based control subsystem in the first component.

(iii) Every FL-based subsystem gives an initial score for each RAT that reflects the suitability of that RAT according the FL subsystem criterion.
The different sets of scores (d 1 ,d2,...,d i ) are sent to the MCDM in the second component.

(iv) The GA component assigns a suitable weight (w 1 ,w2,...,w i ) for each initial decision according to the objective function that is specified by the operator according to the importance and sensitivities of ANS criteria to the different characteristics of a wireless heterogeneous environment.

(v) Using the initial scores coming from the first component and the weights that are assigned manually or using the third component,
	the MCDM will select the most promising AN or will rank the available RATs according to their suitability.



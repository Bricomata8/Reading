\section{Introduction} \label{sec:Introduction}

% Needs, Requirements, Constraints by Statistics (2-3 lines)
The need of Low power wide area network (LPWAN) networks increased quickly these last years.
The main factor is that IoT devices require low power consumption to transmit data in a wide area.
Lora,
	Sigfox and NB-IoT are the most known technologies that satisfy these requirements.
Applications like smart building and smart environment are one of hundreds use cases that need to be deployed with such technologies.
Unlike Sigfox and NB-IoT,
	Lora is more open for academic research because the specification that governs how the network is managed is relatively open.
LoRa is a wireless modulation technique that uses Chirp Spread Spectrum (CSS) in combination with Pulse-Position Modulation (PPM).
The transmission could be configured with 4 parameters:
	Spreading factor (SF),
	Transmission power (Tx),
	Coding rate (CR) and Bandwidth (Bw),
	to achieve better performance.
% Difficulties ans Challenges (1-2 lines)
The main LPWAN research directions are about large scale networks to support massive number of devices,
	interference issues,
	link optimization and adaptability.
% Clear Problem (4-5 lines)
Thus heterogeneous network deployments and Spreading Factor (SF) allocation strategies need to be studied.
In this paper,
	we investigate the performance of homogeneous networks (i.e.
when all the nodes select the same LoRa configuration) and heterogeneous networks (i.e.
when each node selects its LoRa configuration according to its link budget or their needs) for large scale deployments (up to 10000 nodes per gateway).
% Clear Contribution (3-4 lines)
For that purpose we have developed a LoRa Module,
	based on improved WSNet simulator,
	including a spectrum usage abstraction,
	the co-channel rejection due to the quasi-orthogonality of SFs and the gateway capture effect.
% Experimentation & results (4-5 lines)
Simulation results show the performance comparison in terms of reliability,
	network capacity and power consumption for homogeneous and heterogeneous deployments as a function of the number of nodes and the traffic intensity.
The comparison shows the benefits of the heterogeneous deployment where each node selects its configuration according to its link budget.

% The structure 
This paper is organized as follows.
Section \ref{sec:Related work} elucidates summary of related works.
% Section \ref{sec:Background} provide the required background.
In section \ref{sec:Approach}, we propose our ... to ....
Section \ref{sec:Experimentation} evaluates the performance of our ... in terms of packet delivery ratio,
	throughput,
	and power consumption.
% Our findings are presented in section \ref{sec:Results}.
Section \ref{sec:Conclusion} concludes the article and gives some ideas for future work.

% \subsection{Context}% Current needs

% \subsection{Problem statement}

% \subsection{Purpose (Goal)}

% \subsection{Challenges}

% \subsection{Method}

% \subsection{Structure}

